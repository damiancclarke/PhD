\begin{figure}[htpb!]
\centering
\begin{subfigure}{.5\textwidth}
  \centering
  \includegraphics[scale=0.53]{\twinloc/figures/ferttrend_35_all.eps}
  \caption{Trends in Fertility}
  \label{TWINfig:fertrend}
\end{subfigure}%
\begin{subfigure}{.5\textwidth}
  \centering
  \includegraphics[scale=0.52]{\twinloc/figures/eductrend_all.eps}
  \caption{Trend in Education}
  \label{TWINfig:eductrend}
\end{subfigure}
\caption{Education and Fertility}
\label{TWINfig:trends}
\floatfoot{Note to figure \ref{TWINfig:trends}: Cohorts are made up of all individuals 
from the DHS who are over 35 years (for fertility), and over 15 years (for education).  
In each case the sample is restricted to those who have approximately completed fertility 
and education respectively.  Full summary statistics for these variables are provided 
in table \ref{TWINtab:sumstats}, and a full list of country and survey years are 
available in table \ref{TWINtab:countries}.}
\end{figure}
\vspace{1cm}

\begin{figure}[htpb!]
\begin{center}
\caption{Proportion of Twins by Birth Order}
\label{TWINfig:bord}
\includegraphics[scale=0.86]{\twinloc/figures/twinbybord.eps} 
\vspace{-8mm}
\floatfoot{Note to figure \ref{TWINfig:bord} The fraction of twin births are calculated
from the full sample of DHS data.  The solid line represents the average fraction of 
twins in the full sample (1.85\%), while the dotted line presents twin frequency by 
birth order.  The dotted line joins points at each birth order $\in \{1,\ldots,10\}$.  
The fraction of singleton births is $1-$frac(twin).}
\end{center}
\end{figure}

\begin{figure}[htpb!]
\begin{center}
\caption{Twin Births and Total Fertility}
\label{TWINfig:births}
\includegraphics[scale=0.92]{\twinloc/figures/famsize.eps} 
\end{center}
\vspace{-8mm}
\floatfoot{Note to figure \ref{TWINfig:births} Densities of family size come from the
full sample of DHS data.  Kernel densities are plotted, and present the frequency of
the total number of children per family by family type.}
\end{figure}

\begin{figure}[htpb!]
\begin{center}
\caption{Proportion of Twins of All Births (USA)}
\label{TWINfig:USTwin}
\includegraphics[scale=0.92]{\twinloc/figures/USTwinFLE.eps} 
\end{center}
\end{figure}

\begin{landscape}
\begin{figure}[htpb!]
\begin{center}
\caption{Intra- and Inter-country trends: height and twinning}
\label{TWINfig:arrows}
\includegraphics[scale=1.5]{\twinloc/figures/height_country.eps} 
\end{center}
\end{figure}
\end{landscape}

\begin{figure}[htpb!]
\begin{center}
\caption{Relaxing Strict Exogeneity (two plus)}
\label{TWINfig:ltz2}
\includegraphics[scale=0.84]{\twinloc/figures/LTZ_two.eps}
\end{center}
\end{figure}

\begin{figure}[htpb!]
\begin{center}
\caption{Relaxing Strict Exogeneity (three plus)}
\label{TWINfig:ltz3}
\includegraphics[scale=0.84]{\twinloc/figures/LTZ_three.eps} 
\vspace{-8mm}
\floatfoot{Note to figure \ref{TWINfig:ltz2}-\ref{TWINfig:ltz3}: Confidence 
intervals and point estimates are calculated according to \citet{Conleyetal2012}
using DHS data.  
Estimates reflect a range of priors regarding the validity of the exclusion 
restriction required to consistently estimate $\hat\beta_{fert}$ using twinning in 
a 2SLS framework.  The local to zero (LTZ) approach applied here assumes that 
$\gamma$, the sign on the instrument when included in the first stage, is 
distributed $\gamma\sim U(0,\delta)$.  The vertical dashed line indicates 
2$\times\hat\gamma$, the point at which the estimate for $\gamma$ lies precisely 
halfway between [0,$\delta$]. Further discussion is provided in appendix
\ref{TWINscn:gamma} and table \ref{TWINtab:Conley}.}
\end{center}
\end{figure}

\begin{figure}[htpb!]
\begin{center}
\caption{Plausibly Exogenous Bounds: School Z-Score (USA)}
\label{TWINfig:PEx-USA}
\begin{subfigure}{.5\textwidth}
  \centering
  \includegraphics[scale=0.5]{\twinloc/figures/ConleyUSA_EducationZscore_two.eps}
  \caption{Two Plus}
  \label{TWINfig:PEx-USA2}
\end{subfigure}%
\begin{subfigure}{.5\textwidth}
  \centering
  \includegraphics[scale=0.5]{\twinloc/figures/ConleyUSA_EducationZscore_three.eps}
  \caption{Three Plus}
  \label{TWINfig:PEx-USA2}
\end{subfigure}
\end{center}
\floatfoot{Note to figure \ref{TWINfig:PEx-USA}: See notes to figure
\ref{TWINfig:ltz3}. An identical approach is employed, however now using USA 
(NHIS) data.}
\end{figure}

\begin{figure}[htpb!]
\begin{center}
\caption{Plausibly Exogenous Bounds: Excellent Health (USA)}
\label{TWINfig:HPEx-USA}
\begin{subfigure}{.5\textwidth}
  \centering
  \includegraphics[scale=0.5]{\twinloc/figures/ConleyUSA_excellentHealth_two.eps}
  \caption{Two Plus}
  \label{TWINfig:HPEx-USA2}
\end{subfigure}%
\begin{subfigure}{.5\textwidth}
  \centering
  \includegraphics[scale=0.5]{\twinloc/figures/ConleyUSA_excellentHealth_three.eps}
  \caption{Three Plus}
  \label{TWINfig:HPEx-USA2}
\end{subfigure}
\end{center}
\floatfoot{Note to figure \ref{TWINfig:HPEx-USA}: See notes to
figure \ref{TWINfig:ltz3}. An identical approach is employed, however now using USA 
(NHIS) data.}
\end{figure}

