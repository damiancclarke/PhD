%-------------------------------------------------------------------------------
Fertility decisions are fundamental to human welfare. As animals we are driven 
to reproduce, while a deep desire for the wellbeing of our offspring promotes 
investments to maximise their chances of success.  These behaviours play out 
continually over our lives: first as the subjects of fertility choices, and 
later as their architects.

This thesis examines fertility choices and their effects on human well-being.
It joins an enormous range of literature, both academic and non-academic, and
from economics as well as other fields.  Particularly, this thesis is a 
collection of papers which discuss how to estimate the causal effects of
fertility decisions on the individuals and families which make them.  The 
estimation of causal effects in an area of life (child-bearing) that is related 
to and affects many other areas of life is not trivial.  However, in the 
existing economic literature a number of techniques have been developed to 
circumnavegate the difficulties involved in the estimation of causal links
between child-bearing and other human outcomes.  In the chapters contained in 
this thesis I provide extensions and caveats to these techniques.

The body of this thesis consists of four chapters.  Three of these are central
to the thesis and contain new empirical and theoretical considerations and 
results.  The fourth provides a survey of the state of the art,
along with a much more precise definition of the terminology and field in
which this thesis is located.  This overview chapter is the first chapter,
and a useful starting point for a reader interested in quickly gaining an
understanding of the methodologies which are commonly employed to estimate the
causal effects of fertility, and the empirical results in the literature.

%-------------------------------------------------------------------------------
The research contribution of this thesis begins with chapter \ref{chap:twins}.
This chapter focuses on twins, and their use as an instrumental variable. The
prevailing logic is that twin births are (at least conditionally) randomly 
assigned to women who conceive.  If this is the case, twins are the perfect
example of an instrumental variable: they have strong effects on completed
fertility, while being unrelated to all other parental and family 
characteristics.  This has given rise to a number of influential papers using
twin births to estimate the effect of fertility at the extensive margin on
women's and men's labour supply, as well as sibling quality.

Chapter \ref{chap:twins} documents that the assumption of conditional 
exogeneity of twin birth does not hold in the many cases examined.  It is 
argued (and demonstrated) that healthier women are more likely to give birth 
to twins.  It is shown that conditional on twin conception (which may or may 
not be random), healthier mothers are more likely to take twins to term. This 
is shown to be true prior to the introduction of assisted reproductive 
technologies like IVF, and to hold in a large number of contexts, including 
administrative data from the USA, UK, and Spain, and  survey data from Africa, 
Asia and Latin America.

The nature of this violation of exogeneity is problematic.  We document that
healthy mothers are more likely to give live birth to twins.  Maternal health
is very hard (or impossible) to fully capture in data.  In various contexts
we show that health as proxied by height, alcohol and drug consumption prior
to birth, maternal stress, smoking, and access to prenatal care all predict
twinning.  It seems unlikely that all such measures (and the many which we
don't observe) could ever be fully observable in data.  What's more, these
measures are very likely to be correlated with unobserved factors which predict
the very outcomes we are interested in analysing (labour force participation
or child quality).

In order to illustrate the problems involved in this methodology we turn to
the quantity-quality (QQ) trade-off.  Recent empirical analysis has suggested 
that an additional birth within a family may have very little causal impact 
on existing children in a family.  As we demonstrate, any two-stage least 
squares estimates which ignore the role of maternal health in twinning are
likely to attenuate estimates of the QQ trade-off.  We partially correct for
this bias, and show that upon partial correction, a significant trade-off
does begin to emerge in data from the developing world, as well as in data
from the USA.

The estimates we produce are at best lower bounds of the trade-off. In closing
the chapter we turn to ways to tighten the bounds on the estimate of the effect 
of interest.  We use partial identification techniques to bound both IV as well 
as OLS estimates, and then use our IV and OLS estimates (which are likely to 
suffer different biases) to find the range in which the QQ trade-off lies.

%-------------------------------------------------------------------------------
Chapter \ref{chap:pill} turns to the effect of contraceptive technologies on 
women's outcomes.  I focus on the effect of the emergency contraceptive pill in
Chile.  This is a particularly interesting context given the high rates of 
adolescent (and unplanned) pregnancy, and the absence of any other legal post-%
coital birth control technologies.  The arrival of the emergency contraceptive
pill heralded a large shift in the ability to post-coitally contracept, which
previously was illegal and risky.

The few existing studies of the effects of the emergency contraceptive pill
(which have been conducted in the USA) on fertility suggest that its role may be 
minor when considering rates of pregnancy.  However, these studies are conducted 
in considerably different circumstances: namely, where free or cheap options to
interrupt pregnancy already existed (via surgical abortion).  The evidence from 
Chile, however, suggests that in the absence of alternative post-coital 
technologies, the emergency contraceptive pill can act as an important technology 
in reducing rates of pregnancy in the short term.

In order to identify the effect, I focus on the staggered arrival of the 
emergency contraceptive pill to the country. Following juridical and 
constitutional rulings in 2008, the decision of whether or not to freely 
distribute the emergency contraceptive pill in municipal health centers was 
left in the hands of the mayor of each municipality. This interim legislation
(which existed from 2008 until 2011) resulted in a situation in which 
approximately half of the municipalities stated that they would prescribe the 
pill, while the remainder did not. In this period, over 15,000 pills were 
freely prescribed and issued to women.

I document the effect of this reform on rates of child-bearing using a flexible 
difference-in-differences (DD) methodology. Na\"ive DD estimates suggest a 4.5\% 
reduction in rates of teen birth, and a 3\% reduction for women aged between 
20-34.  I also document the effects of the reform on rates of fetal death.  DD 
estimates suggest that the reform had important effects on rates of early term 
fetal deaths reported by hospitals, but no similar effect on late term fetal 
deaths. Although a reduction in fetal deaths in morning after pill 
municipalities may be an indicator of many things, the fact that this reduction
\emph{only} occurs early in gestation is suggestive of a reduction in illegal
and risky abortions.

In closing the chapter I examine the possibility that women may have crossed
municipal boundaries (defying their treatment status) to access the emergency
contraceptive pill. I demonstrate that there is evidence of reform `spillovers'
which may travel up to 30km from the nearest treatment municipality, and which
have important impacts on estimated treatment effects.

%-------------------------------------------------------------------------------
I return to the idea of spillovers much more extensively in chapter 
\ref{chap:spill}---the final content chapter of the thesis.  Frequently in the
economic literature, policy analysis is undertaken using a DD methodology. This
is particularly the case where a reform arrives to certain areas or groups 
within a country but not others, allowing for a clear classification between
treatment and control individuals.  This methodology relies explicitly on the
stable unit treatment value assumption (SUTVA), which, among other things,
implies that no spillovers can occur between treatment and control areas.

This is at odds with the incentives put in place in many policy reforms. Where
policies are geographically (or otherwise) bounded, and where the policy offers
clear benefits to those in treatment regions, nearby individuals will have the
incentive to access treatment, violating their status, and the SUTVA. I propose
a methodology to estimate causal treatment effects in the absence of SUTVA.
Rather, I outline a weaker set of conditions, where this assumption need only 
hold between \emph{some} units.  I derive a full set of conditions under which 
the average treatment effect on the treated (ATT) and the average treatment 
effect on the close to treatment (ATC) can be estimated in the presence of
local spillovers.

This spillover-robust DD methodology is demonstrated by examining two 
contraceptive reforms.  Contraceptive reform is a perfect example of such a 
situation given that the costs of having an undesired pregnancy are very high, 
the arrival of new contraceptive technology to certain areas is often slowed by 
legal challenges, and violation of treatment status is possible if individuals 
from non-covered areas can travel and convince health care providers in covered 
areas to prescribe treatment.  This is shown by using a 2007 abortion reform in 
Mexico DF, and the aforementioned emergency contraceptive pill reform in Chile
in 2008.

In both cases of sub-national reform it is shown not only that the policy 
reduced rates of teenage pregnancy treatment areas, but also that reductions
occured in areas `close to' treatment.  In the case of Mexico DF (where 
abortion was only available in one geographical area of the country), the 
existence of spillovers is limited only to areas local to the state in question.
These spillovers are shown to \emph{not} have significant effects on traditional
DD estimates.  In the case of Chile, however, where treatment was geographically
disperse, spillovers are seen, and once these spillovers are properly taken into
account, the estimated treatment effect is \emph{larger} than that estimated 
when SUTVA is assumed to hold.
%-------------------------------------------------------------------------------

Overall then, this thesis seeks to push forward the applied microeconometric 
literature on the estimation of fertility and fertility control.  As more and
more data on babies and long-term outcomes becomes available, significantly
more precise effects of fertility on human outcomes can be estimated.  However,
as this thesis documents, even with considerable amounts of information and
carefully designed estimation strategies, the assumptions underlying our 
methodologies should be tested demandingly.  Fertility decisions are complex, 
and economic estimates must take account for this complexity when purporting to
report causal estimates for policy and academics.
