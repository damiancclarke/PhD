%-------------------------------------------------------------------------------
\begin{chapabstract}
Child-bearing decisions are not made in isolation.  They are taken in concert 
with decisions regarding work, marriage, health investments and stocks, as well 
as many other observable and non-observable considerations.  Drawing causal 
inferences regarding the effect of child birth is complicated by these 
endogenous factors.  This review chapter outlines the identification of causal
estimates of fertilty, and the assumptions underlying the range of estimators
and methodologies proposed in the economic, as well as the non-economic, 
literature.
\end{chapabstract}

\newpage
\section{Introduction}
Human decisions regarding births, and how these decisions affect individual 
outcomes, are central to human welfare.  They are also widely relevant to the
world population.  In 2014, 18.7 per every 1,000 people had a child, which is 
the equivalent of 4.3 births per second \citep{CIA2014}. Over the course of her 
lifetime, the average woman in 2013 will have 2.46 births, down from 4.98 in 
1960 \citep{WorldBank2015}. The importance of choice, and control over the 
timing and number of children has been documented as early as \emph{c}1800 BC, 
with discussions of a range of contraceptive methods included in the Kahun 
Gynaecological Papyrus \citep{OdowdPhillip1994}.  Estimates suggest that 
between then and today,
contraceptive use has grown to reach global coverage of 60\% of all married,
fertile-aged women \citep{Darroch2013}. This review chapter examines the effect 
of childbirth and birth timing decisions on human outcomes.

I review the literature on the effect of individual fertility behaviours on
individual outcomes: namely, the microeconomic effects of fertility.  This
chapter, and indeed the literature on which it is based, largely focuses on 
the effect of a woman or family's fertility decisions on the mother's life 
outcomes and the outcomes of her children.  Some, although relatively less, 
focus is paid to the effect on her partner (if existing and present).  

Rather than focus on correlations between fertility and other outcomes, this 
article is centered on the \emph{causal} analysis of the effects of fertility.  
I discuss the theoretical and empirical considerations required to infer causality 
in a behaviour which cannot be manipulated directly in an experimental context.  
Given the interruptive nature of child birth on a large range of other life 
outcomes, any study of causal effects must isolate changes in fertility from 
corresponding changes in simultaneously determined, or dynamically dependent, 
outcomes.  As a simple example, if women jointly choose to exit the labour 
market and have a child, any inference regarding the effect of fertility on her 
or her child's \emph{other} outcomes must be independent of her labour market 
choice.

Since the boom in fertility related research in microeconomics in the mid-%
1970s,\footnote{From 1970 onwards, the frequency of the occurrence of the word 
``fertility'' 
in titles of all articles published in the \emph{Journal of Political Economy} 
is 46.5 per 100,000 words. Prior to the 1970s, the frequency was 1.45 per 
100,000 words. Though admittedly based on a small sample, the frequency by 
decade is 11.17 (`60s), 72.72 (`70s), 66.33 (`80s), 29.41 (`90s), and 41.07 
(`00s). It was not mentioned in titles of articles published between 1863 and 
the 1960s.} a range of methodologies have been proposed to permit inference 
in precisely these circumstances. These include the use of instrumental 
variables (IV), combining difference-in-difference (DD) with IV, the use of 
quasi-experimental fertility shocks, or trying to artificially construct a 
treatment and control group using family members or other matching 
methods.  In the sections which follow, I describe these methodologies, the 
papers which propose and estimate them, and the identifying assumptions that
motivate causality in each case.

Empirical considerations regarding the causal effects of fertility require the
consideration of (at least) three questions: `Effects on what?', `Effects at
what margin?', and `Effects of what? (timing or quantity)'.  Frequently, these 
questions are split further, and examined for specific groups of women or 
children.

Regarding the first question, the theoretical and empirical microeconomic 
literature has hypothesised that marginal births may have causal implications 
for many individual-level outcomes.  This includes effects on children: their 
health indicators, cognitive and non-cognitive achievements, long-term labour 
market outcomes, education inputs, and social outcomes such as age at marriage 
and crime incidence; as well as effects on parents. Parental outcomes often 
considered are centered around labour market participation and returns, rates 
of education completion, marriage market outcomes and socioeconomic indicators 
such as welfare-receipt.  I outline these measures in the following sections, 
along with the evidence that a causal link exists (or does not exist) with 
fertility.

Human fertility decisions exist at two (very different) margins.  The choice of
whether or not to have a child (the extensive margin), and, conditional on 
choosing to have any children, the decision regarding \emph{how many} births
to have (the intensive margin).  The nature of links between fertility decisions
and outcomes vary considerably when considering the intensive and the extensive
margin.  The consensus in the literature is that the causal effects are 
certainly not a linear function of births, with important non-linearities, and 
indeed non-monotonic relationships, described between fertility and (some) of 
the previously mentioned outcomes. In order to quantify effects at different 
margins, a range of estimation samples and methodologies need be employed.  

Finally, causal effects of fertility are not independent of mother's age at 
birth.  Often, rather than estimating the effect of a marginal birth, we will be
interested in determining the effect of child birth at a particular age (such as
during adolescence).  Considerations of these effects have important life-cycle
implications for future investment decisions.

The study of fertility is common to a huge range of fields: social sciences, 
physical sciences, demography and medicine, and in many subfields within 
disciplines.  As a result, any moderate-length review of the literature must
be necessarily pointed.  As such, this chapter firmly focuses on the effects
of fertility on other individual-level outcomes, and not the determinants of
fertility\footnote{More recently, \citet{KearneyLevine2012} provide an extensive 
discussion of teenage childbearing and its determinantes in the USA, 
\citet{Bailey2013} reviews the old and new evidence on the effect of access to 
contraception, and \citet{Moffitt2005} has provided a discussion and application 
of causal inference to the effects of teenage childbearing on child outcomes. All
provide extremely useful reviews of the relevant literature.  The handbook chapter
of \citet{Schultz2008} is perhaps the definitive reference for microeconomists
interested in an analysis with a very broad scope.  There are many papers
discussing education and fertility (ie \citet{Blacketal2008}), which won't be 
discussed in this chapter.}, the macroeconomic effects \citep{Enke1966,Enke1971}, 
or discussions of the broader effects of population control policies
\citep{GMiller2009,RosenzweigWolpin1986}.

In what remains of this chapter I briefly layout a number of considerations
surrounding causal analysis of endogenous fertility decisions, before turning to
particular methodologies and samples used to generate empirical estimates.  I
consider the causal effects of fertility on child and on parental outcomes in
turn, outlining main results from relevant papers in the field.

%-------------------------------------------------------------------------------
\section{Causality and Fertility}
\label{Fscn:causality}
\subsection{A Framework}
Consider an outcome $Y_i$, for each member $i$ of a sample, $i \in \{1,\ldots,
N\}$, where $Y$ denotes an outcome variable of interest, and the sample is drawn
from a population of child-bearing mothers (or, as discussed later, their 
children). We are 
interested in determining the effect of manipulations of fertility, which we
denote $F_i$, on our outcome variable of interest $Y_i$.  It is assumed that 
$Y_i$ is a function of fertility, an unobserved variable $U_i$, and a series of 
other variables which are summarised as the error term $\varepsilon_Y$:
\begin{equation}
\label{Feqn:outcome}
Y_i=f_Y(F_i,U_i,\varepsilon_Y).
\end{equation}
Fertility is assumed to be a function of the unobserved $U_i$, and stochastic 
$\varepsilon_F$:
\begin{equation}
\label{Feqn:fert}
F_i=f_F(U_i,\varepsilon_F),
\end{equation}
and finally $U_i=f_U(\varepsilon_U)$. These error terms $\bm\varepsilon$ are 
assumed mutually independent. To fix ideas, we could consider an outcome variable 
$Y_i$ as average years of education of $i$'s children, $F_i$ as completed 
fertility, and $U_i$ as unobserved positive health behaviours of the mother. By 
iterative substitution of the $\bm\varepsilon$ terms into (\ref{Feqn:outcome}) 
and (\ref{Feqn:fert}) it becomes apparent (in the defined system of equations) 
that changes in fertility and health are unrelated to $\varepsilon_Y$, but that 
both average years of education and fertility are related to unobserved maternal 
health behaviours.\footnote{\emph{Ie} $F_i$ and $U_i$ are not functions of 
$\varepsilon_Y$, however $Y_i$ and $F_i$ are functions of $\varepsilon_U$.}

In causal terms as per \citet{Haavelmo1943,Haavelmo1944} (and particularly, the
recent exposition in \citet{HeckmanPinto2015}), we are interested in the change
in $Y$ resulting from the hypothetical manipulation of fertility $F$, while 
other elements of the system of equations ($U,\bm\varepsilon$) remain unchanged.  
We define $b$ as a particular draw of $F$, and are thus interested in the causal 
effect of manipulating fertility from $b$ to $b+1$, which throughout this paper
we will call $\beta$:\footnote{For now we make no distinction between different 
values of $b$ in (\ref{Feqn:estimand}).  Generally however, we will be interested 
in at least two seperate situations. The first, comparing having any children 
to having no children (the extensive decision), while the second refers to 
having $b+1$ children versus having $b$ children for $b\in{1,\ldots,k}$ (the 
intensive margin).  We return to discussions of $b$ for different parities when
discussing empirical results in the sections which follow.}
\begin{equation}
\label{Feqn:estimand}
\beta\equiv E_{U_i,\varepsilon_Y}[Y_i(b_i+1)-Y_i(b_i)]
\end{equation}

The hypothetical manipulation envisioned by \citeauthor{Haavelmo1943} is 
generally not feasible in real-world fertility decisions. And in observational
studies using data over space or time, the presence of factors similar to $U$
considerably hinders the estimation of causal effects.  In the sections which
follow we return to this system of equations, and outline the existing 
techniques which aim to recover causal estimates despite the lack of explicit 
exogenous manipulation of $F$.

%-------------------------------------------------------------------------------
\subsection{Questions and Applications}
Discussions of causality can be entirely agnostic to the deeper questions of
why effects are observed, or why we may want to quantify causal effects at 
all.\footnote{Discussions of how one should estimate causal parameters, and 
the limits of estimation without theoretical underpinnings are a topic of great 
debate, and have been for many years. As early as 1947 in ``Measurement Without 
Theory'', \citeauthor{Koopmans1947} states: 
\begin{quote}
for the purpose of systematic and large scale observation of such a many-sided 
phenomenon, theoretical preconceptions about its nature cannot be dispensed with, 
and the authors do so only to the detriment of the analysis.
\end{quote}
In discussions of parameter estimates, I do not zoom out to examine these
deeper issues.  Many resources discussing inference in economics provide 
fascinating insight into these issues, such as \citet{Keane2010}, 
\citet{Wolpin2013} and references therein.}  Nonetheless, these are precisely 
the reasons that research into these questions is undertaken.  In the first 
instance, the estimation of causal effects can allow us to test well-specified 
models or hypotheses regarding behavioural or biological mechanisms underlying
the effect of fertility choices. A correctly estimated parameter in the 
fertility literature is generally of interest given its relation to deeper 
behavioural or technological implications rather than as a curio in its own 
right.  And in the second case, these behavioural or technological implications 
are extremely relevant for policies implicitly or explicitly designed to impact 
human wellbeing.  I discuss these `whys' in what remains of this section.

A range of theories exists to explain why fertility choices may affect other 
human outcomes.  Microeconomic theories of fertility choice have a number of
roots, including biological,\footnote{For example, the \citet{TriversWillard1973}
hypothesis regarding mother's health, environment, and investment in offspring
has been used to motivate economic applications \citep{AlmondEdlund2007}.}
behavioural, or technological. The earliest work such as \citet{Becker1960} 
approached the problem firmly through the lens of consumer behaviour.  Utility 
maximising individuals (or families) were assumed to choose the number of 
children that they would like to have in the same way that they were assumed to 
decide on other consumption: based on relative prices (or shadow prices), and a 
budget constraint.  In later work, \citep{Becker1965}, the idea of a household 
production function was introduced, in which both monetary and time costs of 
production and consumption were considered to motivate household decisions.

These theories of ``demand for children'' gave rise to theoretical predictions 
regarding household fertility decisions.  Perhaps most central to these is the
quantitly--quality (QQ) trade-off, which posits that an inverse relationship
will exist between child quality and child quantity.  This relationship is
suggested to exist given that children are a special composite ``good'', where 
along with deciding how many are desired (an extensive decision), parents decide 
on how much to invest in child quality (an intensive decision).  I return to 
discuss the QQ trade-off later in this chapter, and, extensively in chapter
\ref{chap:twins} of this thesis.

These demand-based theories which can also incorporate time endowments of 
family members give rise to a number of other hypothesised relationships, 
including an inverse relationship between costs of child bearing (including 
outside labour market options for parents), and the number of children born.  
Concerned with the narrow nature of demand-based theories where all variation 
in fertility is due to parental tastes and the prices they face, 
\citet{Easterlin1975} proposed a more comprehensive theory, in which demand for 
children, the supply of children (the theoretical quantity of children if 
parents do not contracept), and the cost of fertility regulation interact to 
determine completed fertility.\footnote{\citeauthor{Easterlin1975} provides a 
deeper analysis of the machinery behind these decisions.  He discusses the 
``basic'' and ``proximate'' determinants of fertility, which include 
(respectively) socioeconomic conditions, modernisation variables, cultural 
factors and genetic factors; and exposure to intercourse, fecundability, 
duration of postpartum infecundability and the use of fertility control.} This 
theory conserved the most important predictions from Becker and coauthors' 
models, while also opening new considerations regarding the determinants and 
effects of fertility.  Perhaps most importantly, these theories opened the 
door to external determinants of fertility such as fertility control 
technologies as explicit determinants which may have a direct effect over 
fertility outcomes.

The search for the theoretical and empirical implications of fertility choices
and determinants reaches far beyond the academic literature. These effects are
frequently cited in policy documents drafted by governments and international 
organisations when defining, classifying or justifying policy choices which have
the potential to remarkably change fertility choices, and life courses, of 
affected individuals.

There exist a range of proclamations and charters defining reproductive rights
for individuals.  As early as 1968, the United Nations Declaration of Human 
Rights incorporated that ``[p]arents have a basic human right to determine 
freely and responsibly the number and the spacing of their children'' 
\citep{UN1968}, a statement echoed in many subsequent proclamations, 
including the Proclamation form the Cairo Program of Action in 1994, and in the 
current World Health Organisation defintion of reproductive rights. 

Currently, the proportion of countries whose governments explicitly state that
they are trying to alter population levels is very high.  In 2013, of the 66 
countries classified as ``high fertility'' (greater than 3.2 births per woman), 
90\% of these have explicit policies in place to lower fertility rates 
\citep{UN2013}.  In many cases, these policies' stated aim is to allow families
to access desired contraceptive technologies.\footnote{According to 
\citet{BongaartsSinding2011}, approximately 40\% of pregancies in the developing
world are unintended.}  However, in many others, it is well recognised that
high fertility has direct implications for development, including effects on 
educational attainment, and mother and child health \citep{UN2013}.  In the
remainder of this paper, I turn to the underlying estimates estimates which are
(at least partially) used to justify expensive and wide-reaching policies of 
these types.

%-------------------------------------------------------------------------------
\section{The Effects of Family Size on Children}
\label{Fscn:kids}
In 1973, the \emph{Journal of Political Economy} released a special issue 
dedicated to new economic approaches to fertility.  Interest in determining
the causal effect of total fertility on the outcomes of children in the 
household blossomed from the articles it contained.  A common theme in a number 
of articles in this issue \citep{BeckerLewis1973, DeTray1973, Willis1973} 
concerns a family's decisions regarding fertility (the quantity of children) 
and investments in child human capital (the `quality' of children). Abstracting 
from intra-household variations in child quality\footnote{An extensive 
literature exists which looks at \emph{intra}-household endowment and investment
decisions among siblings.  \citet{Behrmanetal1982} provides initial discussion,
and \citet{AizerCunha2012} embed these considerations in a quantity--quality-type 
framework.}, each of the aforementioned articles demonstrates the theoretical 
existence of a quantity--quality (QQ) trade-off.\footnote{As \citet{Willis1973} 
succinctly describes:
\begin{quote}
`Thus, parents not only balance the satisfactions they receive from their
children against those received from all other sources not related to 
children \ldots, but they must also decide whether to augment their 
satisfaction from children at the ``extensive'' margin by having another
child or at the ``intensive'' margin by adding to the quality of a given
number of children.'
\end{quote}
}

The QQ trade-off described in the above series of articles as well as in
\citet{BeckerTomes1976,BeckerTomes1986} owes to the joint entry of quality and 
quantity in the household budget constraint.  As the number of children 
enters in the shadow price of quality, and the quality of desired children 
enters the shadow price for quantity, decisions regarding fertility and quality 
cannot be made in isolation.  Holding all else constant, increases in fertility 
increase the shadow price of quality, and increases in quantity increases the
shadow price of the marginal birth. This considerably complicates causal 
inference. What's more, as recognised in early articles by \citet{%
BenPorathWelch1972,BenPorath1976}, quality decisions may \emph{directly} feed
back to quantity via child mortality.

%-------------------------------------------------------------------------------
\subsection{Observational Data}
Given the aforementioned theoretical structure of the relationship between 
child quality and child quantity, it is apparent that estimating OLS on 
observational data will lead to consistent estimates of $\beta$ only in very
particular circumstances. To see this, we return to equation \ref{Feqn:outcome}.
If we consider standard OLS with a linear model, we re-write 
(\ref{Feqn:outcome}) as:
\[
Y=\beta F + U + \varepsilon_Y,
\]
where we assume that $E[\varepsilon_Y]=0$.  To estimate $\beta$ from
the above, we can consider conditioning on two distinct values of $F$:
\begin{eqnarray}
E[\hat\beta] & = & E[Y|F=b+1]-E[Y|F=b]. \label{Feqn:OLS} \\
                      & = & E[\beta(b+1)+U|F=b+1] - E[\beta(b)+U|F=b] 
\nonumber \\
                      & = & \beta + \{E[U|F=b+1] - E[U|F=b]\} \nonumber
\end{eqnarray}
Thus, (\ref{Feqn:OLS}) is only identical to the causal estimate in 
(\ref{Feqn:estimand}) in a very limited set of circumstances: above this is when
$E[U|F=b+1] = E[U|F=b]$.  This is simply a specific example of 
the well-known OLS requirement that the independent variable of interest ($F$) 
must be uncorrelated with the omitted error term, given that 
$\plim(\hat\beta)=\beta+\Cov(F,U)/\Var(F)$.  Where variation in fertility
in a cross-sectional dataset is correlated with movements of other variables
related to the outcome of interest (which here we summarise as $U$), we will fail 
to identify the true causal effect of fertility given the lack of 
\citeauthor{Haavelmo1943}'s hypothetical manipulation of $F$. 

Due to the limitations laid out in the preceding paragraph, very few papers in 
the literature aim to infer causality by estimating linear models with 
cross-sectional data.\footnote{Early work, such as \citet{Desai1995}, provides 
across-sectional descriptive 
evidence to document correlations, while \citet{Hanushek1992} estimates some
cross-sectional (though value-added) models.}  However many papers which use 
alternative methods to infer causality (discussed in the sections which follow)
estimate OLS as a base specification, which can provide some information on the 
type and degree of bias in OLS.  Beyond recognising that a bias is likely to 
exist, relatively few of these papers provide an explicit discussion of why this
may be.  Notable exceptions include \citet{Qian2009}, who suggests joint
parental preferences for more education and fewer children as well as optimal
stopping rules which depend on the quality of the first child, and 
\citet{Blacketal2010}, who additionally note that family size effects are 
confounded with birth order effects.  Indeed there are a number of reasons one
could use to suggest bias.  These include parental education, discount rates,
maternal health, or network effects driving both fertility and child quality.  
Generally it seems likely that these factors will induce a negative bias in OLS
estimates of the effect of fertility, given that factors which lead to fewer 
births (contraceptive knowledge, opportunity cost of time, aspirations, and so 
forth) also seem likely to drive greater investments in children who are 
eventually born.  However, there is no reason why this theoretically must be the
case, as fertility choices may interact in a range of ways with many different 
unobserved characteristics of mothers or families.  Empirically, although it 
largely seems to be the case that OLS estimates of the effect of fertility are 
lower (more negative) than more credibly causal estimates, there are some cases, 
particularly in more developed countries, where this is not the case. We return 
to provide more details on these estimates in the sections which follow.

%-------------------------------------------------------------------------------
\subsection{Instrumental Variables}
\label{Fsscn:kidIV}
In systems of equations of the type described in \ref{Fscn:kids}, one way to 
drive inference is through the
use of shifters (or instrumental variables) which affect the quantity of one of
the variables without affecting the other.  In order to identify the effect of 
fertility on children's outcomes, this instrumental variable must affect only 
fertility, with no indirect effects on quality.\footnote{\emph{Ie} the exclusion 
restriction must hold, implying that the estimation of the structural equation 
which contains quality on fertilty and the instrument must result in a 
coefficient on the instrumental variable which is precisely equal to zero.}
Returning to the nomenclature introduced in section \ref{Fscn:causality},
consider $Y_i$ as child quality, $F_i$ as child quantity, and the unobserved
$U_i$, all generated as described in \ref{Fscn:causality}.  However, now consider 
the case where a new variable from outside the system is observed, denoted 
$Z_i$.  $Z_i$ is assumed to directly affect $F_i$:
\[
F_i = f_F(U_i,Z_i,\varepsilon_F).
\]
The function \ref{Feqn:outcome} is unchanged, reflecting the fact that the only
channel with which $Z_i$ affects $Y_i$ is through $F_i$ (in other words, the 
exclusion restriction holds). Finally, to close, assume that 
$Z_i=f(\varepsilon_Z)$, and once again the error terms $\bm{\varepsilon}$ are 
assumed mutually independent.

The above situation leads to an explicit way of generating the hypothetical 
variation discussed in section \ref{Fscn:causality}.  By taking advantage of 
variation in $F$ induced by variation in $Z$, the effect of $F$ on $Y$ can
be identified in the absence of any movement in $U$.  The most simple way to
consider this is by observing the Wald estimator:
\begin{eqnarray}
\label{Feqn:Wald}
\hat\beta = \frac{E[Y|Z=1]-E[Y|Z=0]}{E[F|Z=1]-E[F|Z=0]}.
\end{eqnarray}
Here rather than explicitly being based on a movement from $F=b$ to $F=b+1$
estimation is driven by the effect which $Z$ has on $Y$, scaled by the
degree to which it moves $F$.  If the instrument increases birth by exactly
one, then (\ref{Feqn:Wald}) collapses to an expression similar to 
(\ref{Feqn:OLS}).  Fundamentally in strategies of this type, the identifying
assumption shifts from concerns regarding correlations between $U$ and $F$
to correlations between $U$ and $Z$.  Consistent causal estimation now 
requires that $\Cov(U,Z)=0$.

The earliest discussion of these types of shifters and the corresponding 
exclusion restriction required for the estimation of the causal effects of 
fertility was in \citet{RosenzweigWolpin1980}. They point out that if multiple 
births are unanticipated, their occurrence will cause some families to exceed 
their desired fertility, shifting the total number of births in the absence of 
any change in parental considerations of quality investments. This has motivated 
estimation in a number of papers, where twin births are employed as instrumental 
variables.  Twin instruments have been employed in a range of contexts and
to examine various different `quality' outcome variables of children. These 
include \citet{Blacketal2005,Caceres2006,Lietal2008,Dayiogluetal2009,Sanhueza2009,
Blacketal2010,Angristetal2010,FitzsimonsMalde2010} and \citet{SouzaPonczek2012}, 
and focus on child quality measures including years of education, IQ, private 
school enrolment, BMI and height, college completion and age at marriage.  The 
evidence on the existence of a QQ trade-off in these studies is mixed, 
although recent influential results suggest that the evidence in favour of a 
trade-off may be weak.  In table \ref{Ftab:childQQ} I lay out outcome variables, 
contexts, and estimates of $\beta$ presented in the IV literature.

As per the above series of equations, causal estimates rely on the fact that
$Z$ truly is independent of $U$.  This has been questioned in a number of ways.
\citet{RosenzweigZhang2009} suggest that the close birth-spacing of twins, and 
the fact that twins have lower health stocks at birth \citep{Almondetal2005}
means that parents may change behaviours to reinforce or compensate for intra-%
household human capital differences.  While this can be tested directly, it 
requires data on early life human capital endowments such as birthweight.  
\citet{BhalotraClarke2015} question the exogeneity assumption in another way.
They demonstrate that healthier mothers are more likely to take twin births
to term, and at the same time that healthier mothers are more likely to have
additional resources to invest in child quality later in life.  This critique is
laid out and estimated in chapter \ref{chap:twins} of this thesis.  At the very
least however, both of these critiques will lead to predictable biases in 
estimates of $\beta$, resulting in bounds on the effect of fertility on child
outcomes.

A frequently used alternative to twin births consists of instrumenting with the 
gender mix of children born in the family. Generally, it is argued that parents
prefer to have offspring of both genders \citep{ConleyGlauber2006,Angristetal2010,
Beckeretal2010,MillimetWang2011,FitzsimonsMalde2014}, and so those having various 
children of the same sex are more likely to continue childbearing. Alternatively, 
in some circumstances it is argued that parents have a son preference, and so are 
more likely to continue after having early-birth girls \citep{Lee2008,
KumarKugler2011}. In both cases these are empirically shown to be important 
drivers of fertility.  Again, like estimates driven by twin births, empirical 
results are mixed, although recent evidence seems to point to statistically 
insignificant (though nearly universally negative) estimates of the trade-off, 
as outlined in panel B of table \ref{Ftab:childQQ}.

Causality in this case requires that child sex mix has no direct effect on 
quality.  This implies (among other things), that there are no gender-specific 
economies of scale which facilitate child quality investments more when children
are of the same sex \citep{ButcherCase1994}.  While one could argue (and indeed
hope) that goods which could be employed in the household for boy's education 
could also be employed for girl's education, generally there are other concerns.
\citet{DahlMoretti2008} show that gender composition affects the likelihood that
parents live together. \citet{ButcherCase1994} provide extensive discussion of 
the potential that different child gender mixes may affect child costs, and 
demonstrate that in the USA girls with sisters are significantly less educated 
than girls with brothers, postulating that this may be due to a reference group 
effect where parents have lower aspirations for their children when all children 
are girls. Concerns such as these cast doubt on the validity of the exclusion 
restriction described earlier in this section.

A range of other instruments have been proposed, including infertility
\citep{Bougmaetal2015}, miscarriage \citep{Hotzetal1997,Marlani2008,Miller2009}
and distance to family planning \citep{DangRogers2013}. The outcomes and 
empirical results related to these studies are displayed in table 
\ref{Ftab:childQQ}.  While these instruments---all generally related to the 
ability to conceive or control conception---clearly drive fertility, in each 
case the exclusion restriction is questionable.  This is explicitly treated in
\citet{Hotzetal1997}, who motivate techniques to recover bounds on the estimate
of the effect of fertility.  At the very least, in each case if unhealthy
women are more likley than healthy women to be infertile or suffer miscarriage,
this suggests a positive bias in IV estimates of $\beta$.

\begin{table}[htpb!]
\caption{Empirical Results: Fertility and Child Outcomes (IV)}
\label{Ftab:childQQ}
\begin{tabular}{lllc} \toprule
\textsc{Author} & \textsc{Country} & \textsc{Outcome} & \textsc{Estimate} \\
                &                  &                  & \textsc{(Std.\ Err.)} \\ \midrule
\multicolumn{4}{l}{\textbf{Panel A: Twins}} \\
\citet{Blacketal2005}            &Norway   &Years of Educ        &-0.16(0.44) \\
\citet{Caceres2006}              &USA      &Private School       &-0.000(0.005)\\
                                 &         &Behind cohort        & 0.005(0.004)\\
\citet{Lietal2008}               &China    &Educ (categorical)   &-0.027(0.014)\\
                                 &         &Educ (enrolment)    &-0.025(0.013)\\
\citet{Dayiogluetal2009}         &Turkey   &Attendance           & 0.203(0.245)\\
\citet{Sanhueza2009}             &Chile    &Years of Educ        &-0.280(0.092)\\
\citet{Blacketal2010}            &Norway   &IQ (standardised 1-9)&-0.170(0.052)\\
\citet{Angristetal2010}          &Israel   &Years of Educ        & 0.167(0.117)\\
                                 &         &Some college         & 0.059(0.036)\\
                                 &         &College grad         & 0.052(0.032)\\
\citet{FitzsimonsMalde2010}      &Mexico   &Years of Educ (F)    & 0.096(0.063)\\
                                 &         &Enrolment (F)        &-0.019(0.014)\\
\citet{SouzaPonczek2012}         &Brazil   &Years of Educ (F)    &-0.634(0.194)\\
                                 &         &Years of Educ (M)    &-0.060(0.164)\\ \midrule
\multicolumn{4}{l}{\textbf{Panel B: Gender Mix}} \\
\citet{ConleyGlauber2006}        &USA      &Private school       &-0.061(0.021) \\
                                 &         &Grade repetition     &0.007(0.004)  \\
\citet{Lee2008}                  &Taiwan   &Total ln(educ spend) &0.328(0.088)  \\
\citet{Angristetal2010}          &Israel   &Years of Educ        &-0.067(0.120)\\
                                 &         &Some college         &-0.025(0.025)\\
                                 &         &College grad         &-0.032(0.022)\\
\citet{Beckeretal2010}           &Prussia  &Enrolment            &-0.430(0.189)\\
\citet{KumarKugler2011}          &India    &Years of Educ        &-0.363(0.061)\\
\citet{FitzsimonsMalde2014}      &Mexico   &Years of Educ (F)    &-0.015(0.125)\\ 
\citet{MillimetWang2011}         &Indonesia&BMI for Age          & 0.049(0.013)\\
\midrule
\multicolumn{4}{l}{\textbf{Panel C: Fertility Shock}} \\
\citet{Bougmaetal2015}           &Burkina Faso&Years of Educ        &-0.99(0.40)\\ 
\citet{Marlani2008}              &Indonesia   &Years of Educ (early)&-0.167(0.117)\\ 
                                 &            &Years of Educ (late) &-0.054(0.055)\\
\citet{Hotzetal1997}             &USA         &Complete highschool&-0.147(0.406)\\
\citet{DangRogers2013}           &Vietnam     &Years of Educ        &-0.589(0.392)\\
                                 &            &Private tutoring     &-0.318(0.147)\\
\bottomrule
\multicolumn{4}{p{12.9cm}}{\begin{footnotesize}\textsc{Notes:} In the case that 
various samples are reported in the papers, the pooled estimate for female and
male children of all women from the most recent time period is reported. In the 
case of twins estimates, the 3+ sample (twins at third birth as instrument) is
reported.  Where the original studies report $p$-values associated with estimates 
rather than standard errors, these are converted into standard errors for 
inclusion in this table.
\end{footnotesize}}
\end{tabular}     
\end{table}

Beyond general threats to inference discussed in this section, instrumental 
variable estimates lead to the question of `inference for whom?'.  Estimates 
based on IV lead to a local average treatment effect (LATE), not an average 
treatment effect for the population in general \citep{ImbensAngrist1994,
Ebenstein2009}.  This LATE implies that any estimates of $\beta$ holds for that 
group of the population who would be induced to change their behaviour (ie their 
fertility) by the instrument in question. Thus, all instrumental estimates (even 
assuming causality) should be cast in terms of the sub-population (compliers) of 
interest.  This is a point explicitly discussed in \citet{Angristetal2010}, who 
suggest that the twin instrument is relevant for the whole population, while 
sex-composition instruments are relevant for only certain groups. 
\citet{RosenzweigWolpin1980}'s original article, although based on a reduced 
form equation, suggests that twin births are relevant for a more specific group 
than that suggested by \citet{Angristetal2010}: namely, those families who have 
a twin birth where the twin birth causes them to exceed their desired fertility. 
I return to discuss LATE and external validity in the following section of this 
paper.

%Make table of potential biases...  Ie we know that coefficient between 
%instrument and fertility is +/-.  Then if correlation between instrument
%and error is +/-, this implies:


%-------------------------------------------------------------------------------
\subsection{Natural Experiments}
\label{Fsscn:kidNExp}
An alternative manner to deal with correlation between $F$ and $U$ consists of
taking advantage of externally defined (to $U$) reforms.  If reforms are 
applicable to a subgroup of a particular population and are designed to affect
fertility, this suggests a natural `treatment' and `control' group which can
be compared. Those who receive coverage from the fertility reform are considered 
treated, and those who don't are considered as controls.  If reforms are truly 
put in place for reasons entirely divorced from $U$, causal conclusions can be 
drawn regarding the effect of the reform.  Typically the effect of reforms is 
estimated using difference-in-differences (DD).  This compares pre-reform 
differences between treated and control units with post-reform differences, 
inferring that any change in the level of differences is driven by the reform, 
or stated in another way, that \emph{no} differential and simultaneously 
occurring phenomena seperate treatments from controls.  This is the well known 
`parallel trends assumption' and is central to this line of inference.

These studies can be broadly split into two groups: those which examine the
effect of public policies or other natural experiments on fertility itself,
and those which leverage the externally-defined effect on fertility to quantify
the effect of fertility on some other outcome of interest. In the latter case, 
the differentiation between difference-in-differences and IV estimates is 
artificial, as the (DD estimated) effect of the policy on fertility is 
simply plugged in as the first stage in a 2SLS IV framework.\footnote{
\citet{Duflo2001} is a well known example of this design.  We discuss examples 
of this framework applied to fertility later in this section and in 
\ref{Fsscn:motherNExp}.} The first set of studies are of fundamental importance 
in analysing the \emph{determinants} of fertility and the effect of new 
contraceptive methods on life-cycle childbearing, but do not directly quantify 
the causal effects of fertility itself.  Nevertheless, given their relevance both 
as a first stage in causal estimates and as a reduced form estimate itself, I 
outline a number of these studies below, before moving on to a more comprehensive 
discussion of their link to causal estimates.

\begin{table}[htpb!]
\caption{The Estimated Effect of Reforms on Fertility (Selected Studies)}
\begin{tabular}{lccl} \toprule
Author & Abortion Effect & Pill Effect & Note \\ \midrule
\citet{AngristEvans1996}      & -0.012(0.004) &               & $a,(x=19)$   \\
\citet{Levineetal1996}        & -0.019(0.007) &               & $b$          \\
\citet{Gruberetal1999}        & -0.059(0.005) &               & $c$          \\
\citet{Bailey2006}            & -0.093(0.043) & -0.074(0.057) & $a,(x=22)$ \\
\citet{Guldi2008}             & -0.100(0.054) & -0.085(0.041) &              \\
\citet{Bailey2009}            & -0.012(0.007) &  0.028(0.048) & $a,(x=22)$ \\
\citet{OltmansHungerman2012}  & -0.043(0.015) & -0.088(0.023) &              \\
\bottomrule
\multicolumn{4}{p{12.2cm}}{\begin{footnotesize}\textsc{Note:} All figures report 
the results of short term access of a fertility reform on birth rates of young 
women unless otherwise specified in notes. \newline 
$^a$ Binary model with outcome 1=first birth by age $x$. \citet{Bailey2009} is
an erratum for \citeyear{Bailey2006}.\newline
$^b$ Estimate expressed as births per woman.  Mean rate is 0.110 \newline 
$^c$ Estimate for states adopting 1974-1975. Estimate for 1971-1973 is 
-0.021(0.005). %\newline
%$^d$ The second version (2009) is an erratum, fixing coding errors from 2006.
\end{footnotesize}} \\
\end{tabular}
\end{table}

Historically, many fertility reforms have been atypical when compared to other 
large publicly defined policies.  The nature of fertility control technologies 
has meant that large changes in contraceptive availability have often occurred
which were quite different (in both timing and design) to the stated aims of
public planners.  For example, the advent of the contraceptive pill in the 
1950s, as well as the repeal of contraceptive laws in \emph{Griswold v.\ 
Connecticut} in 1965 \citep{Bailey2013} meant that the largest contraceptive 
reform in the USA in the 20\textsuperscript{th} century occurred without an 
explicit public reproductive policy.\footnote{More recently, the US Supreme 
Court finding in \emph{Burwell v. Hobby Lobby} affected birth control access 
in the absence of any reproductive policy.  Similar examples exist in many other 
contexts.  An additional example is examined in chapter \ref{chap:pill} of this
thesis.} The piecemeal and unexpected nature of these reforms makes it difficult 
in many cases to quantify the effect of reforms on their stated aims due to the
lack of centralised planning. Nevertheless, in some circumstances, comprehensive
reviews can be conducted, focusing on the total effect of reproductive policy
reforms on its actual aim.  For example, \citet{MolyneuxGertler2000} analyse 
the effect of a national policy (Indonesia) while many analyses exist of the 
local experimental policy in Matlab Bangladesh \citep{JoshiSchultz2013}. However,
even in the absence of large-scale analyses of the effects of reproductive policy 
reforms on their stated aims, historical contraceptive reforms are still an 
excellent candidate to isolate the effect of changes in fertility on other
outcomes of interest given that these events have large effects on fertility,
and are potentially divorced from other simulatenous reforms or differential 
trends.

Of the large number of studies which use reforms of fertility-control policies%
\footnote{For abortion: \citet{Ananatetal2007,Ananatetal2009,AngristEvans1996,
CharlesStephens2006,Cookeetal1999,Currieetal1996,Gruberetal1999,Guldi2008,
KaneStaiger1996,Levineetal1996,Levineetal1996b,Levineetal1999,PopEleches2005,
PopEleches2006}, for the oral contraceptive pill: \citet{OltmansHungerman2012,
Bailey2006,Bailey2011,Bailey2012,Bailey2013,Christensen2012,Goldin2006,
GoldinKatz2002a,GoldinKatz2002b,KearnerLevine2009} and for the emergency 
contraceptive pill: \citet{Durrance2013,Grossetal2014} and chapter 
\ref{chap:pill} of this thesis.} to examine the effect on fertility in a 
DD-style framework, only a relatively small number then employ this 
as the first stage to estimate the causal effect of fertility---the focus of
this review chapter.  Among those that \emph{do} directly estimate the effect
of fertilty on child outcomes are \citet{Gruberetal1999,Ananatetal2009} and
\citet{OltmansHungerman2012}. \citet{Gruberetal1999} examines the effect of 
fertility (via 2SLS) on the likelihood that a child lives with single parents, 
lives in poverty, receives welfare, and on rates of infant mortality and low 
birth rates.  Of these, it is suggested that fertility significantly increases 
the probability of living in poverty and having single parents, as well rates 
of infant mortality. \citet{Ananatetal2009} also examine these outcomes, and 
suggest that in the long-run, the marginal child is more likely to have lived 
with a single parent, receive welfare, and not have graduated college.
Finally, \citet{OltmansHungerman2012} return to these same outcomes and report
a Wald ratio as in (\ref{Feqn:Wald}).  These Wald estimates allow them to look
at the characteristics of marginal child not born due to both the diffusion of 
the pill, and the legalisation of abortion.  Their results suggest that the two 
fertility control policies had remarkably different effects on marginal child 
characteristics.  In agreement with the above studies they suggest that the
marginal child not born due to abortion legalisation would have been 49.2\%
(se=25.5) more likely to live in a welfare-receiving household.  However, the 
marginal child not born due to pill diffusion looks remarkably different: 8.0\% 
(se=4.4) \emph{less} likely to belong to a welfare receiving household.  These 
comparisons make manifestly clear the distinction between compliers for 
different instruments discussed at the end of section \ref{Fsscn:kidIV}.
Given that the group of `compliers' in the two policies had very different
characteristics, estimated effects of fertility on outcomes are very different
despite being plausibly causal in both cases.

Despite not directly estimating the causal effect of fertility on child 
outcomes, a number of other contraceptive-based natural experiment papers
estimate the effect of the natural experiment directly on child outcomes.
This reduced form technique provides an estimate of the numerator of the ratio
in (\ref{Feqn:Wald}), and so can be thought of as an unscaled estimate of
the effect of fertility.  Papers of this type include \citet{PopEleches2006} 
who finds that the illegalisation of abortion in Romania worsens child 
education and labour market outcomes (conditional on parental characteristics)
and \citet{Bailey2013} who reports that US contraceptive pill laws had long-%
standing impacts on children's eventual college completion, labour force 
participation, and family incomes.

The validity of using policies of this type to isolate the effects of 
childbearing on child outcomes hinges upon the fact that the timing (or
allowance) of fertility control reforms should not depend upon pre-existing
differences between areas affected and those not affected by the reform. Any
phenomena which will imply that `treated' and `untreated' areas would follow
different paths \emph{in the absence} of the reform will lead to inconsistent
estimates of the effect of fertility on child outcomes.  Generally, papers
which propose estimation by leveraging reforms of this type run a series of
tests, including event-study analysis, placebo regressions, or a regression
of receipt of treatment on pre-existing characteristics.\footnote{For example,
\citet{Bailey2006} demonstrates that early access to the pill was unrelated
to education, fertility norms, poverty rates, availability of household 
technologies such as washers and dryers, as well as labour market participation
at a state level.  The probability of early access is, however, related to the
percent of Catholic residents in a state.}  However, directly testing the
validity of such estimation methodologies is, of course, impossible, given
that the counterfactual outcome--- the world where the pill was not available%
---is never observed.  This has lead to back-and-forth discussion, questioning 
the validity of the use of policy-defined reforms to drive estimation (for 
example, see \citet{Joyce2013}, who questions the exclusion restriction vs.\ 
\citet{Baileyetal2013} who defend current state-of-the-art results). 

While concerns that reforms may be systematically correlated with other 
unobservable factors are of course justifiable, the best sets of studies aim to 
use judiciously chosen control groups (including using women of different ages 
subject to the same geographic factors and institutions), to minismise concerns 
such as these. An alternative concern in identification strategies of this type 
surrounds the possiblity that \emph{local} reforms have more widely spread
effects.  For example, the availability of abortion in one region does not
necessarily imply that nearby non-treated individuals cannot travel to
treated areas, defying their quasi-experimental status to receive treatment 
\citep{Levineetal1999}. Fortunately, violations of this type will, at worst, 
bias downwards estimated results. While this is reassuring if our object of
interest is a bounds estimate, generally with policy reforms this will not be
the case.  Given the large cost that large contraceptive policies entail, it
is important to be able to identify the precise effect of competing options. 
As a result, concerns such as these are often examined empirically, as is the 
case in \citet{Christensen2012}. The importance of estimating a causal estimate
in this context is returned to more extensively, and tested explicitly, in 
chapter \ref{chap:spill} of this thesis.

Finally, a number of other natural experiments have been used in the 
literature to examine the effect of fertility on child outcomes.\footnote{
Similarly, \citet{BleakleyLange2009} use a natural experiment: the eradication
of hookworm in USA, to test the QQ hypothesis.  However, the elimination of
hookworm is used as a shifter for child quality, \emph{not} child quantity.
This allows them to quantify the effect of quality increases on subsequent
fertility decisions of households, and they find that increases in quality
do lead to fertility declines in line with the QQ model discussed earlier.}
Perhaps most notably among these, \citet{Qian2009} uses the relaxation of 
China's one child policy to estimate the causal effect of movements from one 
child to two child households.  This study is unique for two reasons: the low 
parity shift of the experiment (an expansion from one to two children), and the 
fact that it finds that higher fertility in this case \emph{increases} child 
schooling outcomes, especially among households who have two children of the 
same gender. These results suggests that estimates of fertility at the intensive 
margin may not be linear, and indeed may not even be monotonic by parity, 
changing from positive to negative at higher orders.

\subsection{Structure and Dynamics}
A number of dynamic or dynamic structural papers motivate estimation of the
effects of fertility (or fertility timing) on birth outcomes based on a finite 
horizon, rather than static, estimation framework.  While these papers allow for 
a more extensive examination of \emph{timing} and life-cycle decisions, 
estimation generally relies on an exclusion restriction similar to those 
discussed in section \ref{Fsscn:kidIV}.  \citet{RosenzweigWolpin1995} motivate 
the estimation of a dynamic model to examine the effect of early fertility (teen 
motherhood) on child birth outcomes (gestation and birthweight).  By formulating 
an over-identified series of equations where identification (in a Fixed Effect-%
IV framework) comes from family background variables, and idiosyncratic elements 
shared by siblings. \citet{RosenzweigSchultz1985} take advantage of variations 
in fecundity, or births per attempt, which they suggest are unobserved by 
parents prior to contraception attempts, but observable after the fact in data, 
as births per period.  Identification in these studies usually depends on 
correct functional form and distributional assumptions for the stochastic error 
term(s), though it is important to point out that precisely the same conditions 
are the case for parametric OLS (DD) and IV estimates discussed in previous 
subsections.

%-------------------------------------------------------------------------------
\section{The Effects of Child Birth on Mothers}
\label{Fscn:mothers}
Beyond the analysis of a child's effect on his or her siblings' outcomes, a
birth, at the extensive or the intensive margin, has myriad impacts on parents
or other carers.  The analysis of these effects has received considerable and 
ongoing attention in the economics literature.  Much of the focus of this work
falls on the effect of marginal births on mothers' labour market outcomes and
trajectories.

\citet{FleisherRhodes1979} provides a summary of the early literature, with 
considerable coverage also provided in the \emph{JPE} Fertilty issue described
in section \ref{Fscn:kids} \citep{Willis1973,Gronau1973}.  As is the case with
child investment and fertility decisions, choices regarding fertility, labour
market participation, and (adult) human capital attainment are linked, and 
dynamic in nature.  Total fertility, and, if child-bearing, birth timing have
important impacts on labour market participation, non-labour market work, 
accrued experience, and wages, while participation, experience and wages also 
influence timing and fertility decisions. Inferring causality in systems of this 
type is once again challenging, relying on the use of plausible instruments, 
natural experiments, structural estimation, or a combination of methods.

\subsection{Natural Experiments}
\label{Fsscn:motherNExp}
Frequently, natural experiments experiments of the type discussed in section 
\ref{Fsscn:kidNExp} are leveraged to quantify the effect of fertility on 
parent outcomes.  Estimation is based on the fact that---at the level of the
family---living in treatment or non-treatment areas is a randomly assigned
variable which can be used to isolate effects on fertility in the absence of 
changes in other outcomes.  This requires that these experiments be clearly 
demarcated, unexpected, and not propogate from treated to untreated areas.

One of the most common natural experiments employed in these types of analyses 
is the arrival of new birth control technologies to a particular geographic 
area.  There are a very large range of microeconomic studies which discuss 
the effect of these types of programs on a mother's total fertility.  These can
be broadly split into those which examine the short-run effects of 
contraceptives on fertility,\footnote{This does not imply using data over a 
short time frame, but rather examining the effect of a birth control method up 
to an age \emph{less} than the end of the fertile life (eg \citet{Bailey2006}'s
focus on childbearing before the age of 22).} and long-run analyses, which 
account for both short-run delays and long-run rearrangements in timing 
afforded by new technologies.  Short run analyses include those examining the 
contraceptive pill \citep{Bailey2006,Bailey2009,Christensen2012}, abortion
\citep{Guldi2008,Levineetal1999}, the morning after pill 
(\citeauthor{Grossetal2014},\citeyear{Grossetal2014}; \citeauthor{Durrance2013}, 
\citeyear{Durrance2013}; chapter \ref{chap:pill}) and medicare access 
\citep{KearnerLevine2009}, while those examining the long-run effects of
contraceptive reform on completed fertility include (among others)  
\citet{Bailey2011,Bailey2013,Bailey2012} for the contraceptive pill and 
\citet{AngristEvans1996,OltmansHungerman2012} for abortion.

Once again however, beyond the direct relevance of this swarth of studies for
policy focused on fertility control, in order to apply these results to
\emph{causal} analysis of parental outcomes, the specifications discussed above 
can only act as a first-stage effect.  For the full system of equations, we
are interested in a two-step process: first quantifying the effect of reforms
on fertility, and then, from this, the flow-on effect that exogenous shifts in
fertility have on mother (or carer) outcomes.

Only a subset of papers which focus on fertility reforms then go on to examine
the second stage of interest here.  As discussed in sections \ref{Fsscn:kidIV} 
and \ref{Fsscn:kidNExp}, consistent estimation relies on an exclusion restriction
assumption, whereby the only effect of the program on outcomes is driven by
its effect on fertility. \citet{OltmansHungerman2012} use both the pill and
abortion to examine different groups of compliers, and report Wald estimates
of the effect of fertility on single parenthood: for pill-compliers, marginal
fertility reductions occur in contexts with less single parenthood, while the
reverse is true for abortion compliers.\footnote{They also show that the 
long-term effect of the pill on mothers results in a higher likelihood of college 
completion, having a college educated spouse, and a lower likelihood of 
divorce.} \citet{AngristEvans1996} report similar estimates for 1970 abortion 
reforms in the USA.  They report that the effects of a particular type of 
child-bearing (teen and unmarried) reduces the education and employment 
probability, particulary of black women.  Other significant outcomes discussed 
in this framework include \citet{Baileyetal2012,Bailey2006,Bailey2013}, who 
show that it has effects on wages over the life cycle or female labour force 
participation rates, and \citet{Christensen2012}, who finds (reduced form) 
effects on cohabitation.

\subsection{Instrumental Variables}
The use of instrumental variables to examine the effect of fertility on 
\emph{mothers'} outcomes (rather than children's outcomes as described
in section \ref{Fsscn:kidIV}) follows a similar logic to that outlined in
equation (\ref{Feqn:Wald}).  An external variable which has strong effects
on fertility but no direct effects on the outcome of interest except via its
effect on fertility can be used to drive causal estimates.  Instrumental 
variable estimates are a popular methodology employed to determine the 
effect of fertility on mothers.

Outcome variables of interest are typically related to parental labour 
force outcomes,\footnote{Largely, these papers focus on maternal labour
market participation rates. \citet{KimAassve2006} studies both mothers' and
fathers' responses to fertility, using fecundity (births per attempt) as an
instrument.  They find that on average mothers reduce hours of work in the short 
run, while paternal hours of work increase (in rural areas).} including female 
labour force participation \citep{AgueroMarks2008,AgueroMarks2011,ChunOh2002,
Caceres2008,AngristEvans1998}, or earnings \citep{Caceres2006,Hotzetal1997,
Jacobsenetal1999}.  A range of instruments has been proposed including twins,
as in section \ref{Fsscn:kidIV} \citep{RosenzweigWolpin1980b,Jacobsenetal1999,
BronarsGrogger1994}, gender mix \citep{AgueroMarks2008,AgueroMarks2011,
ChunOh2002}, and fertility shocks \citep{Miller2011,Cristia2008,
RosenzweigSchultz1987}.\footnote{\citet{Ribar1994} proposes three alternative
exclusion restrictions (age at first period, availability of Ob/Gyn, and local
abortion rates) for use in selection models.  While the identification 
methodology is different to IV, the requirements for inferring causality are 
identical.}  The signs and magnitude of existing estimates of the effect of 
fertility on labour market outcomes largely point towards significant negative 
impacts, though not universally so.  A summary of point estimates and 
confidence intervals of estimates is presented in table \ref{ferttab:motherIV}.

\pgfplotstablegetrowsof{\dataB}
\let\numberofrows=\pgfplotsretval

\begin{table}
\caption{Fertility and Mother's Labour Market Outcomes}
% Print the table
\pgfplotstabletypeset[
columns={name,error,beta},
  every head row/.style={before row=\toprule, after row=\midrule},
  every last row/.style={after row=[1ex]},
  columns/name/.style={string type,column name={}},
  columns/error/.style={
    column name={$\hat\beta \pm$ se($\hat\beta$)},
    assign cell content/.code={% use \multirow for Z column:
    \ifnum\pgfplotstablerow=1
    \pgfkeyssetvalue{/pgfplots/table/@cell content}
    {\multirow{\numberofrows}{6.0cm}{\errplot{\dataB}}}%
    \else
    \pgfkeyssetvalue{/pgfplots/table/@cell content}{}%
    \fi
    }
  },
  % Format numbers and titles
  columns/name/.style={column name=Authors, string type, column type={l}},
  columns/beta/.style={column name=$\beta$, string type, column type={S[table-format=-2.2]}},
  columns/ci/.style={column name=$95\%$ CI, string type, column type={S[table-format=-1.2]}},
 ]{\dataB}

\pgfplotstablegetrowsof{\dataA}
\let\numberofrows=\pgfplotsretval

\pgfplotstabletypeset[
columns={name,error,beta},
  every head row/.style={output empty row, after row=\\},
  every last row/.style={after row=[3ex]\bottomrule},
  % Set header name
  columns/name/.style={string type,column name={}},
    % Use the ``error'' column to call the \errplot command in a multirow cell in the first row, keep empty for all other rows
  columns/error/.style={
    column name={$\hat\beta \pm$ se($\hat\beta$)},
    assign cell content/.code={% use \multirow for Z column:
    \ifnum\pgfplotstablerow=1
    \pgfkeyssetvalue{/pgfplots/table/@cell content}
    {\multirow{\numberofrows}{6.0cm}{\errplot{\dataA}}}%
    \else
    \pgfkeyssetvalue{/pgfplots/table/@cell content}{}%
    \fi
    }
  },
  % Format numbers and titles
  columns/name/.style={column name=Authors, string type, column type={l}},
  columns/beta/.style={column name=$\hat\beta$, string type, column type={S[table-format=-2.2]}},
  columns/ci/.style={column name=$95\%$ CI, string type, column type={S[table-format=-1.2]}},
  ]{\dataA}
\\
\begin{small}
\begin{quote}
\textsc{Notes to table:} Points represent coefficients, while error bars represent 95\% confidence intervals.  Estimates are ordered by date of publication.  In the case that various samples are reported in the papers, the pooled estimate for all women from the most recent time period is reported.  In the case of twins estimates, the 3+ sample (twins at third birth as instrument) is reported.
\end{quote}
\end{small}
\end{table}


These point estimates appear to be largely negative, with only two (from the
same context and cohorts) suggesting non-negative results of fertility on
female labour force participation or hours worked.  What's more, these are
largely statistically significant negative estimates, despite the well known 
caveat that IV estimates typically suffer from very wide confidence intervals 
\citep{Angristetal2010}.  However, as discussed earlier, all of these
estimates are LATEs which hold for particular populations and compliers, and
so do not provide external validity for inference in other populations. 
However, a literature pointing in the direction of a negative effect is
suggestive that this result could be observed in other contexts.  This is
something examined extensively by \citet{Deheijaetal2015} who, perhaps
unsurprisingly, find that quasi-experimental evidence generalises more
readily to countries which share closer geographical, education, time, 
and labor force participation characteristics.

While the majority of these instruments can only be used to estimate the effect
of fertility at the intensive margin, interestingly, those based on fertility
shocks can also be applied to quantify the effects of extensive margin births. 
\citet{Cristia2008} for example proposes using the outcome of fertility 
treatments (pregnant or not) as an instrument, suggesting that delays in 
child-bearing lead to an increase in wages and hours worked.

Finally, miscarriage has been proposed as an alternative IV that can be employed 
to estimate the effect of child birth on maternal outcomes \citep{Hotzetal2005,
Fletcher2012}. This line of argument relies on fetal deaths in utero being 
randomly assigned to mothers, in order to compare treated (live births) to 
control (no live births) women. This also relies on miscarriage not having any 
other effect on (propsective) mothers' outcomes of interest, beyond its direct 
effect on fertility. \citet{Hotzetal2005} suggest that following this line of 
argument, early (teenage) child-bearing is associated with small effects on 
educational attainment, and life cycle changes in labour market rates.

The use of this instrument is of course complicated if characteristics which
predict miscarriage are also correlated with mother unobservables. Given that
miscarriage is considerably more likely for unhealthy mothers, this seems likely,
and a range of studies address these concerns.  Foremost is \citet{Hotzetal1997} 
who discuss how to bound the effect of fertility where the instrumental variable
is composed of a mixture of both women who randomly miscarry, and those who 
non-randomly miscarry.  They show that tight bounds on the effect of fertility can 
be estimated, if the proportion of non-random and random miscarriages can be
estimated.  Applying these bounds estimates they suggest that teenage child-bearing
significantly increases the number of hours worked during early adulthood, and
(weakly) decreases the likelihood of completing a GED certificate in the USA.
\citet{FletcherWolfe2009} provide additional discussion of the challenges in
estimating causal effects using miscarriage.  They suggest that unobserved
\emph{community}-level characteristics are likely correlated with miscarriage, 
and once including community fixed-effects find that teen child-bearing reduces
education and wages, and increase the likelihood of welfare receipt.


\subsection{Other Methods}
A range of other methods have been employed in the economic and non-economic
literature to examine the effects of fertility on mother's outcomes. These involve
RCTs \citep{DiCensoetal2002} between-effects using siblings \citep{Holmlund2005,
GeronimusKorenman1992,Ribar1999}, and other matching methods 
\citep{ChevalierViitanen2003,LevinePainter2003}.  In the case of the last two
methods (between-effects and matching), the identification of casual effects
relies on the comparison method fully controlling for relevant differences between
those having children, and those not having children.  In matching this collapses
to an assumption regarding `selection on observables' (which is to say that any
characteristic predicting child-bearing is observed by the econometrician), and in
siblings or relative fixed effects, that on average, those who become pregnant
early in life are otherwise identical to those who become pregnant later in life.
\citet{Ribar1999} and \citet{RosenzweigSchultz1985} provide additional discussion, 
and examination of, the validity of these estimation techniques.

Finally, \citet{RosenzweigWolpin1980b}---in their initial proposition of twins as
an exclusion restriction---return to the Beckerian (\citeyear{BeckerLewis1973})
simultaneous equation framework for fertility, child quality, \emph{and} life-%
cycle (mother's) labour supply.  They are the first to use twins to estimate
the structural equation linking fertility and labour supply.  They estimate
that for younger women, additional births reduce labour supply, but this fades
as women age.  Once again---as they indeed highlight---consistent estimation
relies on twins being entirely orthogonal to labour supply.  This assumption 
is questioned in previous sections of this review chapter. Using the presumed 
exogeneity of twins as an identifying assumption, \citet{RosenzweigWolpin1980b} 
provide a very interesting series of tests casting considerable doubt on the 
assumption that fertility is exogenous to labour supply decisions, as maintained 
in the prevailing literature at the time of their work.


%-------------------------------------------------------------------------------
\section{Conclusion}
This review chapter serves to provide an overview of the causal estimation of 
the effect of fertility on child and parental outcomes.  It surveys the wide 
range of methodologies employed in the existing microeconometric literature, and 
discusses how various techniques aim to skirt issues of endogenous fertility 
choices.  In each case, I outline the identifying assumptions implicitly or
explicitly invoked, as well as the threats to which these are subject.

The evidence discussed in this chapter is mixed.  While there seems to be 
quite clear evidence in favour of moderate-to-large effects of marginal child 
births and early births on parental labour market outcomes, the existing 
microeconometric child-level estimates are less compelling.  Despite a large
body of theoretical microeconomic work which posits that such a QQ trade-off 
may exist, causal estimates are certainly not conclusive, and seem to suggest 
that the trade-off is small or non-existent.  While there are a number of 
papers which \emph{do} find significant effects on a number of outcomes, these 
are context- and complier-specific.

In some cases this lack of evidence may be due to threats to exclusion 
restrictions or other identifying assumptions. It is in line with this that 
the present thesis now moves forward.  The remaining chapters are essays on the 
causal estimate---or more specifically, the challenges and involved in causally 
estimating---the effect of fertility and contraceptive programs on human 
outcomes, and some solutions which may be employed to recover bounds and causal
estimates of these effects.

%\newpage
%\begin{table}
%\caption{Requirements for Causality}
%\begin{tabular}{lcc}\toprule
%Instrument & Positive Bias & Negative Bias  \midrule
%Twins & $\Cov(Twin,U)>0$ & & $\Cov(Twin,U)<0$ & 
%\end{tabular}
%\end{table}



%FROM TWIN IV
%Discuss this...
%This is particularly accute for studies which use later born \citet{Glicketal2007}
%children as their subjects, and those who focus on early life variables of 
%children born before twins.

