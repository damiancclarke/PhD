%-------------------------------------------------------------------------------
\begin{chapabstract}
Child-bearing decisions are not made in isolation.  They are taken in concert 
with decisions regarding work, marriage, health investments and stocks, as well 
as many other observable and non-observable considerations.  Drawing causal 
inferences regarding the effect of child birth is complicated by these 
endogenous factors.  This review chapter outlines the identification of causal
estimates of fertilty, and the assumptions underlying the range of estimators
and methodologies proposed in the economic, as well as the non-economic, 
literature.
\end{chapabstract}

\newpage
\section{Introduction}
Make point about extensive versus intensive margin.

Outcomes.

Identification.  Shifters: twins, fertility policies
Methodologies. IV, DD, (RD?), FE.

Timing as well as number.

General points: \citet{Enke1966} (and others by him: 1960, 1971), 
\citet{KearneyLevine2012, Schultz2008, GMiller2009, RosenzweigWolpin1986}
(maybe add Miller cite to twins).

%Black et al EJ 2008 type citations of other drivers of fertility?
%Discussion of Causality: Pearl, Heckman, Haavelmo etc?

\newpage
\section{Causality and Fertility}
\label{Fscn:causality}
\citet{Moffitt2005} is a useful starting point.

We define an outcome $Y$



\newpage
%-------------------------------------------------------------------------------
\section{The Effects of Family Size on Children}
In 1973, the \emph{Journal of Political Economy} released a special issue 
dedicated to new economic approaches to fertility.  Interest in determining
the causal effect of total fertility on the outcomes of children in the 
household blossomed from the articles it contained.  A common theme in a number 
of articles in this issue \citep{BeckerLewis1973, DeTray1973, Willis1973} 
concerns a family's decisions regarding fertility (the quantity of children) 
and investments in child human capital (the `quality' of children). Abstracting 
from intra-household variations in child quality, each of the aforementioned
articles demonstrates the theoretical existence of a quantity--quality
(Q--Q trade-off).\footnote{As \citet{Willis1973} succinctly describes:
\begin{quote}
`Thus, parents not only balance the satisfactions they receive from their
children against those received from all other sources not related to 
children \ldots, but they must also decide whether to augment their 
satisfaction from children at the ``extensive'' margin by having another
child or at the ``intensive'' margin by adding to the quality of a given
number of children.'
\end{quote}
}

The Q--Q trade-off described in the above series of articles as well as in
\citet{BeckerTomes1976,BeckerTomes1986} owes to the joint entry of quality and 
quantity in the household budget constraint.  As the number of children 
enters in the shadow price of quality, and the quality of desired children 
enters the shadow price for quantity, decisions regarding fertility and quality 
must be taken jointly.  Holding all else constant, increases in fertility 
increase the shadow price of quality, and increases in quantity increases the
shadow price of the marginal birth. This considerably complicates causal 
inference. What's more, as recognised in early articles by \citet{%
BenPorathWelch1972,BenPorath1976}, quality decisions may \emph{directly} feed 
back to quantity via child mortality.

\textcolor{red}{Discussion of assumptions of Q--Q...
 examine the effect of child 
characteristics (mortality) on family size., and loosening by 
\citet{AizerCunha2012} (from \citet{Behrmanetal1982}'s theory, gender discussion 
of \citet{ButcherCase1994}) recently. \citet{Lawsonetal2012} is one of many 
biological papers.}

%-------------------------------------------------------------------------------
\subsection{Instrumental Variables}
In systems of equations of this type, one way to drive inference is through the
use of shifters (or instrumental variables) which affect the quantity of one of
the variables without affecting the other.  In order to identify the effect of 
fertility on children's outcomes, this instrumental variable must affect only 
fertility, with no indirect effects on quality.\footnote{\emph{Ie} the exclusion 
restriction must hold, implying that the estimation of the structural equation 
which contains quality on fertilty and the instrument must result in a 
coefficient on the instrumental variable which is precisely equal to zero.}
Returning to the nomenclature introduced in section \ref{Fscn:causality},
\textcolor{red}{\ldots}

The earliest discussion of these types of exclusion restrictions was in 
\citet{RosenzweigWolpin1980}. They point out that if multiple births are 
unanticipated, their occurrence will cause some families to exceed their desired
fertility, shifting the total number of births in the absence of any change in 
parental considerations of quality investments.  This has motivated estimation 
in a number of papers, where twin births are employed as instrumental 
variables.  Twin instruments have been employed in a range of contexts and
to examine various different `quality' outcome variables.  These include
\citet{Blacketal2005,Caceres2006,Lietal2008,Dayiogluetal2009,Sanhueza2009,
Blacketal2010,Angristetal2010,FitzsimonsMalde2010} and 
\citet{SouzaPonczek2012}, and focus on child quality measures including years of 
education, IQ, private school enrollment, BMI and height, college completion and 
age at marriage.  The evidence on the existence of a Q--Q trade-off in these 
studies is mixed, although recent influential results suggest that the evidence 
in favour of a trade-off may be weak.  In table \ref{Ftab:childQQ}

Point on estimation requirements.

Sex instruments: \citet{ConleyGlauber2006,Lee2008,Angristetal2010,KumarKugler2011,
FitzsimonsMalde2014}

\citet{Beckeretal2010,MillimetWang2011,Lee2008}
Note (or footnote) the birth order discussion in \citet{Blacketal2005}.
\citet{Bougmaetal2015},
Make table of potential biases...  Ie we know that coefficient between 
instrument and fertility is +/-.  Then if correlation between instrument
and error is +/-, this implies:

Timing: \citet{Miller2009}

\begin{table}
\caption{Empirical Results: Fertility and Child Outcomes}
\label{Ftab:childQQ}
\begin{tabular}{lllc} \toprule
\textsc{Author} & \textsc{Country} & \textsc{Outcome} & \textsc{Estimate} \\
                &                  &                  & \textsc{(Std Err)} \\ \midrule
\multicolumn{4}{l}{\textbf{Panel A: Twins}} \\
\citet{Blacketal2005}            &Norway   & Yrs of Educ         &-0.16(0.44) \\
\citet{Caceres2006}              &USA      &Private School       & -0.000(0.005)\\
                                 &        &Behind cohort        & 0.005(0.004)\\
\citet{Lietal2008}               &China    &Educ (categorical)   & -0.027(0.014)\\
                                 &         &Educ (enrollment)    & -0.025(0.013)\\
\citet{Dayiogluetal2009}         &Turkey   &Attendance           & 0.203(0.245)\\
\citet{Sanhueza2009}             &Chile    &Yrs of Educ          &-0.280(0.092)\\
\citet{Blacketal2010}            &Norway   &IQ (standardised 1-9)& -0.170(0.052)\\
\citet{Angristetal2010}          &Israel   &Yrs of Educ          & 0.167(0.117)\\
                                 &         &Some college         & 0.059(0.036)\\
                                 &         &College grad         & 0.052(0.032)\\
\citet{FitzsimonsMalde2010}      &Mexico   &Yrs of Educ (F)      & 0.096(0.063)\\
                                 &         &Enrolment (F)        &-0.019(0.014)\\
\citet{SouzaPonczek2012}         &Brazil   &Yrs of Educ (F)      & -0.634(0.194)\\
                                 &         &Yrs of Educ (M)      & -0.060(0.164)\\ \midrule
\multicolumn{4}{l}{\textbf{Panel B: Gender Mix}} \\
\citet{ConleyGlauber2006}        &USA      &Private school       &-0.061(0.021) \\
                                 &         &Grade repetition     &0.007(0.004)  \\
\citet{Lee2008}                  &Taiwan   &Total ln(educ spend) &0.328(0.088)  \\
\citet{Angristetal2010}          &Israel   &Yrs of Educ          & -0.067(0.120)\\
                                 &         &Some college         & -0.025(0.025)\\
                                 &         &College grad         & -0.032(0.022)\\
\citet{KumarKugler2011}          &India    &Yrs of Educ          & -0.363(0.061)\\
\citet{FitzsimonsMalde2014}      &Mexico   &Yrs of Educ (F)      & -0.015(0.125)\\ 
\citet{MillimetWang2011}         &Indonesia&BMI for Age          &  0.049(0.013)\\
\midrule
\multicolumn{4}{l}{\textbf{Panel C: Other}} \\
\citet{Bougmaetal2015}           &Burkina Faso&&\\ \bottomrule
\multicolumn{4}{p{10cm}}{\begin{footnotesize}\textsc{Notes:} \end{footnotesize}}
\end{tabular}     
\end{table}

\subsection{Natural Experiments}
EMAIL MARTHA BAILEY FOR ARTICLE ON CHILDREN'S OPPORTUNITIES (listed on her site)
\citet{OltmansHungerman2012, Gruberetal1999,PopEleches2006,BleakleyLange2009,
RosenzweigZhang2009, Qian2009,Hossain1989}
(see literature section in Gruber et al., (1999))

\subsection{Other}
Structure: \citet{RosenzweigWolpin1995,RosenzweigSchultz1985,
RosenzweigWolpin1980b}
OLS: \citet{Hanushek1992} is descriptive. \citet{Desai1995}


\section{The Effects of Child Birth on Mothers}
\citet{FleisherRhodes1979} seems to be an early important reference. 
\citet{Willis1973} builds female wages into Q-Q.  Also look at \citet{Reuben1973}.
Make a graph of estimated effects horizontally with error bars (see BMJ 
article for ref).

\subsection{Natural Experiments}
Split into short run and long run
\citet{Bailey2011,Baileyetal2012,Bailey2006,Bailey2013,Bailey2012,Christensen2012,
GoldinKatz2002a,Guldi2008,KearnerLevine2009,Levineetal1999,AngristEvans1996,
Jacobsenetal1999,AngristEvans1998}
(\citet{Bailey2009} is an erratum for Bailey 2012).

Miscarriage as natural experiment (\citet{Hotzetal2005,Fletcher2012}).  
\citet{Hotzetal1997} talk about how to bound this, \citet{FletcherWolfe2009} 
provide further discussion.

MAKE A GRAPH OF DATES OF REFORMS USED TO IDENTIFY OUTCOMES.

\subsection{Instrumental Variables}
FLFP: \citet{Cristia2008,AgueroMarks2008,AgueroMarks2011,ChunOh2002}
Earnings:

\citet{Ananatetal2009,Miller2011,
BronarsGrogger1994,KimAassve2006,RosenzweigSchultz1987,
Caceres2006}
or selection with exclusion restriction, which is fundamentally similar: 
\citet{Ribar1994}.

\subsection{Other}
RCT: \citet{DiCensoetal2002}
Between effects: \citet{Holmlund2005,GeronimusKorenman1992} \citet{Ribar1999} 
tests sibling effects models.\\
Matching: \citet{ChevalierViitanen2003,LevinePainter2003}

\newpage
\pgfplotstablegetrowsof{\dataB}
\let\numberofrows=\pgfplotsretval

\begin{table}
\caption{Fertility and Mother's Labour Market Outcomes}
% Print the table
\pgfplotstabletypeset[
columns={name,error,beta},
  every head row/.style={before row=\toprule, after row=\midrule},
  every last row/.style={after row=[1ex]},
  columns/name/.style={string type,column name={}},
  columns/error/.style={
    column name={$\hat\beta \pm$ se($\hat\beta$)},
    assign cell content/.code={% use \multirow for Z column:
    \ifnum\pgfplotstablerow=1
    \pgfkeyssetvalue{/pgfplots/table/@cell content}
    {\multirow{\numberofrows}{6.0cm}{\errplot{\dataB}}}%
    \else
    \pgfkeyssetvalue{/pgfplots/table/@cell content}{}%
    \fi
    }
  },
  % Format numbers and titles
  columns/name/.style={column name=Authors, string type, column type={l}},
  columns/beta/.style={column name=$\beta$, string type, column type={S[table-format=-2.2]}},
  columns/ci/.style={column name=$95\%$ CI, string type, column type={S[table-format=-1.2]}},
 ]{\dataB}

\pgfplotstablegetrowsof{\dataA}
\let\numberofrows=\pgfplotsretval

\pgfplotstabletypeset[
columns={name,error,beta},
  every head row/.style={output empty row, after row=\\},
  every last row/.style={after row=[3ex]\bottomrule},
  % Set header name
  columns/name/.style={string type,column name={}},
    % Use the ``error'' column to call the \errplot command in a multirow cell in the first row, keep empty for all other rows
  columns/error/.style={
    column name={$\hat\beta \pm$ se($\hat\beta$)},
    assign cell content/.code={% use \multirow for Z column:
    \ifnum\pgfplotstablerow=1
    \pgfkeyssetvalue{/pgfplots/table/@cell content}
    {\multirow{\numberofrows}{6.0cm}{\errplot{\dataA}}}%
    \else
    \pgfkeyssetvalue{/pgfplots/table/@cell content}{}%
    \fi
    }
  },
  % Format numbers and titles
  columns/name/.style={column name=Authors, string type, column type={l}},
  columns/beta/.style={column name=$\hat\beta$, string type, column type={S[table-format=-2.2]}},
  columns/ci/.style={column name=$95\%$ CI, string type, column type={S[table-format=-1.2]}},
  ]{\dataA}
\\
\begin{small}
\begin{quote}
\textsc{Notes to table:} Points represent coefficients, while error bars represent 95\% confidence intervals.  Estimates are ordered by date of publication.  In the case that various samples are reported in the papers, the pooled estimate for all women from the most recent time period is reported.  In the case of twins estimates, the 3+ sample (twins at third birth as instrument) is reported.
\end{quote}
\end{small}
\end{table}


