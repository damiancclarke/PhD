\begin{center}
\textbf{A Note on Authorship}
\end{center}

Chapters \ref{chap:pill} and \ref{chap:spill} were written alone.  Chapter 
\ref{chap:twins} was coauthored with Professor Sonia Bhalotra, to whom I am 
indebeted for sharing knowledge with me.  The work is indeed a co-authored 
effort, which has evolved over a number of years.  Both authors have made 
substantive contributions to the ideas, derivations and empirical results
presented in the paper.
\vspace{1cm}

\begin{center}
\textbf{A Note on Reproducability}
\end{center}

\noindent In the interests of openness, code to replicate the entirety of this 
thesis is publicly available for inspection at 
\href{https://github.com/damiancclarke/PhD}{https://github.com/damiancclarke/PhD}. 
This code allows any and all interested 
parties to fully generate any result from the thesis.  All code is released 
under the GNU General Public License, implying that it is free software.  The 
thesis can be replicated from source (including the download of publicly 
available data) on any machine, provided that a number of languages or compilers 
are installed.  These requirements are (the free) Python, R, Octave, Fortran, 
git and LaTeX, and the (non-free) Stata.  Replicating the entire analysis is 
relatively easy.  On a Unix machine: 
\vspace{1mm} \\
\indent\texttt{\$ git clone https://github.com/damiancclarke/PhD.git}\\
\indent\texttt{\$ make}
\vspace{1mm} \\
runs the entirety of the analysis from source.  Printed copies of this thesis
contain a microSD card containing this and data.  Full details are provided at
https://github.com/damiancclarke/PhD/wiki.

%Note: add note that following program files are released with this thesis:
% - plausexog.ado
% - arrowgraph.ado, etc.

