%EMAIL MARTHA BAILEY FOR ARTICLE ON CHILDREN'S OPPORTUNITIES (listed on her site)
%-------------------------------------------------------------------------------
\begin{chapabstract}
Child-bearing decisions are not made in isolation.  They are taken in concert 
with decisions regarding work, marriage, health investments and stocks, as well 
as many other observable and non-observable considerations.  Drawing causal 
inferences regarding the effect of child birth is complicated by these 
endogenous factors.  This review chapter outlines the identification of causal
estimates of fertilty, and the assumptions underlying the range of estimators
and methodologies proposed in the economic, as well as the non-economic, 
literature.
\end{chapabstract}

\newpage
\section{Introduction}
Human decisions regarding births, and how these decisions affect individual 
outcomes, are central to human welfare.  They are also widely relevant to the
world population.  In 2014, 18.7 per every 1,000 people had a child, which is 
the equivalent of 4.3 births per second \citep{CIA2014}. Over the course of her 
lifetime, the average woman in 2013 will have 2.46 births, down from 4.98 in 
1960 \citep{WorldBank2015}. The importance of choice, and control over the 
timing and number of children has been documented as early as \emph{c}1800 BC, 
with discussions of a range of contraceptive methods included in the Kahun 
Gynaecological Papyrus \citep{OdowdPhillip1994}.  This review chapter examines 
the effect of childbirth on human outcomes.

I review the literature on the effect of individual fertility behaviours on
individual outcomes: namely, the microeconomic effects of fertility.  This
chapter, and indeed the literature on which it is based, largely focuses on 
the effect of a woman or family's fertility decisions on the mother's life 
outcomes and the outcomes of her children.  Some, although relatively less, 
focus is paid to the effect on her partner (if present).  

Rather than focus on correlations between fertility and other outcomes, this 
article is centered on the \emph{causal} effects of fertility.  I discuss
the theoretical and empirical requirements regarding how to infer causality 
in a behaviour which cannot be manipulated directly in an experimental context.  
Given the interruptive nature of child birth on a large range of other life 
outcomes, any study of causal effects must isolate changes in fertility from 
corresponding changes in simultaneously determined, or dynamically dependent, 
outcomes.  As a simple example, if women jointly choose to exit the labour 
market and have a child, any inference regarding the effect of fertility on her 
or her child's \emph{other} outcomes must be independent of her labour market 
choice.

Since the boom in fertility related research in microeconomics in the mid-%
1970's,\footnote{From 1970 onwards, the frequency of the occurrence of the word 
``fertility'' 
in titles of all articles published in the \emph{Journal of Political Economy} 
is 46.5 per 100,000 words. Prior to the 1970s, the frequency was 1.45 per 
100,000 words. Though admittedly based on a small sample of, the frequency by 
decade is 11.17 (`60s), 72.72 (`70s), 66.33 (`80s), 29.41 (`90s), and 41.07 
(`00s). It was not mentioned in titles of articles published between 1863 and 
the 1960s.} a range of methodologies have been proposed to permit inference 
in precisely these circumstances. These include the use of instrumental 
variables (IV), combining difference-in-difference (DD) with IV, the use of 
quasi-experimental fertility shocks, or trying to artificially construct a 
treatment and control group using family members or other matching 
methods.  In the sections which follow, I describe these methodologies, the 
papers which propose and estimate them, and the identifying assumptions that
motivate causality in each case.

Empirical considerations regarding the causal effects of fertility require the
consideration of (at least) three questions: `Effects on what?', `Effects at
what margin?', and `Timing or quantity?'.  Frequently, these questions are 
split further, and examined for specific groups of women or children.

Regarding the first question, the theoretical and empirical microeconomic 
literature has hypothesised that marginal births may have causal implications 
for many individual-level outcomes.  This includes effects on children: their 
health indicators, cognitive and non-cognitive achievements, long-term labour 
market outcomes, education inputs, and social outcomes such as age at marriage 
and crime incidence; as well as effects on parents. Parental outcomes often 
considered are centered around labour market participation and returns, rates 
of education completion, marriage market outcomes and socioeconomic indicators 
such as welfare-receipt.  I outline these measures in the following sections, 
along with the evidence that a causal link exists (or does not exist) with 
fertility.

Human fertility decisions exist at two (very different) margins.  The choice of
whether or not to have a child (the extensive margin), and, conditional on 
choosing to have any children, the decision regarding \emph{how many} births
to have (the intensive margin).  The nature of links between fertility decisions
and outcomes vary considerably when considering the intensive and the extensive
margin.  The consensus in the literature is that the causal effects are 
certainly not a linear function of births, with important non-linearities, and 
indeed non-monotonic, relationships described between fertility and (some) of 
the previously mentioned outcomes. In order to quantify effects at different 
margins, a range of estimation samples and methodologies need be employed.  

Finally, causal effects of fertility are not stationary by mother's age at 
birth.  Often, rather than estimating the effect of a marginal birth, we will be
interested in determining the effect of child birth at a particular age (such as
during adolescence).  Considerations of these effects have important life-cycle
implications for future investment decisions.

The study of fertility is common to a huge range of fields: social sciences, 
physical sciences, demography and medicine, and in many subfields within 
disciplines.  As a result, any moderate-length review of the literature must
be necessarily pointed.  As such, this chapter firmly focuses on the effects
of fertility on other individual-level outcomes, and not the determinants of
fertility\footnote{More recently, \citet{KearneyLevine2012} provided an extensive 
discussion of teenage childbearing and its determinantes in the USA, 
\citet{Bailey2013} reviewed the old and new evidence on the effect of access to 
contraception, and \citet{Moffitt2005} has provided a discussion and application 
of causal inference to the effects of teenage childbearing on child outcomes. All
provide extremely useful reviews of the relevant literature.  The handbook chapter
of \citet{Schultz2008} is perhaps the definitive reference for microeconomists
interested in an analysis with a very broad scope.  There are many papers
discussing education and fertility (ie \citet{Blacketal2008}), which won't be 
discussed in this chapter.}, the macroeconomic effects \citep{Enke1966,Enke1971}, 
or discussions of the broader effects of population control policies
\citep{GMiller2009,RosenzweigWolpin1986}.

In what remains of this chapter I briefly layout a number of considerations
surrounding causal analysis of endogenous fertility decisions, before turning to
particular methodologies and samples used to generate empirical estimates.  I
consider the causal effects of fertility on child and on parental outcomes in
term, outlining main results from relevant papers in the field.

%-------------------------------------------------------------------------------
\section{Causality and Fertility}
\label{Fscn:causality}
We define an outcome $Y_i$, for each member $i$ of a sample, $i\ \in \{1,\ldots,
N\}$, where $Y$ denotes an outcome variable of interest, and the sample is drawn
from a population of childbearing mothers (or, as discussed later, their 
children). We are 
interested in determining the effect of manipulations of fertility, which we
denote $F_i$, on our outcome variable of interest $Y_i$.  It is assumed that 
$Y_i$ is a function of fertility, an unobserved variable $U_i$, and a series of 
other variables which are summarised as the error term $\varepsilon_Y$:
\begin{equation}
\label{Feqn:outcome}
Y_i=f_Y(F_i,U_i,\varepsilon_Y).
\end{equation}
Fertility is assumed to be a function of the unobserved $U_i$, and stochastic 
$\varepsilon_F$:
\begin{equation}
\label{Feqn:fert}
F_i=f_F(U_i,\varepsilon_F),
\end{equation}
and finally $U_i=f_U(\varepsilon_U)$. These error terms $\bm\varepsilon$ are 
assumed mutually independent. To fix ideas, we could consider an outcome variable 
$Y_i$ as average years of education of $i$'s children, $F_i$ as completed 
fertility, and $U_i$ as unobserved positive health behaviours of the mother. By 
iterative substitution of the $\bm\varepsilon$ terms into (\ref{Feqn:outcome}) 
and (\ref{Feqn:fert}) it becomes apparent (in the defined system of equations) 
that changes in fertility and health are unrelated to $\varepsilon_Y$, but that 
both average years of education and fertility are related to unobserved maternal 
health behaviours.\footnote{\emph{Ie} $F_i$ and $U_i$ are not functions of 
$\varepsilon_Y$, however $Y_i$ and $F_i$ are functions of $\varepsilon_U$.}

In causal terms as per \citet{Haavelmo1943,Haavelmo1944} (and particularly, the
recent exposition in \citet{HeckmanPinto2015}), we are interested in the change
in $Y$ resulting from the hypothetical manipulation of fertility $F$, while 
other elements of the system of equations ($U,\bm\varepsilon$) remain unchanged.  
We define $b$ as a particular draw of $F$, and are thus interested in the causal 
effect of manipulating fertility from $b$ to $b+1$, which throughout this paper
we will call $\beta$:\footnote{For now we make no distinction between different 
values of $b$ in (\ref{Feqn:estimand}).  Generally however, we will be interested 
in at least two seperate situations. The first, comparing having any children 
to having no children (the extensive decision), while the second refers to 
having $b+1$ children versus having $b$ children for $b\in{1,\ldots,k}$ (the 
intensive margin).  We return to discussions of $b$ for different parities when
discussing empirical results in the sections which follow.}
\begin{equation}
\label{Feqn:estimand}
\beta\equiv\mathbb{E}_{U_i,\varepsilon_Y}[Y_i(b_i+1)-Y_i(b_i)]
\end{equation}

The hypothetical manipulation envisioned by \citeauthor{Haavelmo1943} is 
generally not feasible in real-world fertility decisions. And in observational
studies using data over space or time, the presence of factors similar to $U$
considerably hinders the estimation of causal effects.  In the sections which
follow we return to this system of equations, and outline the existing 
techniques, and empirical results, which aim to recover causal estimates despite
the lack of explicit exogenous manipulation of $F$.

%-------------------------------------------------------------------------------
\section{The Effects of Family Size on Children}
\label{Fscn:kids}
In 1973, the \emph{Journal of Political Economy} released a special issue 
dedicated to new economic approaches to fertility.  Interest in determining
the causal effect of total fertility on the outcomes of children in the 
household blossomed from the articles it contained.  A common theme in a number 
of articles in this issue \citep{BeckerLewis1973, DeTray1973, Willis1973} 
concerns a family's decisions regarding fertility (the quantity of children) 
and investments in child human capital (the `quality' of children). Abstracting 
from intra-household variations in child quality\footnote{An extensive 
literature exists which looks at \emph{intra}-household endowment and investment
decisions among siblings.  \citet{Behrmanetal1982} provides initial discussion,
and \citet{AizerCunha2012} embed these considerations in a quantity--quality-type 
framework.}, each of the aforementioned articles demonstrates the theoretical 
existence of a quantity--quality (Q--Q) trade-off.\footnote{As \citet{Willis1973} 
succinctly describes:
\begin{quote}
`Thus, parents not only balance the satisfactions they receive from their
children against those received from all other sources not related to 
children \ldots, but they must also decide whether to augment their 
satisfaction from children at the ``extensive'' margin by having another
child or at the ``intensive'' margin by adding to the quality of a given
number of children.'
\end{quote}
}

The Q--Q trade-off described in the above series of articles as well as in
\citet{BeckerTomes1976,BeckerTomes1986} owes to the joint entry of quality and 
quantity in the household budget constraint.  As the number of children 
enters in the shadow price of quality, and the quality of desired children 
enters the shadow price for quantity, decisions regarding fertility and quality 
cannot be made in isolation.  Holding all else constant, increases in fertility 
increase the shadow price of quality, and increases in quantity increases the
shadow price of the marginal birth. This considerably complicates causal 
inference. What's more, as recognised in early articles by \citet{%
BenPorathWelch1972,BenPorath1976}, quality decisions may \emph{directly} feed 
back to quantity via child mortality.

%-------------------------------------------------------------------------------
\subsection{Observational Data}
Given the aforementioned theoretical structure of the relationship between 
child quality and child quantity, it is apparent that estimating OLS on 
observational data will lead to consistent estimates of $\beta$ only in very
particular circumstances. To see this, we return to equation \ref{Feqn:outcome}.
If we consider standard OLS with a linear model, we re-write 
(\ref{Feqn:outcome}) as:
\[
Y=\beta F + U + \varepsilon_Y,
\]
where we assume that $\mathbb{E}[\varepsilon_Y]=0$.  To estimate $\beta$ from
the above, we can consider conditioning on two distinct values of $F$:
\begin{eqnarray}
\mathbb{E}[\hat\beta] & = & \mathbb{E}[Y|F=b+1]-\mathbb{E}[Y|F=b]. \label{Feqn:OLS} \\
                      & = & \mathbb{E}[\beta(b+1)+U|F=b+1] - \mathbb{E}[\beta(b)+U|F=b] 
\nonumber \\
                      & = & \beta + \{\mathbb{E}[U|F=b+1] - \mathbb{E}[U|F=b]\} \nonumber
\end{eqnarray}
Thus, (\ref{Feqn:OLS}) is only identical to the causal estimate in 
(\ref{Feqn:estimand}) in a very limited set of circumstances: above this is when
$\mathbb{E}[U|F=b+1] = \mathbb{E}[U|F=b]$.  This is simply a specific example of 
the well-known OLS requirement that the independent variable of interest ($F$) 
must be uncorrelated with the omitted error term, given that 
$\plim(\hat\beta)=\beta+\Cov(F,U)/\Var(F)$.  Insofar as variation in fertility
in a cross-sectional data set is correlated with movements of other variables
related to the outcome of interest (which here we summarise as $U$), we will fail 
to identify the true causal effect of fertility given the lack of 
\citeauthor{Haavelmo1943}'s hypothetical manipulation of $F$. 

Due to the limitations laid out in the preceding paragraph, very few papers in 
the literature aim to infer causality by estimating linear models with 
cross-sectional data.\footnote{Early work, such as \citet{Desai1995}, provides 
across-sectional descriptive 
evidence to document correlations, while \citet{Hanushek1992} estimates some
cross-sectional (though value-added) models.}  However many papers which use 
alternative methods to infer causality (discussed in the sections which follow)
estimate OLS as a base specification, which can provide some information on the 
type and degree of bias in OLS.  Beyond recognising that a bias is likely to 
exist, relatively few of these papers provide an explicit discussion of why this
may be.  Notable exceptions include \citet{Qian2009}, who suggests joint
parental preferences for more education and fewer children as well as optimal
stopping rules which depend on the quality of the first child, and 
\citet{Blacketal2010}, who additionally note that family size effects are 
confounded with birth order effects.  Indeed there are a number of reasons one
could use to suggest bias.  These include parental education, discount rates,
maternal health or network effects driving both fertility and child quality.  
Generally it seems likely that these factors will cause OLS estimates to 
induce a negative bias in estimates of the effect of fertility, given that 
factors which lead to fewer births (contraceptive knowledge, opportunity cost of
time, aspirations, and so forth) also seem likely to drive greater investments 
in children who are eventually born.  Empirically, this overwhelmingly seems to
be the case, with OLS estimates of the effect of fertility being universally 
lower (more negative) than more credibly causal estimates.  We return to provide
more details on these estimates in the sections which follow.

%-------------------------------------------------------------------------------
\subsection{Instrumental Variables}
\label{Fsscn:kidIV}
In systems of equations of the type described in \ref{Fscn:kids}, one way to 
drive inference is through the
use of shifters (or instrumental variables) which affect the quantity of one of
the variables without affecting the other.  In order to identify the effect of 
fertility on children's outcomes, this instrumental variable must affect only 
fertility, with no indirect effects on quality.\footnote{\emph{Ie} the exclusion 
restriction must hold, implying that the estimation of the structural equation 
which contains quality on fertilty and the instrument must result in a 
coefficient on the instrumental variable which is precisely equal to zero.}
Returning to the nomenclature introduced in section \ref{Fscn:causality},
consider $Y_i$ as child quality, $F_i$ as child quantity, and the unobserved
$U_i$, all generated as described in \ref{Fscn:causality}.  However, now consider 
the case where a new variable from outside the system is observed, denoted 
$Z_i$.  $Z_i$ is assumed to directly affect $F_i$:
\[
F_i = f_F(U_i,Z_i,\varepsilon_F).
\]
The function \ref{Feqn:outcome} is unchanged, reflecting the fact that the only
channel with which $Z_i$ affects $Y_i$ is through $F_i$ (ie the exclusion 
restriction holds).  Finally, to close, assume that $Z_i=f(\varepsilon_Z)$,
and once again the error terms $\bm{\varepsilon}$ are assumed mutually 
independent.

The above situation leads to an explicit way to generate the hypothetical 
variation discussed in section \ref{Fscn:causality}.  By taking advantage of 
variation in $F$ induced by variation in $Z$, the effect of $F$ on $Y$ can
be identified in the absence of any movement in $U$.  The most simple way to
consider this is by considering the Wald estimator:
\begin{eqnarray}
\label{Feqn:Wald}
\hat\beta = \frac{\mathbb{E}[Y|Z=1]-\mathbb{E}[Y|Z=0]}{\mathbb{E}[F|Z=1]-\mathbb{E}[F|Z=0]}.
\end{eqnarray}
Here rather than explicitly being based on a movement from $F=b$ to $F=b+1$
estimation is driven by the effect which $Z$ has on $Y$, scaled by the
degree to which it moves $F$.  If the instrument increases birth by exactly
one, then (\ref{Feqn:Wald}) collapses to an expression similar to 
(\ref{Feqn:OLS}).  Fundamentally in strategies of this type, the identifying
assumption shifts from concerns regarding correlations between $U$ and $F$
to correlations between $U$ and $Z$.  Consistent causal estimation now 
requires that $\Cov(U,Z)=0$.

The earliest discussion of these types of shifters and the corresponding 
exclusion restriction required for the estimation of the causal effects of 
fertility was in \citet{RosenzweigWolpin1980}. They point out that if multiple 
births are unanticipated, their occurrence will cause some families to exceed 
their desired fertility, shifting the total number of births in the absence of 
any change in parental considerations of quality investments. This has motivated 
estimation in a number of papers, where twin births are employed as instrumental 
variables.  Twin instruments have been employed in a range of contexts and
to examine various different `quality' outcome variables of children. These 
include \citet{Blacketal2005,Caceres2006,Lietal2008,Dayiogluetal2009,Sanhueza2009,
Blacketal2010,Angristetal2010,FitzsimonsMalde2010} and \citet{SouzaPonczek2012}, 
and focus on child quality measures including years of education, IQ, private 
school enrollment, BMI and height, college completion and age at marriage.  The 
evidence on the existence of a Q--Q trade-off in these studies is mixed, 
although recent influential results suggest that the evidence in favour of a 
trade-off may be weak.  In table \ref{Ftab:childQQ} I lay out outcome variables, 
contexts, and estimates of $\beta$ presented in the IV literature.

As per the above series of equations, causal estimates rely on the fact that
$Z$ truly is independent of $U$.  This has been questioned in a number of ways.
\citet{RosenzweigZhang2009} suggest that the close birth-spacing of twins, and 
the fact that twins have lower health stocks at birth \citep{Almondetal2005}
means that parents may change behaviours to reinforce or compensate for intra-%
household human capital differences.  While this can be tested directly, it 
requires data on early live human capital endowments such as birthweight.  
\citet{BhalotraClarke2015} question the exogeneity assumption in another way.
They demonstrate that healthier mothers are more likely to take twin births
to term, and at the same time that healthier mothers are more likley to have
additional resources to invest in child quality later in life.  This critique is
laid out and estimated in chapter \ref{chap:twins} of this thesis.  At the very
least however, both of these critiques will lead to predictable biases in 
estimates of $\beta$, resulting in bounds on the effect of fertility on child
outcomes.

A frequently used alternative to twin births consists of instrumenting with the 
gender mix of children born in the family. Generally, it is argued that parents
prefer to have offspring of both genders \citep{ConleyGlauber2006,Angristetal2010,
Beckeretal2010,MillimetWang2011,FitzsimonsMalde2014}, and so those having various 
children of the same sex are more likely to continue childbearing. Alternatively, 
in some circumstances it is argued that parents have a son preference, and so are 
more likely to continue after having early birth girls \citep{Lee2008,
KumarKugler2011}. In both cases these are empirically shown to be important 
drivers of fertility.  Again, like estimates driven by twin births, empirical 
results are mixed, although recent evidence seems to point to statistically 
insignificant (though nearly universally negative) estimates of the trade-off, 
as outlined in panel B of table \ref{Ftab:childQQ}.

Causality in this case requires that child sex mix has no direct effect on 
quality.  This implies (among other things), that there are no gender-specific 
economies of scale which facilitate child quality investments more when children
are of the same sex \citep{ButcherCase1994}.  While one could argue (and indeed
hope) that goods which could be employed in the household for boy's education 
could also be employed for girl's education, generally there are other concerns.
\citet{DahlMoretti2008} show that gender composition affects the likelihood that
parents live together. \citet{ButcherCase1994} provide extensive discussion of 
the potential that different child gender mixes may affect child costs, and 
demonstrate that in the USA girls with sisters are significantly less educated 
than girls with brothers, postulating that this may be due to a reference group 
effect where parents have lower aspirations for their children when all children 
are girls. Concerns such as these cast doubt on the validity of the exclusion 
restriction described earlier in this section.

A range of other instruments have been proposed, including infertility
\citep{Bougmaetal2015}, miscarriage \citep{Hotzetal1997,Marlani2008,Miller2009}
and distance to family planning \citep{DangRogers2013}. The outcomes and 
empirical results related with these studies are displayed in table 
\ref{Ftab:childQQ}.  While these instruments---all generally realted to the 
ability to conceive or control conception---clearly drive fertility, in each 
case the exclusion restriction is questionable.  This is explicitly treated in
\citet{Hotzetal1997}, who motivate techniques to recover bounds on the estimate
of the effect of causality.  At the very least, in each case if unhealthy
women are more likley than healthy women to be infertile or suffer miscarriage,
this suggests a positive bias in IV estimates of $\beta$.

\begin{table}[htpb!]
\caption{Empirical Results: Fertility and Child Outcomes (IV)}
\label{Ftab:childQQ}
\begin{tabular}{lllc} \toprule
\textsc{Author} & \textsc{Country} & \textsc{Outcome} & \textsc{Estimate} \\
                &                  &                  & \textsc{(Std Err)} \\ \midrule
\multicolumn{4}{l}{\textbf{Panel A: Twins}} \\
\citet{Blacketal2005}            &Norway   & Yrs of Educ         &-0.16(0.44) \\
\citet{Caceres2006}              &USA      &Private School       &-0.000(0.005)\\
                                 &         &Behind cohort        & 0.005(0.004)\\
\citet{Lietal2008}               &China    &Educ (categorical)   &-0.027(0.014)\\
                                 &         &Educ (enrollment)    &-0.025(0.013)\\
\citet{Dayiogluetal2009}         &Turkey   &Attendance           & 0.203(0.245)\\
\citet{Sanhueza2009}             &Chile    &Yrs of Educ          &-0.280(0.092)\\
\citet{Blacketal2010}            &Norway   &IQ (standardised 1-9)&-0.170(0.052)\\
\citet{Angristetal2010}          &Israel   &Yrs of Educ          & 0.167(0.117)\\
                                 &         &Some college         & 0.059(0.036)\\
                                 &         &College grad         & 0.052(0.032)\\
\citet{FitzsimonsMalde2010}      &Mexico   &Yrs of Educ (F)      & 0.096(0.063)\\
                                 &         &Enrolment (F)        &-0.019(0.014)\\
\citet{SouzaPonczek2012}         &Brazil   &Yrs of Educ (F)      &-0.634(0.194)\\
                                 &         &Yrs of Educ (M)      &-0.060(0.164)\\ \midrule
\multicolumn{4}{l}{\textbf{Panel B: Gender Mix}} \\
\citet{ConleyGlauber2006}        &USA      &Private school       &-0.061(0.021) \\
                                 &         &Grade repetition     &0.007(0.004)  \\
\citet{Lee2008}                  &Taiwan   &Total ln(educ spend) &0.328(0.088)  \\
\citet{Angristetal2010}          &Israel   &Yrs of Educ          &-0.067(0.120)\\
                                 &         &Some college         &-0.025(0.025)\\
                                 &         &College grad         &-0.032(0.022)\\
\citet{Beckeretal2010}           &Prussia  &Enrolment            &-0.430(0.189)\\
\citet{KumarKugler2011}          &India    &Yrs of Educ          &-0.363(0.061)\\
\citet{FitzsimonsMalde2014}      &Mexico   &Yrs of Educ (F)      &-0.015(0.125)\\ 
\citet{MillimetWang2011}         &Indonesia&BMI for Age          & 0.049(0.013)\\
\midrule
\multicolumn{4}{l}{\textbf{Panel C: Fertility Shock}} \\
\citet{Bougmaetal2015}           &Burkina Faso&Yrs of Educ        &-0.99(0.40)\\ 
\citet{Marlani2008}              &Indonesia   &Yrs of Educ (early)&-0.167(0.117)\\ 
                                 &            &Yrs of Educ (late) &-0.054(0.055)\\
\citet{Hotzetal1997}             &USA         &Complete highschool&-0.147(0.406)\\
\citet{DangRogers2013}           &Vietnam     &Yrs of Educ        &-0.589(0.392)\\
                                 &            &Private tutoring   &-0.318(0.147)\\
\bottomrule
\multicolumn{4}{p{12.9cm}}{\begin{footnotesize}\textsc{Notes:} In the case that 
various samples are reported in the papers, the pooled estimate for female and
male children of all women from the most recent time period is reported. In the 
case of twins estimates, the 3+ sample (twins at third birth as instrument) is
reported.  Where the original studies report $p$-values associated with estimates 
rather than standard errors, these are converted into standard errors for 
inclusion in this table.
\end{footnotesize}}
\end{tabular}     
\end{table}

Beyond general threats to inference discussed in this section, instrumental 
variable estimates lead to the question of `inference for whom?'.  Estimates 
based on IV lead to a local average treatment effect (LATE), not an average 
treatment effect for the population in general \citep{ImbensAngrist1994}.  This 
LATE implies that any estimates of $\beta$ holds for that group of the population 
who would be induced to change their behaviour (ie their fertility) by the 
instrument in question.  Thus, all instrumental estimates (even assuming causality)
should be cast in terms of the sub-population (compliers) of interest.  This is
a point explicitly discussed in \citet{Angristetal2010}, who suggest that the
twin instrument is relevant for the whole population, while sex-composition 
instruments are relevant for only certain groups. \citet{RosenzweigWolpin1980}'s 
original article, although based on a reduced form equation, suggests that twin 
births are relevant for a more specific group than that suggested by 
\citet{Angristetal2010}: namely, those families who have a twin birth where
the twin birth causes them to exceed their desired fertility.  I return to discuss 
LATE and external validity in the following section of this paper.

%Make table of potential biases...  Ie we know that coefficient between 
%instrument and fertility is +/-.  Then if correlation between instrument
%and error is +/-, this implies:


%-------------------------------------------------------------------------------
\subsection{Natural Experiments}
An alternative manner to deal with correlation between $F$ and $U$ consists of
taking advantage of externally defined (to $U$) reforms.  If reforms are 
applicable to a subgroup of a particular population and are designed to affect
fertility, this suggests a natural `treatment' and `control' group which can
be compared.  Those who receive the fertility reform are considered treated, and
those who don't are considered as controls.  If reforms are truly put in place
for reasons entirely divorved from $U$, causal conclusions can be drawn regarding
the effect of the reform.  Typically the effect of reforms is estimated using 
difference-in-differences (diff-in-diff).  This compares pre-reform differences
between treated and control units with post-reform differences, inferring that
any change in the level of differences is driven by the reform, or stated in
another way, that \emph{no} differential and simultaneously occurring phenomena
seperate treatments from controls.  This is the well known `parallel trends
assumption'.  

These studies can be broadly split into two groups: those which examine the
effect of public policies or other natural experiments on fertility itself,
and those which leverage the externally-defined effect on fertility to quantify
the effect of fertility on some other outcome.  In the latter case, the 
differentiation between difference-in-differences and IV estimates is artificial,
as the (diff-in-diff estimated) effect of the policy on fertility is simply 
plugged in as the first stage in a 2SLS IV framework.\footnote{\citet{Duflo2001} 
is a well known example of this design.  We discuss examples of this framework 
applied to fertility later in this section and in \ref{Fsscn:motherNExp}.} The 
first set of studies are of fundamental importance in analysing the 
\emph{determinants} of fertility and the effect of new contraceptive methods on 
life-cycle childbearing, but do not directly quantify the causal effects of 
fertility itself.  Nevertheless, given their relevance both as a first stage in 
causal estimates and as a reduced form estimate itself, I outline a number of 
these studies below, before moving on to a more comprehensive discussion of their 
link to causal estimates.

\begin{table}[htpb!]
\caption{The Estimated Effect of Reforms on Fertility}
\begin{tabular}{lccl} \toprule
Author & Abortion Effect & Pill Effect & Note \\ \midrule
\citet{AngristEvans1996}      & -0.012(0.004) &               & $a,(x=19)$   \\
\citet{Levineetal1996}        & -0.019(0.007) &               & $b$          \\
\citet{Gruberetal1999}        & -0.059(0.005) &               & $c$          \\
\citet{Bailey2006}            & -0.093(0.043) & -0.074(0.057) & $a,d,(x=22)$ \\
\citet{Guldi2008}             & -0.100(0.054) & -0.085(0.041) &              \\
\citet{Bailey2009}            & -0.012(0.007) &  0.028(0.048) & $a,d,(x=22)$ \\
\citet{OltmansHungerman2012}  & -0.043(0.015) & -0.088(0.023) &              \\
\bottomrule
\multicolumn{4}{p{12.2cm}}{\begin{footnotesize}\textsc{Note:} All figures report 
the results of short term access of a fertility reform on birth rates of young 
women unless otherwise specified in notes. \newline 
$^a$ Binary model with outcome 1=first birth by age $x$. \newline
$^b$ Estimate expressed as births per woman.  Mean rate is 0.110 \newline 
$^c$ Estimate for states adopting 1974-1975. Estimate for 1971-1973 is 
-0.021(0.005). \newline
$^d$ The second version (2009) is an erratum, fixing coding errors from 2006.
\end{footnotesize}} \\
\end{tabular}
\end{table}

Of the large number of studies which use reforms of fertility-control policies%
\footnote{For abortion: \citet{Ananatetal2007,Ananatetal2009,AngristEvans1996,
CharlesStephens2006,Cookeetal1999,Currieetal1996,Gruberetal1999,Guldi2008,
KaneStaiger1996,Levineetal1996,Levineetal1996b,Levineetal1999,PopEleches2005,
PopEleches2006}, for the oral contraceptive pill: \citet{OltmansHungerman2012,
Bailey2006,Bailey2011,Bailey2012,Bailey2013,Christensen2012,Goldin2006,
GoldinKatz2002a,GoldinKatz2002b,KearnerLevine2009} and for the emergency 
contraceptive pill: \citet{Durrance2013,Grossetal2014}.} to examine the effect 
on fertility in a 
diff-in-diff-style framework, only a relatively small number then employ this 
as the first stage to estimate the causal effect of fertility---the focus of
this review chapter.  Among those which \emph{do} directly estimate the effect
of fertilty on child outcomes are \citet{Gruberetal1999,Ananatetal2009} and
\citet{OltmansHungerman2012}. \citet{Gruberetal1999} examines the effect of 
fertility (via 2SLS) on the likelihood that a child lives with single parents, 
lives in poverty, receives welfare, and on rates of infant mortality and low 
birth rates.  Of these, it is suggested that fertility significantly increases 
the probability of living in poverty and having single parents, as well rates 
of infant mortality. \citet{Ananatetal2009} also examine these outcomes, and 
suggest that in the long-run, the marginal child is more likely to have lived 
with a single parent, receive welfare, and not have graduated college.
Finally, \citet{OltmansHungerman2012} return to these same outcomes and report
a Wald ratio as in (\ref{Feqn:Wald}).  These Wald estimates allow them to look
at the characteristics of marginal child not born due to both the diffusion of 
the pill, and the legalisation of abortion.  Their results suggest that the two 
fertility control policies had remarkably different effects on marginal child 
characteristics: in agreement with the above studies they suggest that the
marginal child not born due to abortion legalisation would have been 49.2\%
(se=25.5) more likely to live in a welfare-receiving household.  However, the 
marginal child not born due to pill diffusion looks remarkably different: 8.0\% 
(se=4.4) \emph{less} likely to belong to a welfare receiving household.  These 
comparisons make manifestly clear the distinction between compliers for 
different instruments discussed at the end of section \ref{Fsscn:kidIV}.
Given that the group of `compliers' in the two policies had very different
characteristics, estimated effects of fertility on outcomes are very different
despite being plausibly causal in both cases.

Despite not directly estimating the causal effect of fertility on child 
outcomes, a number of other contraceptive-based natural experiment papers
estimate the effect of the natural experiment directly on child outcomes.
This reduced form technique provides an estimate of the numerator of the ratio
in (\ref{Feqn:Wald}), and so can be thought of as an unscaled estimate of
the effect of fertility.  Papers of this type include \citet{PopEleches2006} 
who finds that the illegalisation of abortion in Romania worsens child 
education and labour market outcomes (conditional on parental characteristics)
and \citet{Bailey2013} who reports that US contraceptive pill laws had long-%
standing impacts on children's eventual college completion, labour force 
participation, and family incomes.

\textcolor{red}{Threats to validity --- an important paragraph.}

Finally, a number of other natural experiments have been used in the 
literature to examine the effect of fertility on child outcomes.\footnote{
Similarly, \citet{BleakleyLange2009} use a natural experiment: the eradication
of hookworm in USA, to test the Q-Q hypothesis.  However, the elimination of
hookworm is used as a shifter for child quality, \emph{not} child quantity.
This allows them to quantify the effect of quality increases on subsequent
fertility decisions of households, and they find that increases in quality
do lead to fertility declines in line with the Q-Q model discussed earlier.}
Perhaps most notably among these, \citet{Qian2009} uses the relaxation of 
China's one child policy to estimate the causal effect of movements from one 
child to two child households.  This study is unique for two reasons: the low 
parity shift of the experiment (an expansion from one to two children), and the 
fact that it finds that higher fertility in this case \emph{increases} child 
schooling outcomes, especially among households who have two children of the 
same gender. These results suggests that estimates of fertility at the intensive 
margin may not be linear, and indeed may not be monotonic by parity.

\subsection{Structure and Dynamics}
A number of dynamic or dynamic structural papers motivate estimation of the
effects of fertility (or fertility timing) on birth outcomes in based on a
finite horizon, rather than static, estimation framework.  While these papers
allow for a more extensive examination of \emph{timing} and life-cycle 
decisions, estimation generally relies on an exclusion restriction similar to
those discussed in section \ref{Fsscn:kidIV}.  \citet{RosenzweigWolpin1995} 
motivate the estimation of a dynamic model to examine the effect of early 
fertility (teen motherhood) on child birth outcomes (gestation and birthweight).  
By formulating an over-identified series of equations where identification (in
a FE-IV framework) comes from family background variables, and idiosyncratic
elements shared by siblings. \citet{RosenzweigSchultz1985} take advantage of
variations in fecundity, or births per attempt, which they suggest are 
unobserved by parents prior to contraception attempts, but observable later 
given births per period.  CLOSE SECTION.

%-------------------------------------------------------------------------------
\section{The Effects of Child Birth on Mothers}
\label{Fscn:mothers}
Beyond the analysis of a child's effect on his or her siblings' outcomes, a
birth, at the extensive or the intensive margin, has myriad impacts on parents
or other carers.  The analysis of these effects has received considerable and 
ongoing attention in the economics literature.  Much of the focus of this work
falls on the effect of marginal births on mothers labour market outcomes and
trajectories.

\citet{FleisherRhodes1979} provides a summary of the early literature, with 
considerable coverage also provided in the \emph{JPE} Fertilty issue described
in section \ref{Fscn:kids} \citep{Willis1973,Gronau1973}.  As is the case with
child investment and fertility decisions, choices regarding fertility, labour
market participation, and (adult) human capital attainment are linked, and 
dynamic in nature.  Total fertility, and if childbearing, birth timing have
important impacts on labour market participation, accrued experience, and
wages, while participation, experience and wages also influence timing and
fertility decisions.  Inferring causality in systems of this type is once again
challenging, relying on the use of plausible instruments, natural experiments,
structural estimation, or a combination of methods.

\subsection{Natural Experiments}
\label{Fsscn:motherNExp}
Split into short run and long run
\citet{Bailey2011,Bailey2006,Bailey2013,Bailey2012,Christensen2012,
Guldi2008,KearnerLevine2009,Levineetal1999,AngristEvans1996,
Jacobsenetal1999,AngristEvans1998,Cristia2008}
(\citet{Bailey2009} is an erratum for Bailey 2006). \citet{AngristEvans1996} 
uses 2SLS.

\citet{OltmansHungerman2012} show that long term effect of pill on mother is
more likely to graduate college, less likely to divorce, more likely to have
college educated spouse.  \citet{GoldinKatz2002a} discuss education, marriage,
but not direct effect on fertility. \citet{Baileyetal2012} show that it has
effects on wages over the life cycle (long-term) with a demanding specification.

Miscarriage as natural experiment (\citet{Hotzetal2005,Fletcher2012}).  
\citet{Hotzetal1997} talk about how to bound this, \citet{FletcherWolfe2009} 
provide further discussion.


\subsection{Instrumental Variables}
The use of instrumental variables to examine the effect of fertility on 
\emph{mothers'} outcomes (rather than children's outcomes as described
in section \ref{Fsscn:kidIV}) follows a similar logic to that outlined in
equation (\ref{Feqn:Wald}).  An external variable which has strong effects
on fertility but no direct effects on the outcome of interest except via its
effect on fertility can be used to drive causal estimates.  Instrumental 
variable estimates are a popular methodology employed to determine the 
effect of fertility on mothers.

Outcome variables of interest are typically 

FLFP: \citet{AgueroMarks2008,AgueroMarks2011,ChunOh2002,Caceres2008}
Earnings:
\citet{RosenzweigWolpin1980b} motivates twins in labour supply framework,
though it is not IV.

\citet{Ananatetal2009,Miller2011,
BronarsGrogger1994,KimAassve2006,RosenzweigSchultz1987,
Caceres2006,Hotzetal1997}
or selection with exclusion restriction, which is fundamentally similar: 
\citet{Ribar1994}.

\subsection{Other}
RCT: \citet{DiCensoetal2002}
Between effects: \citet{Holmlund2005,GeronimusKorenman1992} \citet{Ribar1999} 
tests sibling effects models.\\
Matching: \citet{ChevalierViitanen2003,LevinePainter2003}

\citet{RosenzweigWolpin1980b} return to the Beckerian (\citeyear{BeckerLewis1973})
simultaneous equation framework for fertility, child quality, \emph{and} life-%
cycle (mother's) labour supply.  They are the first to use twins to estimate
the structural equation linking fertility and labour supply.  They estimate
that for younger women, additional births reduce labour supply, but this fades
as women age.  Once again---as they indeed highlight---consistent estimation
relies on twins being entirely orthogonal to labour supply.  This assumption 
is questioned in previous sections of this paper. Using the presumed exogeneity
of twins as an identifying assumption, \citet{RosenzweigWolpin1980b} provide a
very interesting series of tests casting considerable doubt on the assumption
that fertility is exogenous to labour supply decisions, as maintained in some 
prevailing literature.

\newpage
\pgfplotstablegetrowsof{\dataB}
\let\numberofrows=\pgfplotsretval

\begin{table}
\caption{Fertility and Mother's Labour Market Outcomes}
% Print the table
\pgfplotstabletypeset[
columns={name,error,beta},
  every head row/.style={before row=\toprule, after row=\midrule},
  every last row/.style={after row=[1ex]},
  columns/name/.style={string type,column name={}},
  columns/error/.style={
    column name={$\hat\beta \pm$ se($\hat\beta$)},
    assign cell content/.code={% use \multirow for Z column:
    \ifnum\pgfplotstablerow=1
    \pgfkeyssetvalue{/pgfplots/table/@cell content}
    {\multirow{\numberofrows}{6.0cm}{\errplot{\dataB}}}%
    \else
    \pgfkeyssetvalue{/pgfplots/table/@cell content}{}%
    \fi
    }
  },
  % Format numbers and titles
  columns/name/.style={column name=Authors, string type, column type={l}},
  columns/beta/.style={column name=$\beta$, string type, column type={S[table-format=-2.2]}},
  columns/ci/.style={column name=$95\%$ CI, string type, column type={S[table-format=-1.2]}},
 ]{\dataB}

\pgfplotstablegetrowsof{\dataA}
\let\numberofrows=\pgfplotsretval

\pgfplotstabletypeset[
columns={name,error,beta},
  every head row/.style={output empty row, after row=\\},
  every last row/.style={after row=[3ex]\bottomrule},
  % Set header name
  columns/name/.style={string type,column name={}},
    % Use the ``error'' column to call the \errplot command in a multirow cell in the first row, keep empty for all other rows
  columns/error/.style={
    column name={$\hat\beta \pm$ se($\hat\beta$)},
    assign cell content/.code={% use \multirow for Z column:
    \ifnum\pgfplotstablerow=1
    \pgfkeyssetvalue{/pgfplots/table/@cell content}
    {\multirow{\numberofrows}{6.0cm}{\errplot{\dataA}}}%
    \else
    \pgfkeyssetvalue{/pgfplots/table/@cell content}{}%
    \fi
    }
  },
  % Format numbers and titles
  columns/name/.style={column name=Authors, string type, column type={l}},
  columns/beta/.style={column name=$\hat\beta$, string type, column type={S[table-format=-2.2]}},
  columns/ci/.style={column name=$95\%$ CI, string type, column type={S[table-format=-1.2]}},
  ]{\dataA}
\\
\begin{small}
\begin{quote}
\textsc{Notes to table:} Points represent coefficients, while error bars represent 95\% confidence intervals.  Estimates are ordered by date of publication.  In the case that various samples are reported in the papers, the pooled estimate for all women from the most recent time period is reported.  In the case of twins estimates, the 3+ sample (twins at third birth as instrument) is reported.
\end{quote}
\end{small}
\end{table}


%-------------------------------------------------------------------------------
\section{Conclusion}
This review chapter serves to provide an overview of the causal estimation of 
the effect of fertility on child and parental outcomes.  It surveys the wide 
range of methodologies employed in the existing microeconometric literature, and 
discusses how various techniques aim to skirt issues of endogenous fertility 
choices.  In each case, I outline the identifying assumptions implicitly or
explicitly invoked, as well as the threats to which these are subject.

The evidence discussed in this chapter is mixed.  While there seems to be 
quite clear evidence in favour of moderate-to-large effects of marginal child 
births and early births on parental labour market outcomes, the existing 
microeconometric child-level estimates are less compelling.  Despite a large
body of theoretical microeconomic work positing that such a Q--Q trade-off may
exist, causal estimates are certainly not conclusive, and seem to suggest that
the trade-off is small or non-existent.  While there are a number of papers 
which \emph{do} find significant effects on a number of outcomes, these are
context- and complier-specific.

In some cases this lack of evidence may be due to threats to exclusion 
restrictions or other identifying assumptions. It is in line with this that 
the present thesis now moves forward.  The remaining chapters are essays on the 
causal estimate---or more specifically, the challenges and involved in causally 
estimating---the effect of fertility and contraceptive programs on human 
outcomes, and some solutions which may be employed to recover bounds and causal
estimates of these effect.

%\newpage
%\begin{table}
%\caption{Requirements for Causality}
%\begin{tabular}{lcc}\toprule
%Instrument & Positive Bias & Negative Bias  \midrule
%Twins & $\Cov(Twin,U)>0$ & & $\Cov(Twin,U)<0$ & 
%\end{tabular}
%\end{table}



%FROM TWIN IV
%Discuss this...
%This is particularly accute for studies which use later born \citet{Glicketal2007}
%children as their subjects, and those who focus on early life variables of 
%children born before twins.

