%-------------------------------------------------------------------------------
\begin{chapabstract}
Child-bearing decisions are not made in isolation.  They are taken in concert 
with decisions regarding work, marriage, health investments and stocks, as well 
as many other observable and non-observable considerations.  Drawing causal 
inferences regarding the effect of child birth is complicated by these 
endogenous factors.  This review chapter outlines the identification of causal
estimates of fertilty, and the assumptions underlying the range of estimators
and methodologies proposed in the economic, as well as the non-economic, 
literature.
\end{chapabstract}

\newpage
\section{Introduction}
Make point about extensive versus intensive margin.

Outcomes.

Identification.  Shifters: twins, fertility policies
Methodologies. IV, DD, (RD?), FE.

Timing as well as number.

General points: \citet{Enke1966} (and others by him: 1960, 1971), 
\citet{KearneyLevine2012, Schultz2008, GMiller2009, RosenzweigWolpin1986}
(maybe add Miller cite to twins).

\citet{Moffitt2005} is a useful starting point.

%Black et al EJ 2008 type citations of other drivers of fertility?

\newpage
\section{Causality and Fertility}
\label{Fscn:causality}
%Discussion of Causality: Pearl, Heckman, Haavelmo etc?

%-------------------------------------------------------------------------------
We define an outcome $Y_i$, for each member $i$ of a population, $i\ \in \{1,
\ldots,N\}$, where $Y$ denotes an outcome variable of interest. We are 
interested in determining the effect of manipulations of fertility, which we
denote $F_i$, on our outcome variable of interest $Y_i$.  It is assumed that 
$Y_i$ is a function of fertility, an unobserved variable $X_i$, and a series of 
other variables which are summarised as the error term $\varepsilon_Y$:
\begin{equation}
\label{Feqn:outcome}
Y_i=f_Y(F_i,X_i,\varepsilon_Y).
\end{equation}
Fertility is assumed to be a function of the unobserved $X_i$, and stochastic 
$\varepsilon_F$:
\begin{equation}
\label{Feqn:fert}
F_i=f_F(X_i,\varepsilon_F),
\end{equation}
and finally $X_i=f_X(\varepsilon_X)$. These error terms $\bm\varepsilon$ are 
assumed mutually independent. To fix ideas, we could consider an outcome variable 
$Y_i$ as average years of education of $i$'s children, $F_i$ as completed 
fertility, and $X_i$ as unobserved positive health behaviours of the mother. By 
iterative substitution of the $\bm\varepsilon$ terms into (\ref{Feqn:outcome}) 
and (\ref{Feqn:fert}) it becomes apparent (in the defined system of equations) 
that changes in fertility and health are unrelated to $\varepsilon_Y$, but that 
both average years of education and fertility are related to unobserved maternal 
health behaviours.

In causal terms as per \citet{Haavelmo1943,Haavelmo1944} (and particularly, the
recent exposition in \citet{HeckmanPinto2015}), we are interested in the change
in $Y$ resulting from the hypothetical manipulation of fertility $F$, while 
other elements of the system of equations ($X,\bm\varepsilon$) remain unchanged.  
We define $b$ as a particular draw of $F$, and are thus interested in the causal 
effect of manipulating fertility from $b$ to $b+1$, which throughout this paper
we will call $\beta$:\footnote{For now we make no distinction between different 
values of $b$ in (\ref{Feqn:estimand}).  Generally however, we will be interested 
in at least two seperate situations. The first, comparing having any children 
to having no children (the extensive decision), while the second refers to 
having $b+1$ children versus having $b$ children for $b\in{1,\ldots,k}$ (the 
intensive margin).}
\begin{equation}
\label{Feqn:estimand}
\beta\equiv\mathbb{E}_{X_i,\varepsilon_Y}[Y_i(b_i+1)-Y_i(b_i)]
\end{equation}

The hypothetical manipulation envisioned by \citeauthor{Haavelmo1943} is 
generally not feasible in real-world fertility decisions. And in observational
studies using data over space or time, the presence of factors similar to $X$
considerably hinders the estimation of causal effects.  In the sections which
follow we return to this system of equations, and outline the existing 
techniques, and empirical results, which aim to recover causal estimates despite
the lack of explicit exogenous manipulation of $F$.

%-------------------------------------------------------------------------------
\section{The Effects of Family Size on Children}
In 1973, the \emph{Journal of Political Economy} released a special issue 
dedicated to new economic approaches to fertility.  Interest in determining
the causal effect of total fertility on the outcomes of children in the 
household blossomed from the articles it contained.  A common theme in a number 
of articles in this issue \citep{BeckerLewis1973, DeTray1973, Willis1973} 
concerns a family's decisions regarding fertility (the quantity of children) 
and investments in child human capital (the `quality' of children). Abstracting 
from intra-household variations in child quality, each of the aforementioned
articles demonstrates the theoretical existence of a quantity--quality
(Q--Q trade-off).\footnote{As \citet{Willis1973} succinctly describes:
\begin{quote}
`Thus, parents not only balance the satisfactions they receive from their
children against those received from all other sources not related to 
children \ldots, but they must also decide whether to augment their 
satisfaction from children at the ``extensive'' margin by having another
child or at the ``intensive'' margin by adding to the quality of a given
number of children.'
\end{quote}
}

The Q--Q trade-off described in the above series of articles as well as in
\citet{BeckerTomes1976,BeckerTomes1986} owes to the joint entry of quality and 
quantity in the household budget constraint.  As the number of children 
enters in the shadow price of quality, and the quality of desired children 
enters the shadow price for quantity, decisions regarding fertility and quality 
cannot be made in isolation.  Holding all else constant, increases in fertility 
increase the shadow price of quality, and increases in quantity increases the
shadow price of the marginal birth. This considerably complicates causal 
inference. What's more, as recognised in early articles by \citet{%
BenPorathWelch1972,BenPorath1976}, quality decisions may \emph{directly} feed 
back to quantity via child mortality.

\textcolor{red}{Discussion of assumptions of Q--Q...
 examine the effect of child 
characteristics (mortality) on family size., and loosening by 
\citet{AizerCunha2012} (from \citet{Behrmanetal1982}'s theory, gender discussion 
of \citet{ButcherCase1994}) recently. \citet{Lawsonetal2012} is one of many 
biological papers.}

%-------------------------------------------------------------------------------
\subsection{Observational Data}
Given the aforementioned theoretical structure of the relationship between 
child quality and child quantity, it is apparent that estimating OLS on 
observational data will lead to consistent estimates of $\beta$ only in very
particular circumstances. To see this, we return to equation \ref{Feqn:outcome}.
If we consider standard OLS with a linear model, we re-write 
(\ref{Feqn:outcome}) as:
\[
Y=\beta F + X + \varepsilon_Y,
\]
where we assume that $\mathbb{E}[\varepsilon_Y]=0$.  To estimate $\beta$ from
the above, we can consider conditioning on two distinct values of $F$:
\begin{eqnarray}
\hat\beta & = & \mathbb{E}[Y|F=b+1]-\mathbb{E}[Y|F=b]. \nonumber \\
          & = & \mathbb{E}[\beta(b+1)+X|F=b+1] - \mathbb{E}[\beta(b)+X|F=b] \nonumber \\
          & = & \beta + \mathbb{E}[X|F=b+1] - \mathbb{E}[X|F=b] \label{Feqn:OLS}
\end{eqnarray}
Thus, (\ref{Feqn:OLS}) is only identical to the causal estimate in 
(\ref{Feqn:estimand}) in a very limited set of circumstances: above this is when
$\mathbb{E}[X|F=b+1] = \mathbb{E}[X|F=b]$.  This is simply a specific example of 
the well-known OLS requirement that the independent variable of interest ($F$) 
must be uncorrelated with the omitted error term, given that 
$\plim(\hat\beta)=\beta+\Cov(F,X)/\Var(F)$.  Insofar as variation in fertility
in a cross-sectional data set is correlated with movements of other variables
related to the outcome of interest, we will fail to identify the true causal
effect of fertility given the lack of \citeauthor{Haavelmo1943}'s hypothetical
manipulation of $F$. 

Given the above discussion, very few papers in the literature aim to infer 
causality by estimating linear models with cross-sectional data.\footnote{
Early work, such as \citet{Desai1995}, provides cross-sectional descriptive 
evidence to document correlations, while \citet{Hanushek1992} estimates some
cross-sectional (though value-added) models.}  However many papers which use 
alternative methods to infer causality (discussed in the sections which follow)
estimate OLS as a base specification, which can provide some information on the 
type and degree of bias in OLS.  Beyond recognising that a bias is likely to 
exist, relatively few of these papers provide an explicit discussion of why this
may be.  Notable exceptions include \citet{Qian2009}, who suggests joint
parental preferences for more education and fewer children as well as optimal
stopping rules which depend on the quality of the first child, and 
\citet{Blacketal2010}, who additionally note that family size effects are 
confounded with birth order effects.  Indeed there are a number of reasons one
could use to suggest bias.  These include parental education, discount rates,
maternal health or network effects driving both fertility and child quality.  
Generally it seems likely that these factors will cause OLS estimates to 
induce a negative bias in estimates of the effect of fertility, given that 
factors which lead to fewer births (contraceptive knowledge, opportunity cost of
time, aspirations, and so forth) also seem likely to drive greater investments 
in children who are eventually born.  Empirically, this overwhelmingly seems to
be the case, with OLS estimates of the effect of fertility being universally 
lower (more negative) than more credibly causal estimates.  We return to provide
more details on these estimates in the sections which follow.

%-------------------------------------------------------------------------------
\subsection{Instrumental Variables}
In systems of equations of this type, one way to drive inference is through the
use of shifters (or instrumental variables) which affect the quantity of one of
the variables without affecting the other.  In order to identify the effect of 
fertility on children's outcomes, this instrumental variable must affect only 
fertility, with no indirect effects on quality.\footnote{\emph{Ie} the exclusion 
restriction must hold, implying that the estimation of the structural equation 
which contains quality on fertilty and the instrument must result in a 
coefficient on the instrumental variable which is precisely equal to zero.}
Returning to the nomenclature introduced in section \ref{Fscn:causality},
consider $Y_i$ as child quality, $F_i$ as child quantity, and omitted factor
$X_i$.  However, now consider a new variable $Z_i$, which affects $F_i$ however
which has no 

The earliest discussion of these types of exclusion restrictions was in 
\citet{RosenzweigWolpin1980}. They point out that if multiple births are 
unanticipated, their occurrence will cause some families to exceed their desired
fertility, shifting the total number of births in the absence of any change in 
parental considerations of quality investments.  This has motivated estimation 
in a number of papers, where twin births are employed as instrumental 
variables.  Twin instruments have been employed in a range of contexts and
to examine various different `quality' outcome variables.  These include
\citet{Blacketal2005,Caceres2006,Lietal2008,Dayiogluetal2009,Sanhueza2009,
Blacketal2010,Angristetal2010,FitzsimonsMalde2010} and 
\citet{SouzaPonczek2012}, and focus on child quality measures including years of 
education, IQ, private school enrollment, BMI and height, college completion and 
age at marriage.  The evidence on the existence of a Q--Q trade-off in these 
studies is mixed, although recent influential results suggest that the evidence 
in favour of a trade-off may be weak.  In table \ref{Ftab:childQQ} I lay out 
outcome variables, contexts, and estimates of $\beta$ presented in the IV 
literature.

As per the above series of equations, causal estimates rely on the fact that
$Z_i$ truly be independent of $X_i$.  Discuss this...
This is particularly accute for studies which use later born \citet{Glicketal2007}
children as their subjects, and those who focus on early life variables of 
children born before twins.

A frequently used alternative to twin births consists of instrumenting with the 
gender mix of children born in the family. Generally, it is argued that parents
prefer to have offspring of both genders \citet{ConleyGlauber2006,Angristetal2010,
Beckeretal2010,MillimetWang2011,FitzsimonsMalde2014}, and so those having various 
children of the same sex are more likely to continue childbearing. Alternatively, 
in some circumstances it is argued that parents have a son preference, and so are 
more likely to continue after having early birth girls \citet{Lee2008,
KumarKugler2011}. In both cases these are empirically shown to be important 
drivers of fertility.  Again, like estimates driven by twin births, recent 
evidence seems to point to statistically insignificant trade-offs, though the 
estimates outlined in panel B of table \ref{Ftab:childQQ} are nearly universally 
negative for a range of outcomes.

Causality in this case requires that child sex mix has no direct effect on 
quality.  This implies (among other things), that there are no gender-specific 
economies of scale which facilitate child quality investments more when children
are of the same sex \citet{ButcherCase1994}, and add note that one would hope 
that this is the case as goods that boys can use for education can also be used 
for girls and vice versa, but that generally there are other concerns (see 
footnote 1 of \citet{RosenzweigZhang2009}). \citet{ButcherCase1994} provide 
extensive discussion of this, and demonstrate that in the USA girls with sisters
are significantly less educated than girls with brothers, postulating that this
may be due to a reference group effect where parents have lower aspirations for
their children when all children are girls. Also \citet{DahlMoretti2008} show
that in USA parents are more likely to live together with sons.

%-------------------------------------------------------------------------------

A range of other instruments have been proposed, including infertility
\citep{Bougmaetal2015}, miscarriage \citep{Hotzetal1997,Marlani2008,Miller2009}
and distance to family planning \citep{DangRogers2013}.

Make table of potential biases...  Ie we know that coefficient between 
instrument and fertility is +/-.  Then if correlation between instrument
and error is +/-, this implies:

\begin{table}
\caption{Empirical Results: Fertility and Child Outcomes}
\label{Ftab:childQQ}
\begin{tabular}{lllc} \toprule
\textsc{Author} & \textsc{Country} & \textsc{Outcome} & \textsc{Estimate} \\
                &                  &                  & \textsc{(Std Err)} \\ \midrule
\multicolumn{4}{l}{\textbf{Panel A: Twins}} \\
\citet{Blacketal2005}            &Norway   & Yrs of Educ         &-0.16(0.44) \\
\citet{Caceres2006}              &USA      &Private School       & -0.000(0.005)\\
                                 &         &Behind cohort        & 0.005(0.004)\\
\citet{Lietal2008}               &China    &Educ (categorical)   & -0.027(0.014)\\
                                 &         &Educ (enrollment)    & -0.025(0.013)\\
\citet{Dayiogluetal2009}         &Turkey   &Attendance           & 0.203(0.245)\\
\citet{Sanhueza2009}             &Chile    &Yrs of Educ          &-0.280(0.092)\\
\citet{Blacketal2010}            &Norway   &IQ (standardised 1-9)& -0.170(0.052)\\
\citet{Angristetal2010}          &Israel   &Yrs of Educ          & 0.167(0.117)\\
                                 &         &Some college         & 0.059(0.036)\\
                                 &         &College grad         & 0.052(0.032)\\
\citet{FitzsimonsMalde2010}      &Mexico   &Yrs of Educ (F)      & 0.096(0.063)\\
                                 &         &Enrolment (F)        &-0.019(0.014)\\
\citet{SouzaPonczek2012}         &Brazil   &Yrs of Educ (F)      & -0.634(0.194)\\
                                 &         &Yrs of Educ (M)      & -0.060(0.164)\\ \midrule
\multicolumn{4}{l}{\textbf{Panel B: Gender Mix}} \\
\citet{ConleyGlauber2006}        &USA      &Private school       &-0.061(0.021) \\
                                 &         &Grade repetition     &0.007(0.004)  \\
\citet{Lee2008}                  &Taiwan   &Total ln(educ spend) &0.328(0.088)  \\
\citet{Angristetal2010}          &Israel   &Yrs of Educ          & -0.067(0.120)\\
                                 &         &Some college         & -0.025(0.025)\\
                                 &         &College grad         & -0.032(0.022)\\
\citet{Beckeretal2010}           &Prussia  &Enrolment            & -0.430(0.189)\\
\citet{KumarKugler2011}          &India    &Yrs of Educ          & -0.363(0.061)\\
\citet{FitzsimonsMalde2014}      &Mexico   &Yrs of Educ (F)      & -0.015(0.125)\\ 
\citet{MillimetWang2011}         &Indonesia&BMI for Age          &  0.049(0.013)\\
\midrule
\multicolumn{4}{l}{\textbf{Panel C: Fertility Shock}} \\
\citet{Bougmaetal2015}           &Burkina Faso&Yrs of Educ        &-0.99(0.40)\\ 
\citet{Marlani2008}              &Indonesia   &Yrs of Educ (early)&-0.167(0.117)\\ 
                                 &            &Yrs of Educ (late) &-0.054(0.055)\\
\citet{Hotzetal1997}             &USA         &Complete highschool&-0.147(0.406)\\
\citet{DangRogers2013}           &Vietnam     &Yrs of Educ        &-0.589(0.392)\\
                                 &            &Private tutoring   &-0.318(0.147)\\
\bottomrule
\multicolumn{4}{p{10cm}}{\begin{footnotesize}\textsc{Notes:} \end{footnotesize}}
\end{tabular}     
\end{table}

\subsection{Natural Experiments}
EMAIL MARTHA BAILEY FOR ARTICLE ON CHILDREN'S OPPORTUNITIES (listed on her site)
\citet{OltmansHungerman2012, Gruberetal1999,PopEleches2006,BleakleyLange2009,
RosenzweigZhang2009, Qian2009,Hossain1989}
(see literature section in Gruber et al., (1999))

\subsection{Other}
Structure: \citet{RosenzweigWolpin1995,RosenzweigSchultz1985,
RosenzweigWolpin1980b}
OLS: \citet{Hanushek1992} is descriptive. \citet{Desai1995}


\section{The Effects of Child Birth on Mothers}
\citet{FleisherRhodes1979} seems to be an early important reference. 
\citet{Willis1973} builds female wages into Q-Q.  Also look at \citet{Reuben1973}.
Make a graph of estimated effects horizontally with error bars (see BMJ 
article for ref).

\subsection{Natural Experiments}
Split into short run and long run
\citet{Bailey2011,Baileyetal2012,Bailey2006,Bailey2013,Bailey2012,Christensen2012,
GoldinKatz2002a,Guldi2008,KearnerLevine2009,Levineetal1999,AngristEvans1996,
Jacobsenetal1999,AngristEvans1998,Cristia2008}
(\citet{Bailey2009} is an erratum for Bailey 2012).

Miscarriage as natural experiment (\citet{Hotzetal2005,Fletcher2012}).  
\citet{Hotzetal1997} talk about how to bound this, \citet{FletcherWolfe2009} 
provide further discussion.

MAKE A GRAPH OF DATES OF REFORMS USED TO IDENTIFY OUTCOMES.

\subsection{Instrumental Variables}
FLFP: \citet{AgueroMarks2008,AgueroMarks2011,ChunOh2002,Caceres2008}
Earnings:

\citet{Ananatetal2009,Miller2011,
BronarsGrogger1994,KimAassve2006,RosenzweigSchultz1987,
Caceres2006,Hotzetal1997}
or selection with exclusion restriction, which is fundamentally similar: 
\citet{Ribar1994}.

\subsection{Other}
RCT: \citet{DiCensoetal2002}
Between effects: \citet{Holmlund2005,GeronimusKorenman1992} \citet{Ribar1999} 
tests sibling effects models.\\
Matching: \citet{ChevalierViitanen2003,LevinePainter2003}

\newpage
\input{\litrloc/motherEsts.tex}
