%-------------------------------------------------------------------------------
In these papers I discuss the causal estimation of the effects of fertility on 
mother and child outcomes.  A number of concerns are raised with existing 
identification techniques, and alternative methodologies to consistently 
estimate the effect of interest are proposed.  These concerns and new techniques 
are illustrated using microdata on 100,000,000 births over the past 100 years.

In the first substantive chapter (written with Sonia Bhalotra), we discuss the 
validity of the use of twin births in fertility research.  We demonstrate that 
twin births are not random.  Successfully taking twins to term depends upon 
positive maternal health behaviours and investments in the periods preceding 
birth. We show that this is of considerable concern for estimation techniques 
which rely on the randomness of twin births to isolate causal effects.  To 
illustrate, we consider the estimation of the child quantity--quality (Q--Q) 
trade-off, and show that existing instrumental variable estimates are 
inconsistent in the contexts examined. Upon partially correcting for the fact 
that twin births are not random, a statistically significant Q--Q trade-off 
begins to emerge.  We close by examining a number of partial identification 
techniques to bound the true effect of fertility on child outcomes.

In the second substantive chapter I examine the effect of fertility control 
policies on the fertility decisions and outcomes of women.  I consider the case 
of the emergency contraceptive pill in Chile.  The staggered arrival of this
technology to Chile over the last decade has resulted in the availability of the 
first safe and legal post-coital birth control policies.  In a context of high
teenage pregnancy rates, difference-in-difference (DD) style estimates suggest 
that this policy has accounted for reductions in short-term teen childbearing 
by as much as 7\%, an effect similar to the arrival of abortion in the USA.  
This policy is also shown to reduce fetal deaths recorded in early gestation 
with no similar reduction in late gestation: suggestive evidence that an
alternative fertility control policy may reduce costly and dangerous illegal 
abortions.

Finally, I turn to the use of DD estimators as a policy-analysis tool.  I discuss
how such estimators perform in the case of reforms which may not be sharply 
demarcated to treatment and control clusters, but rather subject to local 
spillovers or externalities.  I propose an extension of the typical DD estimator: 
a spillover robust DD estimator.  This methodology is applied to estimate the 
effect of two fertility control reforms where women local to treatment clusters 
who were not themselves subject to the reform may nonetheless travel to access 
treatment. 
