%===============================================================================
\begin{chapabstract}
\Person examine the effect of quasi-experimental variation in the availability
of the emergency contraceptive (``morning after'') pill in Chile.  Using censal
data on all births and fetal deaths over the period 2005-2011 \person show that
the availability of the pill reduces pregnancy and early gestation fetal death,
which \person argue proxies for illegal abortion.  These effects are 
particularly pronounced among teenagers and young women: point estimates suggest
a 6.9\% reduction in teenage pregnancy and 4.2\% reduction for 20-34 year olds.   
\Person suggest that diffusion of the morning after pill between quasi treatment 
and control areas played an important role, and suggest a way to estimate 
unbiased treatment effects where the stable unit treatment value assumption does 
not hold locally. This paper is the first to provide censal evidence of the
emergency contraceptive's effect, and the first to examine the technology in a
country where no other (legal) post-coital fertility control options exist. 
\end{chapabstract}

%NOTE: Must remove an end/begin space that is in the paper version...

\section{Introduction}
Undesired pregnancy---particularly among young and adolescent women---is a
considerable contributor to poor maternal and child outcomes, and to a lack of
intergenerational mobility.  The last half-century has seen a remarkable 
increase in contraceptive technology, with considerable impacts on rates of such
undesired pregnancy and with far-reaching consequences for the social and 
productive structure of modern society. The widespread introduction of the oral 
contraceptive pill has brought with it lower birth rates, delays in childbearing 
and marriage, higher rates of human capital attainment and labour market 
participation for women \citep{AngristEvans1996,Bailey2006,GoldinKatz2002a,
GoldinKatz2002b}, reductions in the gender wage differential 
\citep{Baileyetal2012}, and, theoretically at least, more empowered women 
\citep{ChiapporiOreffice2008}.  In the long-run, these outcomes have led to 
generations of children less likely to have divorced parents, and more likely to 
live with college educated mothers \citep{OltmansHungerman2012}.

While the contraceptive pill has had a remarkable impact on a woman's capability
to control the timing of her fertility decisions, these treatments require an
expensive ongoing investment, which is difficult or impractical for certain groups
of women.  In contrast to the rich literature on the effects of the contraceptive 
pill, very little evidence is available regarding the effects of post-coital 
(non-abortive) birth control.  In this paper \person examine the effect of
fully-subsidised provision of the emergency contraceptive pill.  This so called 
``morning after pill'' offers an alternative form of contraception in cases where 
other forms were not used or failed during intercourse, or in the case of rape.

The scarce existing literature on this topic suggests that the emergency 
contraceptive (EC) pill may have had surprisingly little effect on both pregnancy
and abortion \citep{Grossetal2012,Durrance2013}.  Here \person present
considerable evidence that, at least in the case of Chile, access to emergency 
contraception does have significant effects on births and abortions, and that
these effects are concentrated on teenagers and young women.  \Person identify 
a plausibly exogenous policy decision in Chile which affects a woman's access to 
the (fully subsidised) emergency contraceptive pill.  Using censal data on each 
woman's pregnancy status in each year, and the outcomes of each pregnancy in Chile, 
\person demonstrate that the availability of the morning after pill reduced the 
likelihood of pregnancy and illegal abortion, and that this effect was transversal 
rather than being enjoyed overwhelmingly by one social class. 

The reform under examination comes from a series of constitutional challenges 
between 2005-2008, which meant that the introduction of the emergency contraceptive 
pill in Chile was entirely controlled by the Supreme court and constitutional 
Tribunals.  Legal challenges resulted in the 2008 finding that it would be illegal 
for all nationally run health centres and hospitals to prescribe the emergency 
contraceptive pill, however that in each of the 346 municipalities of Chile health 
centres were at liberty to do so.  This resulted  in a situation in which a woman's 
access to the pill entirely depended upon the decisions taken by her mayor.  Due 
to this reform it is shown that around half the municipalities in Chile made the 
pill available, while the other half did not.

Using this reform, \Person estimate the effect that the staggered arrival of the 
emergency contraceptive had on women and children, including its effect on births 
and abortions.  The arrival of this new technology is associated with significant 
reductions in these outcomes.  Further, the effects identified are of considerable
magnitude.  It is estimated that among teenage girls, the widespread availability 
of emergency contraception reduces births by around 7\%, and may more than halve 
rates of illegal abortion.  Among older women the reductions in births and illegal 
abortion are more moderate, however still quantitatively important.  For example, 
among 20-34 year olds, the emergency contraceptive pill reduces births by an 
estimated 4.2\% and appears to reduce illegal abortion by around 20\%.

\nocite{Goldin2006, Bailey2011}
\nocite{KearnerLevine2009}
\nocite{Ananatetal2007,ThomasDouglas1996,Levineetal1996}

Naive estimates of the effect of the morning after pill on pregnancies, 
abortions, and other outcomes, are based on the assumption that the arrival of
the emergency contraceptive to approximately half of the women in the country
had no effect on those women who did not live in areas where the pill was
available.  \Person examine the validity of this assumption by comparing women 
who live `close' to areas where the pill was available to those who live
considerably further away.  It is shown that significant treatment spillovers
may occur, and so suggested that naive estimates off the effect of the emergency 
contraceptive pill may significantly underestimate the true effect of the 
expansion of availability in Chile.

Given the spatial nature of these spillovers, a methodology is proposed which
allows for the recovery of a consistent treatment effect even in the presence
of local spillovers.  It is shown that under certain assumptions regarding the
nature of the stable unit treatment value assumption, the estimated treatment 
effect will be significantly attenuated if this consideration is not made.
\Person then propose a number of ways to determine which control clusters should
be considered `close to' treatment clusters.  It is shown that for the morning
after pill, treatment spillover is a quantitatively important consideration,
and that in some groups, diffusion may exist for anywhere up to 30km from a
treatment location.

This study makes a number of contributions.  Foremost, it is one of the first%
---if not the first---study of the effects of emergency contraception using 
censal microdata on a national scale.  It is also the first large scale study 
of which the author is aware which addresses these questions in a country other 
than the United States.  This is of considerable importance given that Chile, 
the country under study here, does not offer legal abortion, and so the 
emergency contraceptive pill is the first legal mechanism for post-coital 
fertility control.
%\footnote{This is not to suggest that \emph{no} post-coital
%contraceptive options exist.  Given the lack of a legal solution, Misoprostol
%accessed on the black market is overwhelmingly to most common method used to
%interrupt undesired pregancies (Casas 2014).}

The results of this study add to the nascent literature on the emergency 
contraceptive pill.  Recent studies such as \citet{Grossetal2012, Durrance2013} 
which have been the first to address this question in the economic literature
have provided evidence to suggest that the effects of this technology may be 
minor.  Here we offer considerable evidence to the contrary, suggesting
that the expansion in the availability of emergency contraceptives may offer 
important effects in certain countries, with large impacts on pregnancy and 
abortion rates, especially among young women.

Finally, \person raise a number of novel points regarding the estimation of 
treatment effects in the presence of spillovers between treatment and control
clusters.  This is fundamentally different to, despite sharing some 
characteristics with, the literature on estimating treatment effects in the 
presence of spillovers between treated and non-treated individuals within 
treatment clusters.  \Person suggest that even in the presence of such 
spillovers, unbiased treatment effects can be estimated if these spillovers
occur in a predictable way, as is likely to be the case when distance to 
treatment varies in an exogenous manner. \Person propose a number of flexible
ways to consider estimating treatment effects in these circumstances.

In what remains of this paper \person first provide background regarding the
emergency contraceptive pill, and the reform under study in Chile in section
\ref{TEENscn:background}.  Section \ref{TEENscn:Data} discusses the censal data 
sets we will use to assess the effects of the reform, while \ref{TEENscn:ID} 
discusses identification and methodology.  In section \ref{TEENscn:results} 
\person present results on the contraceptive's effect on births, abortions and 
aggregate human capital endowments.  Finally, section \ref{TEENscn:conclusion} 
concludes.

%===============================================================================
\section{The History of the Emergency Contraceptive Pill}
\label{TEENscn:background}
The emergency contraceptive pill is a hormonal treatment which can be used 
within 5 days of an unprotected sexual relationship to reduce the probability
of conception.  There are a number of alternative types of emergency 
contraceptive pills, however principally these are composed of doses of the 
progestin levonogestrel, or a combined dose of estrogen and progestin. 
Typically these are taken as a single pill or two pills in a 12 hour period
\citep{vonHertzenetal2002}, however similar doses of hormones can be obtained 
by combining normal birth control pills \citep{Ellersonetal1998}.  

This form of contraception has been shown to be relatively effective at 
avoiding undesired pregnancy.  Estimates of around 75\%-85\% effectiveness 
based on typical usage are common, depending upon the method of emergency 
contraception used.\footnote{The WHO's \citet{WHO1998}, for example suggests 
that a levonogestrel routine reduces pregnancy rates by 85\%, with a 95\% 
confidence interval of 74-93\%.}  The success of these treatments is dependent
upon the delay between intercourse and taking the drug, so widespread---or at 
least quickly available---access is important in reducing undesired 
pregnancies.  While most effective when taken within 12 hours after 
intercourse, effectiveness can continue when taken within as much as 120 hours
\citep{vonHertzenetal2002}.

The emergency contraceptive pill is not an abortive agent, but rather is a 
`postcoital contraceptive' which acts to prevent ovulation 
\citep{Novikovaetal2007, Noeetal2011}. This contraceptive method has been of 
clinical interest
since at least the late 1960s \citep{Demers1971}, however access to these 
methods, either by prescription or over the counter, is still not universal.
The fact that emergency contraception is non-abortive has meant that it is
available in many countries in which abortion is absolutely prohibited, or
prohibited in all cases except where concerns for maternal survival exist.
Some countries have made the EC pill available as early as the mid-1980s 
\citep{UKFPA2006}, while many more countries have legalised this method of 
contraception during the last decade.


\subsection{The Morning After Pill in Chile}
\label{TEENsscn:Chile}
The introduction of the emergency contraceptive pill in Chile has followed 
a complicated path, with early legislation frequently blocked by conservative 
groups in office and in civil society.\footnote{The Chilean political 
framework is marked by a strong conservative axis, and a constitution which 
favours the maintenance of the status quo in economic and valoric policies.  
This has been the case since the return to democracy in 1990, with an 
alliance of right wing parties (and some members of the presiding left wing 
coalition) who have ``resisted more liberal changes in the poorly named value 
judgements''  (\citet{CasasBecerra2008}, p.6, author's translation.)}  While 
initial discussions and administrative inquiries took place in 2001, it was 
not until 2005 that significant advances in legislature were made. In 
December of this year the Chilean supreme Court determined that the Institute 
of Public Health---the pharmaceutical regularity body of Chile---was 
\emph{not} acting unconstitutionally by approving the provision of an 
emergency contraceptive drug on the pharmaceutical register.  However, this 
finding was quickly challenged by detractors, with cases presented before 
ordinary and Constitutional tribunals \citep{CasasBecerra2008}.

These tribunals were followed by a number of years' worth of legislations and
litigations, which resulted in sporadic availability of the morning
after pill, occasionaly freely available from state clinics or by purchase in
private pharmacies.  However, these were generally short-lived and emergency
contraception was not consistently stocked, with both political and economic 
ramifications for groups providing access to the pill.\footnote{For example,
the subsecretary of health was removed from cabinet due to his announcement
in 2005 that emergency contraception would be available to all women who sought
it.}  Details regarding this intervening process and laws passed by parliament
theoretically requiring the provision of emergency contraception are discussed 
more fully in appendix \ref{TEENscn:applegislate}.

The period of interest for this study follows a decision taken by the Chilean
Constitional Tribunal in 2008.  This finding, responding to a demand placed by
36 parlamentary deputies in 2006, made it expressly illegal for the centralised
health system to distribute the emergency contraceptive.  This requirement held
for all centres under direct administration of the national Ministry of Health,
but, fundamentally, provided all municipal-level centres and hospitals the 
freedom to distribute the pill.  Given that these centres are administered by 
the mayor of each municipality (or \emph{comuna}), the availability in each 
municipality was entirely under the control of the mayor \citep{Didesetal2011,
Didesetal2010,Didesetal2009}.\footnote{Of the 346 municipalities in Chile, 320
have their own health systems, while the remaining 26 depend entirely upon the
Ministry of Health.  These 320 municipalities make up 94\% of the population 
of Chile.  Municipal health centres make up the majority of health centres in 
Chile.  Of the 2501 registered health centres and hospitals, 2049 are under the
control of municipalities \citep{DEIS2013}.}  This resulted in a situation in
which around half of the municipalities in Chile distributed the morning after
pill freely, while the remaining half refused to distribute it, or to 
distribute it only in a very restrictive set of circumstances.  At the level
of the woman, her municpality's treatment status was essentially exogenously
determined, being based on the whim of the mayor or representative public 
health bodies in her area of residence.  This strange policy environment 
endured for approximately four years, until a law was passed mandating that the 
emergency after pill must be available to all women who request it.  This new 
law became operational in May of 2013.

The Chilean context is one in which emergency contraception may be expected to
have particularly important effects on pregnancy and maternal health.  Abortion
is entirely illegal in Chile, meaning that in the absence of emergency 
contraception, undesired or accidental pregnancies must either be taken to 
term, or a woman must risk undertaking a dangerous and illegal clandestine
abortion \citep{ShepardCasas2007}.  Figures on the frequency and method of 
clandestine abortion are unclear, however \citet{ShepardCasas2007} suggest that
the primary method is by taking the abortive drug misoprostol, which can be
legally prescribed for treatment of ulcers.  However, the cost of accessing 
this drug without prescription is high.  Dated (2007) figures suggest prices
of 38,000-50,000 Chilean pesos, or around one third of the minimum monthly wage
at this time.

%===============================================================================
\section{Data}
\label{TEENscn:Data}
The data of interest for this study comes from matched administrative data files 
recording all live births and fetal deaths in Chile.  \Person use birth outcomes 
for all women aged between 15 and 49 (inclusive) at some point during 2006-2011.  
This is crossed with data recording all women in Chile and their municipality of
residence, resulting in a record of each woman, and her pregnancy status in each 
period (live birth, fetal death or no pregnancy).  Along with a woman's birth 
status, we observe her baby's birth weight and gestational length in the case 
that a birth or fetal death was recorded.\footnote{Since 1999, each baby born in 
Chile (in public and private hospitals and in private homes with midwives) is 
weighed and measured, and their gestational time is recorded.  This data is 
collated by the Ministry of Health, and is available as a public birth weight 
census.  As well as the baby's characteristics, the mother's education, age, 
labour market status and municipality of residence is collected.  Similar 
details are collected in the case that a woman enters hospital and suffers a 
miscarriage.}

This results in a total sample of 1,391,565 births and 11,387 fetal deaths.
The number of births per year in Chile has remained relatively stable over the 
last decade.  Figure \ref{TEENfig:BirthDeath} displays total births, along
with total fetal deaths during the period under study (a similar plot for
births per women is available in figure \ref{TEENfig:Pregtime}).  Total births 
vary between around 220,000-250,000 per year, while total fetal deaths recorded
in the Ministry of Health data (all fetal deaths in any hospitals or clinics 
in Chile), vary between 1700 and 2100.

Our measure for the pill is a binary variable which records whether the 
emergency contraceptive was freely available to a woman upon request at her 
municipal health centre in the year before her birth outcome is observed.  We
consult two sources to collect data on pill availability.  First, in each of 
2009, 2010 and 2011 an independent survey was conducted, asking health care 
workers from each municipality whether they were able to prescribe the morning
after pill \citep{Didesetal2009,Didesetal2010,Didesetal2011}. This should 
directly reflect the decision by each mayor regarding whether his or her
municipality could prescribe the pill after the 2008 Constitutional ruling.  
In each case, survey respondents were also asked to list the circumstances in 
which they could prescribe the pill.  All municipalities which reported that
they could prescribe the pill freely to women were recorded as treated, while 
all others were recorded as untreated.\footnote{A small number of 
municipalities reported that they could prescribe the emergency contraceptive, 
however that this was only following cases of rape.  These municipalities were 
classed as \emph{un}treated given the lack of widespread availability.}  
Secondly, the Ministry of Health has made available administrative data on 
all pill requests and dispursements at municipality clinics and hospitals.  This
allows us to determine the veracity of the survey data discussed above, 
while also providing concrete numbers regarding the use of the emergency 
contraceptive pill following the reform of interest.  However, \person do not
use pill disbursements as the main measure of treatment.  \Person focus on
reported availability, given that disbursements are endogenous, and jointly
determined by demand as well as supply.

In total, 224 of Chile's 346 municipalities report being able to prescribe the 
pill in at least one year after the 2008 Tribunal result (see table 
\ref{TEENtab:SumStats}). Figure \ref{TEENfig:Pilltime} displays the quantity
of municipalities reporting pill disbursements over time.  Here, the number
of prescribers increases over time in line with greater awareness of the legality 
of distributing the emergency contraceptive pill.  While less than half of all
municipalities report pill availability in 2009, this has increased to around
two thirds by 2011.  Official records of pill prescriptions suggest reasonably
large fluctuations over time.  While nearly 8000 women were reported as 
requesting the pill in 2009, this fell to slightly under 4000 the following
year.  Recent figures suggest that this number has been stable at around 6000-%
7000 requests in 2010-2013 (the most recent two years have been omitted from
this study, and from graphical output, given that official birth records for
2012 and 2013 are not yet finalised).  Figure \ref{TEENfig:PillGeo} displays
the geographic variation of pill availabilitly.  This suggests that the pill
is available in all parts of the country. With the exception of the large and 
very sparsely populated southern region of the country (the 
10\textsuperscript{th} region) which has no municipal health centres, no 
obvious spatial patterns exist.

\Person examine fetal deaths as a manner to proxy illegal abortion.  While it 
is certainly not the case that all (or even the majority) of fetal deaths 
observed in administrative data are results of abortive drugs, there is some
evidence that these are the result of abortion in some cases, although they are 
recorded in a number of different ways in official figures to avoid criminal 
charges against women \citep{ShepardCasas2007}.  To avoid concerns that 
reductions in fetal deaths may be simply due to greater investments in public
health, \person examine a number of subgroups of interest.  Firstly \person focus 
on deaths occurring between 1-20 weeks of gestation, as this is the period in 
which nearly all abortions are conducted.  Secondly, \person remove deaths which,
based on their ICD code,\footnote{The ICD refers to the International 
Classification of Disease, and refers to a set of standardised codes by which
deaths can be classed.  All deaths on the birth register report this code,
(the ICD-10).} are clearly not related to abortion, such as those due to 
congenital malformations, deformations and chromosomal abnormalities.  By
using this methodology, a clear validity check exists by comparing reductions
in fetal deaths during 1-20 weeks (which may represent abortions and should
respond to the morning after pill), to those occurring from week 21 and above,
which should be largely or entirely unaffected by emergency contraceptive
availability.

Full summary statistics are provided in Table \ref{TEENtab:SumStats}.  These
statistics are subdivided by whether or not the municipality has the pill in
a given year.  We see that there are some differences between pill and non-pill
municipalities, such as higher education and health spending in pill 
municipalities.  However, this is largely due to the fact that all years in which
the pill was observed occur after 2008 while non-pill status is observed over
the entire time period under study.  Surprisingly, we see that municipalities in
both groups are approximately balanced in terms of the `conservativeness' of the
party of the mayor, however we do see that female mayors are more likely to be
associated with pill municipalities.  Later in this study \person describe 
pre-treatment trends in pill and non-pill municipalities.

%===============================================================================
\section{Methodology}
\label{TEENscn:ID}
\Person take advantage of the quasi-random nature of the expansion of the 
availability of the morning after pill to women in Chile.  A woman $i$, living
in municipality $j$ in time $t$ is considered as treated if public health 
centres report that the pill is available upon request.  A woman's child 
bearing status $birth_{ijt}$ is regressed on the availability of the pill 
($pill_{jt}$) in the preceding year:
\begin{equation}
 \label{TEENeqn:pill}
birth_{ijt} = \alpha + \delta\cdot \mathbb{I}\{Pill_{jt-1}\} + \phi_t + \eta_j + 
\eta_j\cdot t + X_{jt-1}\gamma + \varepsilon_{ijt}.
\end{equation}
In (\ref{TEENeqn:pill}), full municipality and year fixed effects are included,
and municipality-specific time trends are allowed.  Standard errors are 
estimated which allow for auto-correlation by municipality.  The identifying 
variation in availability of the pill is by municipality and year.  Prior to 
the legal reform the pill was unavailable to all women, while posterior to the
reform the pill was available to those women living in municipalities where the
mayor did not restrict access.  This provides a flexible differences-in-%
differences (hereafter diff-in-diff) framework, and allows us to causally 
estimate the effect of the morning after pill if we believe that typical 
diff-in-diff assumptions hold.  Namely, we require that unobserved components 
$\varepsilon_{ijt}$ in the above specification evolve similarly over time in 
the treated and untreated municipalities.

Given the geographically disperse, and, as discussed in previous sections,
plausibly exogenous nature of the arrival of the morning after pill, we may be
willing to accept that this assumption is valid.  However, to minimise the 
potential that spurrious events confound the arrival of the pill, we 
progressively include higher order time trends and other factors that vary 
non-linearly over time across municipalities.  These factors, $X_{jt-1}$, 
include controls for political and social outcomes such as the mayor's party
(and implicitly the conservativeness of views), the degree of voter support for
the mayor, the mayor's gender, health and education inputs including staffing
and training investments, and time varying measures of female empowerment by
municipality.

We are interested in determining whether the morning after pill affects 
fertility at both the extensive and the intensive margins.  We thus measure
pregnancy in a number of ways: firstly, at the intensive margin by examining
whether a woman gives birth at any parity level, and secondly, only at the
extensive margin by examining whether she moves from 0 to 1 births. Similarly, 
we are interested in determining the degree of heterogeneity of access by age 
groups, and look at teenagers (15-19 year olds), 20-34 year olds, and 35-49 
year olds. 

Similar estimations are run replacing $birth_{ijt}$ with $fetal\ death_{ijt}$,
which---for certain subsets---we believe proxy illegal abortion (as discussed
in section \ref{TEENscn:Data}).  After assessing the pill's impact on pregnancy, 
and abortion \person estimate reduced form effects of the pill's arrival on 
various measures of mother and child outcomes.  These include maternal education,
employment status and marital status, and child birthweight and gestational
length.  While \person don't believe that these regressions are demonstrating
causality in the case of mother's outcomes, these regressions are a useful test
to determine whether certain groups are more likely to access the pill leading to
aggregate compositional change in the cohorts of women who give birth.

\subsection{Identifying Spillovers Between Municipalities}
\label{TEENsscn:spilloverID}
Our diff-in-diff estimates in the previous section potentially underestimate the 
true effect of the morning after pill.  Principally, we may be concerned that
there exist spillovers between treatment and control clusters due to the porous
nature of municipal boundaries. Given that a woman can access municipal health
centres in neighbouring comunas, if she is denied access to the pill in her
comuna she may travel to obtain it elsewhere, or otherwise rely on the close
geographic distance between her municipality and a treatment municipality to 
gain access to the morning after pill\footnote{This may be the case for example
if women rely on friends or contacts in neighbouring municipalities to gain 
access.}.  This motivates the following specification:
\begin{equation}
 \label{TEENeqn:spillover}
y_{ijct} = \alpha + \delta\cdot \mathbb{I}\{Pill_{jt-1}\} + 
\sum_{c=0}^C\zeta_c\cdot close_{cdjt-1} + \phi_t + \eta_j + \eta_j\cdot t +
X_{jt-1}\gamma + \varepsilon_{ijct}.
\end{equation}
where
\[
 close_{cdjt} =
  \begin{cases}
   1 & \text{if } dist_{jt} > c \wedge dist_{jt}\leq c+d   \\
   0 & \text{if } dist_{jt} \leq c \vee  dist_{jt}>c+d.
  \end{cases}
\]
where $dist_{jt}$ is the distance (in km) to the nearest treatment municipality 
(ie a municipality which reports prescribing the pill).  By definition, this takes
0 for all treatment municipalities.

This specification is identical to that in (\ref{TEENeqn:pill}), however here we 
include a number of $close$ controls (indexed by $c$).  These $close$ variables are 
designed to capture spillovers between the pill treatment areas and surrounding 
areas which may also be affected by this treatment status, but which were not 
themselves treated.  As defined above, at most one of these $close$ dummies can
be switched on for a given (non-treated) municipality.  By judiciously selecting 
an appropriate series of $close$ dummy variables, the true effect of the morning 
after pill can be recovered even in the case that spillovers occur between 
certain treatment and control clusters.\footnote{Thus, these controls are 
determined by $c$, the minimum distance to a treatment cluster, and $c+d$, 
the maximum distance to the treatment cluster. For example, $close_{0,15,jt}$ 
will take the value of 1 for any municipality which does not itself prescribe 
the EC pill, but is within (0,15] km of a treatment municipality.  Similarly 
$close_{15,30,jt}\Rightarrow (15,30]$.}

To see this, define $y_{1ijt}$ as the potential outcome for a woman in the 
presence of treatment.  Likewise, $y_{0ijt}$ is the potential outcome for a 
woman in the absence of treatment.  As is well known in the treatment effects
literature\footnote{See for example \citet{CardKruger1994}.}, difference-in-%
differences will allow us to estimate the causal effect of treatment if we are 
willing to make a common trends assumption about treament and control 
municipalities.  Implicitly, this common trends assumption nests an assumption
about spillovers between treatment and control muncipalities: that no such 
spillovers may exist, as these will affect the pre-existing trend in the control 
state.\footnote{Fortunately for the naive estimates of treatment effects in this 
case, any estimates will be attenuated rather than overstated, given that the 
mixture of treated units with the control group will cause outcomes in control 
group to look more like those in the treatment group.}  This is analogous to the 
Stable Unit Treatment Value Assumption (SUTVA) of the Rubin Causal Model.

Now, rather than dealing with the two potential outcomes statuses above, we
define new outcomes.  First, we define $y_{0ijtc}$, the potential outcome for 
woman $i$ in untreated municipality $j$ in time $t$ and with close status $c$.  
For simplicity, in what follows we will consider $c$ as binary, indicating whether
$j$ is `close' or `not close' to a treatment municipality, although the results
for a categorical variable follow logically.  Similarly, we define $y_{1ijtc}$
as the potential outcome in a treated municipality with close status 
$c$.\footnote{However, in the case of treated municipalities $c$ will always
take the value of 0 given that these municipalities are themselves treated rather
than simply close to a treated municipality.}

As is typical in a double-differences framework, an additive structure for 
$y_{ijtc}$ is assumed which consists of a municipality effect, a time effect,
an indicator for treatment ($D_{jt}$), and in our case, an indicator for being 
`close' to treatment ($close_{jtc}$):
\begin{equation}
 \label{TEENeqn:DDa1}
 y_{ijtc} = \eta_j + \phi_t + \delta D_{jt} + \zeta close_{jtc} + 
\varepsilon_{ijtc}
\end{equation}
Now, if we consider the single differences which make up a double-differences 
estimate, we have, for the treatment group:
\begin{eqnarray}
\label{TEENeqn:diftreat}
 E[y_{ijtc}|j=Pill,t=2,c=1]- E[y_{ijtc}|j=Pill,t=1,c=1] = \phi_2-\phi_1+\delta.
\end{eqnarray}
This is the traditional single difference which forms one half of a typical
double-differences estimator.  However, in the case of the control group, the
single difference is no longer simple.  It will now be made up of two components:
the difference over time in control municipalities who are `close' to treatment 
municipalities:
\renewcommand{\theequation}{\arabic{equation}a}
\begin{equation}
\label{TEENeqn:difclose}
 E[y_{ijtc}|j=No Pill,t=2,c=1]- E[y_{ijtc}|j=No Pill,t=1,c=1] = \phi_2-\phi_1+\zeta
\end{equation}
\addtocounter{equation}{-1}
\renewcommand{\theequation}{\arabic{equation}b}
and the difference over time for `non-close' control municipalities:
\begin{equation}
\label{TEENeqn:difnonclose}
 E[y_{ijtc}|j=No Pill,t=2,c=0]- E[y_{ijtc}|j=No Pill,t=1,c=0] = \phi_2-\phi_1.
\end{equation}
\renewcommand{\theequation}{\arabic{equation}}
If we were to naively combine close and non-close control municipalities to make
one large control group, we would have that our second difference consists of the
weighted sum of (\ref{TEENeqn:difclose}) and (\ref{TEENeqn:difnonclose}).  Were
we then to combine the first difference (\ref{TEENeqn:diftreat}) and the second
difference (the weighted average of \ref{TEENeqn:difclose} and 
\ref{TEENeqn:difnonclose}) to form our double-differences estimator, this would 
give:
\begin{equation}
\begin{split}
 \{E[y_{ijtc}|j=Pill,t=2,c=1]- E[y_{ijtc}|j=Pill,t=1,c=1]\}-\\
 \bigg(\frac{N_c}{N_c+N_{nc}}\{E[y_{ijtc}|j=No Pill,t=2,c=1]- E[y_{ijtc}|j=No Pill,t=1,c=1]\}+\\
 \frac{N_{nc}}{N_c+N_{nc}}\{E[y_{ijtc}|j=No Pill,t=2,c=0]- E[y_{ijtc}|j=No Pill,t=1,c=0]\}\bigg) = \\
 \delta - \frac{N_{c}}{N_{c}+N_{nc}}\zeta.
 \end{split}
\end{equation}
Here we clearly see that our naive estimator fails to recover the true parameter
of interest $\delta$.\footnote{This estimator includes as a limiting case the
typical diff-in-diff estimator, as in this case we assume that no spillovers are 
present and SUTVA holds, meaning that $\zeta=0$.  Similarly, if no municipalities 
were close enough to experience spillovers, we would have that $N_c=0$, and once 
again $\delta$ would be recovered.}  Generally, we would suspect that if 
spillovers exist, then they are likely to be of the same direction as the effect 
of treatment, meaning that $\delta$ and $\zeta$ will have the same sign.  If this 
is the case, the inclusion of $\frac{N_c}{N_{nc}+N_c}\zeta$ in the naive estimate 
attenuates the treatment effect.

However, in the above discussion, we are concerned that SUTVA has been violated 
in a specific manner.  We are concerned that the treatment status of women in 
treatment municipalities spillover to those of control muncipalities `close' to 
these treatment municipalities.  Conversely then, we assume that women in 
control muncipalities `far enough away' from those in treatment municipalities 
are not affected by their treatment status, and so SUTVA still holds in these 
cases.  Specifically, imagine that our double-difference estimator is now only 
based upon those control municipalities which are not classed as belonging to 
$close$.  In this case:
\begin{equation}
 \label{TEENeqn:DDaextend}
\begin{split}
 \{E[y_{ijtc}|j=Pill,t=2,c=1]- E[y_{ijtc}|j=Pill,t=1,c=1]\}-\\
 \{E[y_{ijtc}|j=No Pill,t=2,c=0]- E[y_{ijtc}|j=No Pill,t=1,c=0]\} = \delta,
 \end{split}
\end{equation}
and using the sample analogue of these population parameters we are able to
correctly recover the true effect of interest.  Discussion of how to precisely
determine which municipalities are and are not part of the $close$ group is 
delayed until section \ref{TEENsscn:spillover}.

Estimating (\ref{TEENeqn:spillover}) provides a flexible regression-based 
framework for (\ref{TEENeqn:DDaextend}).  Both $Pill_{jt}$ and $close_{cjdt}$
switch on only in those municipalities who are affected by the pill (either
directly or via spillover) in the date after the morning after pill has become
available.  In this case both $\delta$ and $\zeta$ (from \ref{TEENeqn:spillover}) 
identify the effect of living in a pill or close-to-pill municipality by 
comparing them to treatment municipalities which are sufficiently far from the 
morning after pill that we can plausibly assume SUTVA.  In the case of the 
coefficient on pill this is simply estimating our effect of interest
(\ref{TEENeqn:DDaextend}), while the coefficient on $close$ identifies the 
marginal effect of being close to the pill.\footnote{In the framework above, 
this can be viewed as: 
\begin{equation}
\nonumber
\begin{split}
\{E[y_{ijtc}|j=No Pill,t=2,c=1]- E[y_{ijtc}|j=No Pill,t=1,c=1]\}- \\
\{E[y_{ijtc}|j=No Pill,t=2,c=0]- E[y_{ijtc}|j=No Pill,t=1,c=0]\} = \zeta.
\end{split}
\end{equation}}
Given that we are assuming geographic dependence in these estimates, we use 
\citeauthor{Conley1999}'s (1999) spatial standard errors.  This involves defining 
a reasonably flexible covariance matrix which inversely weights observations to 
allow for dependence across individuals based on distance.

%===============================================================================
\section{Results}
\label{TEENscn:results}
\subsection{The Effect of Emergency Contraception on Births}
\label{TEENsscn:rbirths}
Table \ref{TEENtab:PillPreg} provides estimates for specification 
(\ref{TEENeqn:pill}).  This has been estimated using a logistic regression, and
all coefficients are cast as log odds.  Here, \person examine two fertility 
outcomes: the probability that a woman gives birth to any child (columns 1-4), 
and the probability that a woman gives birth to her first child (columns 5-8). 
The latter outcome captures just the effects of the emergency contraceptive pill
at the extensive margin, while the prior outcome captures both extensive (first
birth), and intensive (more births) effects.

In each case we estimate first the simple diff-in-diff specification without 
time-varying controls, and then gradually add time varying controls which may 
confound results of the original specification.  Initial results suggest that the 
effect on pregnancies may be large, particularly so for teenagers.  Point 
estimates on ``All Births'' for the 15-19 year old group suggest that the pill is 
associated with a highly significant 6.2\% reduction in pregnancy 
(1-$\exp(-0.064)$), or a 4.0\% reduction when including potentially confounding 
time-varying controls.  The coefficients on these time varying controls are omitted 
from table \ref{TEENtab:PillPreg} for the sake of clarity; however a full output 
for column (4) of each panel is provided in appendix table \ref{TEENtabPregFull}.

When we examine the effect only on first births, we see a somewhat smaller, but
still important 3.5\% reduction in births (or an imprecisely estimated 2.1\% 
reduction when including controls for condom availability).  This ``First Births''
column must necessarily be less than or equal to the effect of the morning after 
pill on all births, given that all births include first births, along with higher 
order births.  The difference between the results in these two columns allows
for a rough examination of the magnitude of extensive versus intensive effects
on fertility.  Were the entire effect of the pill working at the extensive 
(first birth) margin, we would expect that the coefficients for ``First Births''
should equal those on ``All Births''.  As is, for the teenage group, we see
that the coefficient on first births is 51\% of that on all births, suggesting
that while the extensive margin is important, the morning after pill also has
important effects at the intensive margin.

The effects on older age groups are more moderate than the effect on teenagers,
consistent with the fact that a greater proportion of teenage births are
undesired.  However, for 20-34 year olds we still see that access to the 
emergency contraceptive reduces pregnancy, by 3.0\% for all births, and 2.1\% for 
first births.  In contrast to younger women, there appears to be no effect of the 
morning after pill on women aged 35 and above.  All estimates for the 35-49 
year old group are not significantly different to zero.

\subsection{The Effect of Emergency Contraception on Abortions}
\label{TEENsscn:rabortion}
In table \ref{TEENtab:PillDeath}, difference-in-difference estimates of the 
effect of the emergency contraceptive pill on fetal deaths are presented.  Once
again these are estimated using a logit model. In this case the denominator (or
0 in the outcome variable) is assigned to each live birth, while a fetal death
is assigned a 1.  All effects are thus interpreted as fetal deaths per live 
births.  As discussed in section \ref{TEENscn:Data}, by using certain subsets of 
fetal deaths we aim to proxy for illegal abortion.  We expect that if the 
emergency contraceptive pill affects abortion, this should turn up in fetal 
deaths occurring from 0-20 weeks, however should not turn up in deaths occuring
later in the gestational period, given that abortions rarely take place beyond 
the 20\textsuperscript{th} week.

Column (1) of table \ref{TEENtab:PillDeath} presents the effect of the pill on
\emph{all} fetal deaths.  We are, however, most intersted in columns (2) and (3),
which present results for early (0-20 weeks), and late (21-39 weeks) 
respectively.  In these columns we have removed from the sample any fetal deaths
which have been classified in ICD class Q (a minority of fetal deaths), as these 
represent causes such as congenital defects, which are very unlikely to proxy
abortion.

For the 15-19 year old group, significant evidence is found to suggest that 
the morning after pill may reduce the prevalence of (illegal) abortion.  Some
effect is found when examining the effect on all fetal deaths, however when 
this is examined by subgroups, the effect is entirely driven by early gestation
deaths.  The size of the coefficient is empirically very important: it 
suggests a reduction in early gestation deaths by 55\%, which \person interpret
as strong evidence in favour of reductions of illegal abortion.  When compared 
to the null effect on late-term deaths, this seems to provide more support to
this claim.

A similar pattern is observed for the 20-34 year old group of women, however
effects are smaller and somewhat less significant.  While no significant effect
is found when examining all births, there is evidence (at the 10\% significance
level), that the arrival of emergency contraceptive reduces early gestation
deaths by 17\%.  Once again, no significant effect is found in late gestation
fetal death.

The group of women aged 35 years and above is somewhat less clear, and, while the 
effect sizes of the coefficients follows the pattern outlined above, the
significance on late gestation fetal deaths is somewhat surprising.  Given that
fetal death is much more common as maternal age increases, it is perhaps 
unsurprising that we find some effect for this group.  One possible explanation
for this finding is that the morning after pill allows less healthy women to 
select out of child bearing, although given the lack of covariates recording 
mother's health at the time of childbirth, this cannot be explored fully.

\subsection{Municipality Spillover and Imperfect `Compliance'}
\label{TEENsscn:spillover}
We augment our naive estimates from sections \ref{TEENsscn:rbirths} and
\ref{TEENsscn:rabortion} to account for between-cluster spillovers in table
\ref{TEENtab:Spillover}.  These results are estimated according to equation
(\ref{TEENeqn:spillover}) using a logit regression.  The discussion is section
\ref{TEENsscn:spilloverID} proposes including controls for areas which are `close 
enough' to treatment municipalities that they are likely to be affected by
spillovers.  However, discussion regarding how to determine the threshold has
been putoff until this point. In this section \person examine two related ways 
which this can be done.  Both of these ways rely on the data and specific context 
of the treatment in question to determine the range over which municipalities 
should be considered as close.

The first method involves a series of consecutive regressions and tests on the
coefficient $\widehat\delta$.  First, a regression is run including no close
controls and $\hat\delta^0$ is observed along with its standard errors (where 
superscript 0 refers to the number of close controls included).  Then, a single
close control is included for municipalities within $d$ km of the treatment
municipality (where $d$ can be some small number).  From this regression, we
observe $\hat\delta^1$, and test for the equality of $\delta^0$  and $\delta^1$ 
using a $t$-test.  If this test is rejected, we add another close control, 
this time indicating muncipalities located within $d$ and $2d$ km of the 
treatment municipality.  Again we run a $t$-test for the equality of $\delta^1$ 
and $\delta^2$.  This iterative process is continued until the point that we 
cannot reject the test that $\delta^{C-1}=\delta^C$.  At this point we accept 
that we have saturated our model with sufficient `close' controls to recover 
a consistent estimate of $\delta$, and assume:\footnote{One situation in which 
this will not provide a consistent estimate of $\delta$ is the case in which 
spillovers converge, but do not converge to zero.  If for example beyond a 
certain distance $C$ the effect of spillovers reach some fixed constant, then 
the null that $delta^{C-1}=\delta^C$ will not be rejected, even though the 
marginal $\zeta$ term is not equal to zero.}
\begin{equation}
\bigg|\delta-\frac{c}{c+nc}\zeta\bigg|\simeq|\hat\delta^0|<|\hat\delta^1|<\cdots<
|\hat\delta^{C-1}|=|\hat\delta^C|\simeq|\delta|.
\end{equation}

The second method follows a similar iterative process, but rather than testing
each $\delta^c$ against its predecesor, we run a $t$-test with the null:
$\zeta=0$.  The logic in this case is that rather than assuming that we
have a consistent estimate of $\delta$ once this coefficient is stable, we
assume that we have included enough `close' municipalities once spillover effects
are no longer found in the marginal municipality.

In order for these methodologies to uncover a consistent estimate of $\delta$,
all we require is that there actually are at least \emph{some} control 
municipalities far enough away from treatment municipalities in which no 
spillover effects are felt.  As described in (\ref{TEENeqn:DDaextend}), these 
`non-close' municipalities act as the control group for our diff-in-diff estimator, 
so if no non-close municipalities exist, no consistent estimator can be formed.  
Given the relatively large distance between some non-treated municipalities and 
their nearest treated counterpart in the Chilean context, this seems unlikely in 
this case, although we cannot reject this formally.

Panel A of table \ref{TEENtab:Spillover} estimates using this methodology.  In
this case, using either of the above methods results in an identical number of 
close controls.  For both 15-19 year olds and 20-34 year olds, it appears that
living within 30 km of a treatment municipality results in a spillover effect,
while for the case of 35-49 year olds no spillover is observed.  Now, based on
these updated estimates it appears that the true effect of the morning after 
pill may be significantly higher than that estimated in section 
\ref{TEENsscn:rbirths}.  Compared to the 4.1\% reduction in teenage births 
estimated from specification (\ref{TEENeqn:pill}), here we estimate a 6.9\% 
reduction for women living in treated municipalities ($1-\exp(-0.071)$), with
sizeable effects also found for those living in close, non-treated 
municipalities.  Similar patterns are observed for the 20-39 year old group,
however in this case estimates are increased in magnitude from 3.0\% to 4.2\%.
Figures \ref{TEENfig:Dist1519} and \ref{TEENfig:Dist2034} provide graphical
support of this methodology.  Here, estimates are calculated based on a wide
range of $close$ controls, with a step size $d$  of 2.5 km.  In each case,
the estimated $\hat\delta$ appears to converge when controlling for spillovers
of up to 30km.

In Panel B, similar tests are run for fetal deaths.  In this case it appears 
that spillovers act only over a shorter range or not at all, however this may 
owe partially to the fact that fetal death is a relatively uncommon event, so 
estimates are more imprecise.  Once again however, the inclusion of close 
municipality controls act to increase the magnitude of treatment effects 
estimated.  For 15-19 year olds, point estimates move from a 51.7\% reduction 
in early-gestation fetal deaths, to a 56.6\% reduction, and for 20-39 year olds
these point estimates move from 13.0\% to 16.1\%, however it is worth noting 
that these changes are not statistically significant.

\subsection{Emergency Contraception and Aggregate Human Capital at Birth}
Table \ref{TEENtab:PillAgg} examines the effect of emergency contraception on
aggregate human capital indicators of pregnant women and newborn babies.  While
it is not suggested that the morning after pill itself will affect a woman's
human capital attainment over such a short time frame, if certain subgroups of 
the population are more likely to access the contraceptive, it is likely that 
aggregate compositional changes will be seen in both maternal and child human 
capital outcomes.  There is considerable evidence to this effect when considering 
access to the oral contraceptive pill \citep{Baileyetal2012,OltmansHungerman2012,
ChiapporiOreffice2008}, and the arrival of legal abortion \citep{Whitaker2011,
Ananatetal2009}.

\Person examine three outcome variables for mothers: years of education, 
employment status, and a binary variable for marriage, and three outcome variables
for newborns: weight at birth, weeks of gestation, and length (in cm) at 
birth.\footnote{These outcomes, particularly birthweight, have been shown to
improve outcomes including educational attainment and income throughout life 
\citep{BehrmanRosenzweig2004}}  Each model is estimated as outlined in 
(\ref{TEENeqn:pill}) using OLS.  Surprisingly, we find that the morning after
pill has had no, or very little, effect on aggregate human capital indicators.
This is the case among mothers, and consequently among newborn babies.

Panel A of table \ref{TEENtab:PillAgg} presents estimates by age group.  For
both teenagers and 20-34 year olds, no effect is seen on any of the variables
examined. In general, these results seem to suggest that access to the morning 
after pill is transversal, and is not centred on highly educated or employed 
women.  Moving to the 35-49 year old group, some evidence exists to suggest
that the aggregate education of women giving birth is slightly higher where
the morning after pill is available.  This would be consistent with less
educated (and perhaps less healthy) women selecting out of child bearing in
this age group, which is consistent with the results found for this age group
in table \ref{TEENtab:PillDeath}.

Panel B provides estimates for all children born over the period under study.
Once again, very little evidence is found to suggest that the emergency 
contraceptive pill has created large-scale compositional effects to birth
cohorts.  Given the lack of effect found in mothers, it is not surprising
that similar results are found in babies.  In each case, no effect is observed
on birthweight, gestational period, or length at birth (with the exception
of a very small reduction in gestational length for babies born to 20-34 year
olds).  Each of the reported significance levels is based on a two-tailed 
$t$-test.  If \person were to correct for multiple hypothesis testing using a
Bonferroni correction, finding a significant result would be even less likely.

\subsection{Placebo Tests}
\label{TEENsscn:placebo}
Robustness of the main estimates to the addition of time-varying controls and
municipal-specific time trends provides some confidence in the results, however 
does not directly examine the differential trends assumption underlying 
diff-in-diff estimation.  In order to examine this assumption more closely, 
\person run a number of placebo tests.  These placebo tests allow us to examine
whether the results are driven by pre-existing differences or trends in treatment
and control municipalities. 

\Person thus run analogous tests to (\ref{TEENeqn:pill}) and 
(\ref{TEENeqn:spillover}), however rather than looking at births following the
introduction of the pill, \person examine births \emph{preceding} the introduction
of the pill.  The logic underlying these tests is that if is the arrival of the EC 
pill which reduces undesired births, then there should be no difference between 
trends in births in pill and non-pill municipalities in years preceding the reform.  
If however, the effects are due to general differences in trends in non-pill and 
pill municipalities, we may expect that an effect would be seen even in the 
absence of the EC pill.  We this run the following series of tests:
\begin{equation}
 \label{TEENeqn:placebo}
birth_{ijt-l} = \alpha + \delta\cdot \mathbb{I}\{Pill_{jt}\} + \phi_t + \eta_j + 
\eta_j\cdot t + \varepsilon_{ijt},
\end{equation}
where $l$ refers to a series of lags $l\in 3,4,5$ years.  \Person choose lags of
at least 3 years so that all births observed will occur entirely before the 
arrival of the EC pill in 2008.

These results are presented in table \ref{TEENtab:Placebo}, both for 
specification (\ref{TEENeqn:placebo}) and an analogous specification where 
placebo close municipalities are defined.  These placebo tests support the 
diff-in-diff specification estimated.  In all but 3 of 30 coefficients, small
and statistically insignificant results are observed on placebo treatments.  In
3 of 30 cases, significant effects are found, although these are always on 
placebo `close' treatments, and not on the main treatment itself.  Generally
this is quite strong evidence in favour of an absence of pre-treatment 
differential trends, as at 10\% significance levels, it is expected that 
approximately 3 in 30 coefficients should be falsely accepted (ie a type I error
should occur).

Along with these formal placebo tests, we can examine trends by eye based on
full data on all births occurring in Chile in the past decade and a half. \Person 
present graphical results as appendix figures \ref{TEENfig:Trend1519} and 
\ref{TEENfig:Trend2034}.  These figures suggest that indeed, the sharp 
discontinuity in births occurs precisely following the arrival of the EC pill 
to Chile: further evidence in favour of these results owing to the morning after 
pill, rather than to alternative actions taken in pill and non-pill 
municipalities.

%===============================================================================
\section{Conclusions}
\label{TEENscn:conclusion}
This study provides the first censal estimates of effect of the emergency
contraceptive pill.  In contrast to existing studies based on data from the 
United States, this study focuses on a reform in Chile, a country with high rates
of teenage pregnancy and undesired childbearing, and where abortion is entirely 
outlawed.  The lack of abortion or other post-coital birth control technologies 
means that the arrival of the emergency contraceptive pill heralded the first 
opportunity for women to control fertility in cases where alternative forms of 
birth control were not used or failed during intercourse.

By taking advantage of a legal finding which left decisions regarding the 
availability of the morning after pill in the hands of the mayor of each of
Chile's 346 municipalities, \person estimate the effect of this technology on 
fertility, abortion and aggregate human capital outcomes. In contrast to the 
literature currently available, \person find the emergency contraception has 
large and significant effect on births and early gestation fetal deaths.  For 
teenagers, this effect is estimated to be a reduction of 6.9\% and a remarkable 
55\% in births and early-gestation fetal deaths respectively, while for 20-34 
year old women these figures are a smaller, but still significant 4.1\% and 
16.0\%.  It is argued that these early-gestation deaths proxy for illegal 
abortion, and comparisons with late term deaths add support to this claim.

Given the permeable nature of municipal boundaries within a country, \person 
examine the possibility that the arrival of the pill to a given municipality
is not restricted only to women who live within its boundaries.  Results
suggest that this may be the case, and that treatment spillovers may endure 
for as much as 30km. \Person propose an identification strategy which flexibly 
allows for such spillover effects to be accounted for, while simulatenously 
recovering consistent estimate of the effect of the treatment in the presence 
of contaminated (local) control groups.

All told, this paper provides considerable evidence that emergency contraception 
may play an important role in a woman's contraceptive behaviour.  This finding
is of particular importance to the country under study given that only recently
has law been implemented making the morning after pill available to all.  This
also suggests that despite evidence to the contrary in the United States, the 
emergency contraceptive pill may be an important interim technology in the many 
countries which currently do not allow alternative forms of post-coital 
contraception.

\end{spacing}
\begin{spacing}{1.4}

\newpage
\section*{Figures}
\input{\pillloc/figures/Figures.tex}
\clearpage

\section*{Tables}
\input{\pillloc/tables/Tables.tex}
\newpage

\biblioinc

\end{spacing}
\begin{spacing}{1.4}
\newpage

\appendix
\section*{Appendices}
\section{The Chilean Legislative Environment and the Adoption of Emergency
Contraception}
\label{TEENscn:applegislate}
Discussions surrounding the introduction of emergency contraception in Chile
have taken place since at least 1996, when the Chilean Institute of 
Reproductive Medicine (ICMER for its initials in Spanish) proposed the use of
this method to avoid undesired pregnancies in a country where abortion was
entirely outlawed \citep{Dides2009}.  However, the first legislative attention
given to this matter occurred when the aforementioned (see section 
\ref{TEENsscn:Chile}) Institute of Public Health emitted a resolution allowing
for the production and sale of `Postinol', a drug containing levonogestrel by a
Chilean laboratory in 2001.  The Constitionality of this was quickly 
challenged, and the drug was prohibited by the Supreme Court.

The emergency contraceptive pill again entered legislative attention in 2004,
following the Ministry of Health's publication of a guide suggesting that 
emergency contraception be used following cases of rape.  Following this in 
2005, the Subsecretary of Health Dr.\ Antonio Infante announced that emergency
contraception would be freely available to \emph{all} women who requested it,
however the President of Chile and the Ministry of Health later declared that
this was not the case, leading to removal of the Subsecretary from office.

In November of 2005, the Supreme Court of Chile provided the first 
constitutional support for the emergency contraceptive pill, voting 5-0 to
reverse the decision taken in 2001, allowing emergency contraception to be
provided in the case that the mother's life was in danger.  Once again however,
this finding was challenged shortly thereafter.  The same non-governmental 
institution which had earlier raised a case against ICMER, now challenged the 
private commercial laboratory in charge of producing and distributing the drug.  
However, before this case could reach court, this laboratory voluntarily gave 
up their license to produce the drug, in a three line statement issued by the
General Director of the company on February 14, 2006 \citep{CasasBecerra2008}.

In the same year, a group of 36 parliamentary deputies from conservative 
parties raised a case with the Constitional Tribunal, claiming that the 
provision of the emergency contraceptive pill contravened the ``National Laws
for the Regulation of Fertility'', a set of rules issued by the Ministry of
Health.  This case was only resolved in 2008, with the Constitional Tribunal's
finding in favour of this group, hence making illegal any provision by 
hospitals or health centres controlled by the Ministry of Health (and hence
under the jurisdiction of the National Fertility Laws).  Fundamentally however,
this left the door open for Municipal health centres to distribute the pill
freely to women.  These Municipal Health Centres are run under the directive
of the elected mayor of each Municipality, leaving all remaining legislation 
regarding the distribution of the pill up to the 346 mayors in Chile.

In this study \Person study the period surrounding this 2008 legislation as the 
cutoff of interest.  However, even after this finding the emergency 
contraceptive pill has not been far from legislative action, with a number of
other cases raised.  These cases never entirely threatened the continuity of
supply of the morning after pill by municipalities, however did cause some
confusion for mayors and municipal health bodies in determining whether or not
they were legally allowed to prescribe the contraceptive.  These cases also
resulted in the passing of a number of laws and standards.  Most importantly,
they resulted in national Law 20.418 which ``creates standards for information,
guidance and regulatory services in fertility'' (author's translation), and 
the passing of a decree on March 3, 2013, which makes obligatory the provision
of the morning after pill to women of any age in any health centre in Chile.  
This became operative on May 28, 2013, meaning that---at least officially---%
there are no longer any restrictions in place in the country.

\section{Data Appendix}
With the exception of raw birth and fetal death data which requires that the
user agree to a number of privacy clauses, all raw and processed data used in
this paper is made available online at: 
\url{https://github.com/damiancclarke/morning-after-pill/tree/master/Data}.
Birth and fetal death data can be downloaded online at: 
\url{http://www.deis.cl/descargar-bases-de-datos/} and \person make available 
full processing scripts which convert this into the final dataset used here.  In 
the remainder of this appendix, \person provide further details regarding each 
data source used.

\subsection{Main Data on Births and Fetal Deaths}
Data on all births and deaths in Chile is publicly available for download at
\url{http://www.deis.cl/descargar-bases-de-datos/}.  This contains microdata
registers of every birth and fetal death occurring Chile between 1999 and 2011.
This is censal data, and is unlikely to miss any births given the importance
of registering every child born with authorities in order to receive a national
identity number used in all contact with public and private organisations 
including hospitals and schools.  The main analysis in this paper is based on
births and fetal deaths occurring between 2005 and 2011 (see table 
\ref{TEENtab:SumStats}), however in placebo tests earlier birth data is also
used.

\subsection{Population Data}
In order to link the number of births to the number of women of fertile age in
each municipality and time period, \person consult data from the National 
Institute of Statistics of Chile (INE).  This is made available at 
\url{http://www.ine.cl/canales/chile_estadistico/familias/demograficas_vitales.php}
and provides full demographics by age, municipality, and gender.

\subsection{Time-Varying Municipality and Region Controls}
Time-varying municipal controls such as education and health spending, and the
number of females working in public government is downloaded from the National
System of Municipal Information (SINIM).  This provides data as far back as
2005, and is freely available for download online at
\url{http://www.sinim.gov.cl/indicadores/busq_serie.php}.

Data on municipal elections, mayor's gender, party and vote share is accessed
from the Electoral Service of Chile (SERVEL).  This provides all electoral
results from municipal elections for the full time period of this study.  Raw
data is available online at 
\url{http://www.servel.cl/ss/site/mobile/padron_electoral_comunal_por_ano_informe_comunal_anual.html}
or processed as one line per municipality at the data page of the author's
website linked to above.

Finally, \person calculate data for alternative contraceptive use based on
a series of regionally representative surveys collected every 3 years beginning
in 1994.  The National Survey of Youth asks respondents whether they use any 
method of contraception in both their first and most recent sexual activity.  
In the case that they did not use a condom, they are asked whether this is 
because they did not have access.  Based on this survey, access to condom is
calculated as an additional time-varying control.  However, it should be noted
that this variable can only be calculated at the level of the region (one 
level above the municipality), given that this survey is not representative at 
the level of the municipality.  Once again, processed data and processing 
scripts are made available at the data section of the author's site, and, if 
desired, raw data is available on the web: 
\url{http://extranet.injuv.gob.cl/Encuesta_Nacional_de_la_Juventud/contenido/index.php}.

\section{A Back of The Envelope Consistency Check of Effect Sizes}
\label{TEENscn:BOE}
Using the official Ministry of Health data on the number of pills distributed
in each year, we are able to determine whether the effect sizes identified in
this study seem to be of reasonable magnitude.  These calculations should of
course be taken as illustrative only, given that we do not know if all pills
distributed were taken by the recipient, nor the rates of pregnancy avoidance
conditional upon taking the pill.

According to the adminstrative medications data, 16,857 emergency 
contraceptive pills were prescribed (in total) in 2009, 2010 and 2011. Of 
these, 5,736 were prescribed to women 18 years of age and younger, while the
remaining 11,121 we prescribed to women over that age of 18.  In order to have
a rough idea of whether the estimates we find are reasonable, we can compare
the approximate reduction in pregnancy estimated from our preferred 
specification, with the number of pills given out over the period of interest.

Given that the Ministry of Health's administrative data on prescriptions only
records the ages of women accessing the pill as 18 and under and 19 and over,
\person estimate our specification for these two subgroups.  \Person also
calculate the total number of pregnancies in treated (and close to treated)
municipalities during the period in which the pill was available.  These 
figures are displayed in table \ref{TEENtab:BOE}.  In order to determine the
reduction in pregnancies which these estimates imply, \person compare the 
theoretical number of pregnancies without the pill, to the number recorded 
with the pill.  For example, in the case of the 18 and under group, the pill acts 
to reduce pregnancies by $1-\exp(-0.069)=0.067$, or 6.7\%.  So, we inflate the 
total number of pregnancies for this group (which was 20,713), suggesting that 
the total number of pregnancies without the pill would have been 22,612 (which
we calculate as $\frac{20,713}{1-0.067}$).  Thus, the approximate effect of the 
pill for this group is estaimated as a reduction of 22,200-20,713=1,487
pregnancies.  Similar calculations can be run for each subset, to calculate the
total estimated effect in each age group.

Based on this methodology, our estimates suggest that the pill accounted for 
3,212 fewer pregnancies in the 18 and under age group, and 11,742 fewer 
pregnancies in the 18 and over age group.\footnote{The full calculation for the 
18 and under group is:
\begin{equation}
\nonumber
\left(\frac{20,713}{1-0.067}-20,713\right)+
\left(\frac{10,370}{1-0.072}-10,370\right)+
\left(\frac{6,141}{1-0.048}-6,141\right)=2,596
\end{equation}
and a similar calculation for the 19 and over group gives 
\begin{equation}
\nonumber
\left(\frac{172,557}{1-0.032}-172,557\right)+
\left(\frac{100,749}{1-0.032}-100,749\right)+
\left(\frac{48,756}{1-0.013}-48,756\right)=9,525.
\end{equation}
}
Comparing these to total pill disbursements of 5,736 and 11,121, the estimated
effects seem to be quite close to actual data on pills acquired.  Although the
estimates are slightly higher than expected for the 19 and over group (implying
-0.86 births per pill dispursed) and perhaps slightly lower than expected for the
18 and under group (-0.45 births per pill dispursed), this back of the envelope 
consistency check performs remarkably well, and when considering the standard 
errors on our estimates, certainly falls within the margins that we would expect
given the number of pill requests.\footnote{Further, when considering that
pills may have be transferred between women who received the prescription and
women who ultimately took the pill, we may be more interested in overall rates
for both age groups.}

If \person instead compare the total pill disbursements over the period to the 
total estimated reduction in pregnancies,\footnote{It seems reasonable to make 
such a comparison given the the spillover effects estimated in this paper suggest 
that the person accessing the pill may not be the same as the person using the 
pill.}  this implies an efficiency rate of 71.9\% (or that 71.9\% of pills should 
result in an avoided pregnancy to account for the reduction in births.  For 
reference, the United States FDA reports an effectiveness rate of 89\% based on
typical usage.
%********************************************************************************
%\end{spacing}
%\begin{spacing}{1.4}

\newpage
\section{Appendix Figures}
\input{\pillloc/figures/Appendix_Figures.tex}

\section{Appendix Tables}
\input{\pillloc/tables/Appendix_Tables.tex}


%\appendix
%\section*{Online Appendix}
%\section{Data Agreement with Government of Chile}
%\includepdf[pages={-}]{\teenfolder/Data/DECLARACION_CONFIDENCIALIDAD.pdf}
